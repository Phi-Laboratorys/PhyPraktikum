% Charlotte Geiger - Manuel Lippert - Leonard Schatt
% Physikalisches Praktikum

% 5. Kapitel Fazit

\chapter{Fazit}
\label{chap:fazit}
Diese Auswertung w\"are ohne Transistoren so nicht m\"oglich. Die Verarbeitung von Informationen werden unter anderem auch \"uber Transistoren geleitet und sind wichtigster Bestandteil der Hardware. So finden sich Transistoren sowohl in der Analog- als auch in der Digitaltechnik und sind meist zu Logikgattern oder Flipflop-Speichern zusammengeschaltet und bilden einen integrierten Schaltkreis bzw.  Mikrochip. So merkt man, dass Transistoren die eingetippten Befehle verarbeiten und Daten und Ergebnisse berechnen und speichern k\"onnen. \\
In diesem Versuch haben wir Transistoren n\"aher kennengelernt und uns viel mit Kennlinien und Kenngr\"o\ss{}en besch\"aftigt. Zudem war die Dimensionierung von Transistorschaltungen und der Vergleich mit der Theorie dahinter gro\ss{}er Bestandteil des Versuchs. \\
Aus der Besonderheit und Vielfalt des Transistors und dem oben genannten daraus resultierenden allt\"aglichen Bedarf ist der Versuch super wichtig f\"ur uns, auch damit wir die Technik, die uns im Alltag begleiten nicht nur nutzen, sondern auch verstehen k\"onnen. 