\subsection{Ausgangswiderstand der Emitterschaltung}

Um den Ausgangswiderstand $r_a$ zu ermitteln schließen wir die Widerstandsdekade als Verbraucher an. Da wir $u_a$ kennen (Fall $R_V = \infty$ ) können wir 
$r_a$ auf sehr leicht Art und Weise berechnen. Da wir aus dem Ohm’schen Gesetz wissen, dass bei $R_V = r_a$ die halbe Ausgangsspannung über die Widerstandsdekade 
abfällt folgt, müssen wir nur die Widerstandsdekade so einstellen, dass $u_{R_V} = \frac{u_a}{2}$. Der Fehler setzt sich dabei aus dem Fehler von $R_C$ und dem Fehler der Widerstandsdekade
und dem Ablesefehler zusammen. Der Ablesefehler dominiert hier alle anderen, da es sehr schwer abzulesen war, wann die Spannung genau gleichgroß war. Wir schätzen ihn auf circa 10\%. 
\begin{table}[h!]
    \centering
    \begin{tabular}[h]{c|c||c}
        $r_{a_{gemessen}}/\Omega $ & $s_{r_{a{gemessen}}}/\Omega $ & $r_{a_{gemessen}}/$k$\Omega$\\
       \hline
      
        2500 & 252.3 & \textcolor{red}{2.5$\pm $ 0.3}\\
    \end{tabular}
    \caption{Ausgangswiderstand der Emitterschaltung}
\end{table}
Der gemessene Wert liegt deutlich unter dem errechneten Wert. Dies liegt aber vermutlich 
nur an dem großen Fehler, welcher durch das ungenaue Ablesen entsteht. Der errechnete Wert 
liegt noch in dem Fehlerintervall des gemessenen Wertes.