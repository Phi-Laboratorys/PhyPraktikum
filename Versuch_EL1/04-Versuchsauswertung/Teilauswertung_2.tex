% Charlotte Geiger - Manuel Lippert - Leonard Schatt
% Physikalisches Praktikum

% Teilauswertung 2

\section{Ausgangskennlinie des Transistors}
Nun werten wir die Ausgangskennlinie aus. Daf\"ur plotten wir an unsere gemessenen Kennlinien approximierte Funktionen der verschiedenen Basisstr\"ome. Die analytische Funktion hierf\"ur hat die Form:

%bitte folgende Funktion texen \alphaarctan(\betax)
%bitte Plot f\"ur das Ausgangskennlinienfeld einf\"ugen. 

Hierbei kann man gut erkennen, ....\\
Zudem sollen wir den differentiellen Transistor-Ausgangswiderstand
\begin{equation}
r_{CE}=\frac{\partial{U_{CE}}}{\partial{I_C}}\Bigg|_{I_B=const}
\end{equation}
berechnen. \\
Zuletzt sollen wir noch die DC-Verst\"arkung $B=\frac{I_C}{I_B}$ berechnen. %konntet ihr hier bitte die Tabelle (Nutation_in Auswertung_Kre) und den Plot (omega_v gegen omega_3) einfügen 

%bitte Tabelle (Aufgabe 6.2 DC-Verstärkung) und Plot B(I_B) einfügen

Aus dieser Abbildung erkennt man einen leicht...Das kann man dadurch erkl\"aren, dass...\\

%TODO #8 @ManeLippert