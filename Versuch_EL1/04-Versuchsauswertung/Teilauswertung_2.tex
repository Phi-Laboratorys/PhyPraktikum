% Charlotte Geiger - Manuel Lippert - Leonard Schatt
% Physikalisches Praktikum

% Teilauswertung 2

\section{Ausgangskennlinie des Transistors}
Um die Ausgangskennlinie zu skizzieren verwenden wir die Bilder des Oszilloskops (Kapitel \ref{sec:zusatz}: Zusatz) und nehmen an, dass außerhalb der Anzeige des Oszilloskops die Linie linear fortgeführt werden kann. 
\begin{center}
    \captionof{figure}{$U_{BE}-I_C$ Diagramm - Ausgangskennlinie Transistor}
    \includegraphics[scale=0.2]{6_2-Ausgangskennlinien.PNG}
\end{center}
Wobei die Datenpunkte für $P_{\text{v,max}}$-Kurve so berechnet wurden, dass für unterschiedliche $U_{CE}$ die maximale Kollektorstrom $I_C$ berechnet wurden mit
\begin{align}
    I_C=\frac{P_{\text{v,max}}}{U_{CE}} \tab\text{mit}~P_{\text{v,max}}=300~\text{mW}
\end{align}
\begin{center}
    \captionof{table}{$P_{\text{v,max}}$-Kurve}
    \begin{tabular}{l|rrrrrrrrrrrr}
        $U_{CE}/\text{V}$ & 1.8   & 2     & 2.2   & 2.4   & 2.6   & 2.8   & 3     & 3.2   & 3.4   & 3.6   & 3.8   & 4 \\
        \hline
        $I_C/\text{mA}$  & 167 & 150 & 136 & 125 & 115 & 107 & 100 & 94 & 88 & 83 & 79 & 75 \\
    \end{tabular}
\end{center}
Hierbei kann man gut erkennen, ....\\
Zudem sollen wir den differentiellen Transistor-Ausgangswiderstand berechnet werden.
\begin{equation}
r_{CE}=\frac{\partial{U_{CE}}}{\partial{I_C}}\Bigg|_{I_B=const}
\end{equation}
berechnen. \\
Zuletzt sollen wir noch die DC-Verst\"arkung $B=\frac{I_C}{I_B}$ berechnen. Dafür lesen wir $I_C$ für eine feste Spannung $U_{CE}=0.6~\text{V}$ aus den Bilder des Oszilloskops. Dabei wird $U_{CE}$ so gewählt, dass $I_C$ im linearen Bereich der Ausgangskennlinie beim Arbeitspunkt abgelesen wird. Weiterhin nehmen wir einen großzügigen Fehler für $I_C$ von $s_{I_C}=2.5~\text{mA}$ an, also ein halbes Kästchen auf dem Oszilloskop.  Daraus folgt:
\begin{align}
    s_B=\frac{s_{I_C}}{I_B}
\end{align}
Der Fehler des Basisstroms $s_{I_B}$ wird vernachlässigt, da dieser klein gegenüber den gewählten Fehler $s_{I_C}$ ist und damit vernachlässigbar.
\begin{center}
    \captionof{table}{DC-Verstärkung}
    \begin{tabular}{ccccc}
        $I_B/\text{mA}$ & $I_C/\text{mA}$ & $s_{I_C}/\text{mA}$ & $B$ & $s_B$ \\
        \hline
        0     & 0     & -   & -   & - \\
        0.05  & 13    & 2.5   & 260   & 50 \\
        0.10   & 27    & 2.5   & 270   & 25 \\
        0.20   & 39    & 2.5   & 195   & 12.5 \\
        0.30   & 49    & 2.5   & 163   & 8 \\
        0.40   & 58    & 2.5   & 145   & 6 \\
        0.50   & 63    & 2.5   & 126   & 5 \\
        \end{tabular}
\end{center}
\begin{center}
    \captionof{figure}{$I_B-B$ Diagramm}
    \includegraphics[scale=0.6]{6_2-Verstaerkung.png}
\end{center}
Aus dieser Abbildung erkennt man einen leicht...Das kann man dadurch erkl\"aren, dass...\\