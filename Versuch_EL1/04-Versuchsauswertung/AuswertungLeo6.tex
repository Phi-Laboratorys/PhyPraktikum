\subsection{Eingangswiderstand der Emitterschaltung} \label{Eingspa}

Der Eingangswiderstand und sein Fehler lassen sich analog zu den Fragen zur Vorbereitung wieder berechnen durch:
\begin{equation*}
    r_e = [R_{1}^{-1}+R_{2}^{-1}+\frac{1}{r_{BE}+\beta R_E}]^{-1}
\end{equation*}
\begin{equation*}
    s_{r_e} = | r_e |^2 \sqrt{(\frac{s_{R_1}}{R_1^2})^2+(\frac{s_{R_2}}{R_2^2})^2+(\frac{s_{r_{BE}}}{(r_{BE}+\beta R_E)^2} )^2} 
\end{equation*}
Der letzte Term in der Fehlerrechnung geht jedoch für $R_E$ gegen unendlich gegen Null.
Das ergibt wie in Kapitel %@ManeLippert TODO 
berechnet:

\begin{equation*}
    \textcolor{red}{r_e \approx	 (16.5 \pm 0.2)k\Omega}
\end{equation*}
Die Messung wird ähnlich wie in der vorherigen Aufgebe gemacht. Diesmal wird der Widerstand jedoch davor in Reihe geschaltet. Der Fehler wir wieder durch unseren geschätzten 
Ablesefehler dominiert. Dieser liegt hier bei circa 20\%. Die anderen Fehler sind im Vergleich dazu vernachlässigbar.
\begin{table}[h]
    \centering
    \begin{tabular}{c|c||c}
        $r_{e_{gemessen}}$ in k$\Omega$ & $s_{r_{e_{gemessen}}}$ in k$\Omega$ & Ergebnis\\
        \hline
        20 & 4 & \textcolor{red}{(20$\pm $ 4)k$\Omega$}
    \end{tabular}
    \caption{Gemessener Eingangswiederstand}
\end{table}

Der gemessene und der theoretische Wert passen gut zusammen. Es fällt auf, dass der gemessene Wert deutlich 
über dem errechneten liegt. Dies ist jedoch dadurch zu erklären, dass der Fehler der Messung sehr groß ist. 
Der errechnete Wert liegt im Messfehler.
