\subsection{Gegenkopplung durch das Verbauen unterschiedlicher $R'_E$}

Schließlich betrachten wir einen algemeineren Fall als das, was wir bis jetzt betrachtet haben. 
Bis jetzt ist $R'_E = \infty$. In diesem Fall verändern wir $R'_E$. Im Grenzfall mit $R'_E$ erwarten wir 
jedoch wieder das selbe wie in Kapitel \glqq \ref{Eingspa}  \titleref{Eingspa}".\\
Im folgenden berechnen wir die erwarteten Werte:\\
\begin{equation*}
    \centering
    r'_e = [\frac{1}{R_1}+\frac{1}{R_2}+\frac{1}{r_{BE}+\beta(\frac{1}{R_E}+\frac{1}{R_{E'}})}]^{-1}
\end{equation*}
\begin{equation*}
    \centering
    s_{r'_e} = |r_e|^2 \sqrt{(\frac{s_{R_1}}{R_1^2})^2+(\frac{s_{R_2}}{R_2^2})^2+[r_{BE}+\frac{\beta(R_E+R'_E)}{R_E \cdot R'_E}]^{-4} \cdot [s_{r_{BR}}^2+\frac{\beta^2 s_{R_E}^2}{(R_{E})^4}+(\frac{\beta \cdot s_{R'_E}}{(R'_{E})^2})^2]}
\end{equation*}
Für $R'_E \to \infty$ sind wir im vorherigen Kapitel. Auch die Messwerte sind dabei identisch.\\
Als Fehler der Messung nehmen wir wieder nur den geschätzten Fehler, da dieser viel größer ist, als die anderen.
\begin{table}[h]
    \centering
    \begin{tabular}{r|c|c|c|c||c||c}
         $n$& $r'_{e,t,n}$ in $\Omega $ & $s_{r'_{e,t,n}}$ in $\Omega $ & $r'_{e,m,n}$ in $\Omega$ & $s_{r'_{e,m,n}}$ in $\Omega $& Ergebnis $r'_t$ & Ergebnis $r'_m$\\
        \hline
        68$\Omega$ &
        150$\Omega$
        $\infty\Omega$
    \end{tabular}
\end{table}