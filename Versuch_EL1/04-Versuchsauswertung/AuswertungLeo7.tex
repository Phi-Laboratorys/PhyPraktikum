\subsection{Gegenkopplung durch das Verbauen unterschiedlicher $R'_E$}

Schließlich betrachten wir einen algemeineren Fall als das, was wir bis jetzt betrachtet haben. 
Bis jetzt ist $R'_E = \infty$. In diesem Fall verändern wir $R'_E$. Im Grenzfall mit $R'_E$ erwarten wir 
jedoch wieder das selbe wie in Kapitel \glqq \ref{Eingspa}  \titleref{Eingspa}".\\
Im folgenden berechnen wir die erwarteten Werte:\\
\begin{equation*}
    \centering
    r'_e = [\frac{1}{R_1}+\frac{1}{R_2}+\frac{1}{r_{BE}+\beta(\frac{1}{R_E}+\frac{1}{R'_{E}})^{-1}}]^{-1}
\end{equation*}
\begin{equation*}
    \centering
    s_{r'_e} = |r_e|^2 \sqrt{(\frac{s_{R_1}}{R_1^2})^2+(\frac{s_{R_2}}{R_2^2})^2+[r_{BE}+\frac{\beta(R_E+R'_E)}{R_E \cdot R'_E}]^{-4} \cdot [s_{r_{BR}}^2+\frac{\beta^2 s_{R_E}^2}{(R_{E})^4}+(\frac{\beta \cdot s_{R'_E}}{(R'_{E})^2})^2]}
\end{equation*}
Für $R'_E \to \infty$ sind wir im vorherigen Kapitel. Auch die Messwerte sind dabei identisch.\\
Aus dem Formelblatt \footnote{\url{https://docs-emea.rs-online.com/webdocs/0109/0900766b80109ff9.pdf}} entnehmen wir die fü die Rechnung nötigen Daten.
Als Fehler der Messung nehmen wir wieder nur den geschätzten Fehler, da dieser viel größer ist, als die anderen.
\begin{table}[h]
    \centering
    \begin{tabular}{r|c|c|c|c||c||c}
         $n$& $r'_{e,t,n}$ in $\Omega $ & $s_{r'_{e,t,n}}$ in $\Omega $ & $r'_{e,m,n}$ in $\Omega$ & $s_{r'_{e,m,n}}$ in $\Omega $& Ergebnis $r'_t$ & Ergebnis $r'_m$\\
        \hline
        68$\Omega$ & 9920.2762 & 821.359 & 10000 & 2000& \textcolor{red}{(9.9$\pm$0.8)k$\Omega$}& \textcolor{red}{(10$\pm $ 2)k$\Omega$}  \\
        150$\Omega$ &12260.0619 & 1254.500  &22000 & 4400& \textcolor{red}{12$\pm$1)k$\Omega$}& \textcolor{red}{(22$\pm $ 4)k$\Omega$}  \\
        $\infty\Omega$& Siehe & Kapitel& \ref{Eingspa} &  & \textcolor{red}{(16.5 $\pm$ 0.2)k$\Omega$}& \textcolor{red}{(20$\pm $ 4)k$\Omega$}  \\
    \end{tabular}
    \caption{}
\end{table}
Der Messwert bei dem 150$\Omega$ ist eindeutig ein Messfehler. Anders können wir uns einen so hohen Wert nicht erklären. Die anderen beiden Werte passen sehr gut 
zu den theoretisch ermittelten Werten. Bei den theoretischen Werten fallen die großen Fehler auf. Dies lässt sich jedoch durch die Dominanz dieses Thermes im Fehler erklären:
\begin{equation*}
    [r_{BE}+\frac{\beta(R_E+R'_E)}{R_E \cdot R'_E}]^{-4}
\end{equation*}

\subsubsection{Verstärkung mit Gegenkopplungswiderständen}
Die Verstärkung wird aus dem Messwerten wie in den vorherigen Kapiteln berechnet.
Die Verstärkung\footnote{Gleichung (6) im Skript El1} ergibt sich folgendermaßen: 
\begin{equation*}
    v_t = -R_C[\frac{1}{R_E}+\frac{1}{R'_E}] 
\end{equation*}
\begin{equation*}
    s_v = |v_t| \sqrt{(\frac{s_{R_C}}{R_C})^2+(\frac{s_{R_E}}{R_E})^2+(\frac{s_{R'_E}}{R'_E})^2}
\end{equation*}
Mit diesem Formel folgen folgende Werte:\\
\begin{table}[h]
    \centering
    \begin{tabular}{c|c|c}
        Verstärkung & Fehler & Ergenis\\
        \hline
        $v_{m,68}$ = 36 & $s_{v_{m,68}}$ = 1.846& \textcolor{red}{36$\pm$2}\\
        $v_{m,150}$ = 7 & $s_{v_{m,150}}$ = 0.37& \textcolor{red}{7.0$\pm$0.3}\\
        $v_{m,\infty}$ = 5.0 & $s_{v_{m,68}}$ = 0.2695& \textcolor{red}{5.0$\pm$0.3}\\
        \hline
        $v_{t,68}$ = 49.1176 & $s_{v_{t,68}}$ = 2.552227& \textcolor{red}{49$\pm$2}\\
        $v_{t,150}$ = 25.000 & $s_{v_{t,68}}$ = 1.29903& \textcolor{red}{25$\pm$1}\\
        $v_{t,\infty}$ = 5.000 & $s_{v_{t,infty}}$ = 0.25988& \textcolor{red}{5.0$\pm$0.2}\\
    \end{tabular}
\end{table}
Der Wert für den 150$\Omega$ Widerstand ist wie erwartet nicht übereinstimmend. Die anderen Werte passen in der 
Gößenordnung, wobei der untere Wert sehr gut mit dem theoretischen Wert übereinstimmt. Bei großen Werten 
scheint die Theorie nicht mehr ganz mit der Beobachtung übereinzustimmen.