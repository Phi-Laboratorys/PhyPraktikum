\subsection{Wechslespannungsverstärkung im Vergleich}

Wir tragen die Verstärkung in einem Diagramm auf, da uns aus den vorherigen Beobachtungen klar ist, dass die Verstärkung nicht linear ist. Die Fehler von $u_a$
nehmen wir im Protokoll vermerkt an, also für das Oszilloskop 3\% des gemessen Wertes und der Ablesefehler beträgt 0.1 Unterteilungen.
\begin{table}[h]
    \centering
    \begin{tabular}[h]{c|c|c|c|c|c}


        $u_e$ in mV & $u_A$ in V & $v${A} & $s_{u_e}$ in mV&  $s_{u_a}$ in V & $s_v$ \\
       \hline
       %Nicht in der Literatur  & 3135 & sdfgsdf \\
        0 & 0 & - & 0.
       %Nicht in der Literatur  & 8100 & asdf \\
       %Nicht in der Literatur  & 8730 & asdf \\
    \end{tabular}
\end{table}
