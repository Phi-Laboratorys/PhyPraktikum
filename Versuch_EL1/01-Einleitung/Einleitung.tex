% Charlotte Geiger - Manuel Lippert - Leonard Schatt
% Physikalisches Praktikum

% 1. Kapitel Einleitung

\chapter{Einleitung}
\label{chap:einleitung}

Die Kommunikation der Welt beruht auf Apps und Features auf Handys und Laptops immer und überall erreichbar sein für jeden und überall Informationen abrufen können, das sind die Anforderungen an die Technik heutzutage. Möglich ist dies nur durch die kleinen und leichten Geräte, die es derzeit auf dem Markt gibt. Ein zentraler Bauteil der Mikroelektronik der Geräte ist der Transistor. Dieser ermöglicht es, ohne mechanischen Prozessen, Stromflüsse innerhalb einer elektrischen Schaltung zu kontrollieren. Das ist der Grund, weshalb man Millionen von Transistoren in Beispielsweise Smartphones vorfindet. Der dauerhafte alltägliche Nutzen verdeutlicht die Wichtigkeit, die dieses elektronische Bauteil einnimmt und den Grund, weshalb der Versuch so wichtig ist. In dem Versuch eignen wir uns diesen Baustein anhand von selbstgesteckten Schaltungen an und erlernen unter anderem anhand von der Dimensionierung mit unterschiedlichen Schaltungen damit umzugehen.