% Charlotte Geiger - Manuel Lippert - Leonard Schatt
% Physikalisches Praktikum

% Main-Datei für die Auswertung in TeX

% Struktur:
% Für jeden Abschnitt gibt es einen Ordner, damit jeder individuell an seinen Aufgaben arbeiten
% kann, ohne beim merge in GitHub Konflikte zu erhalten. Deshalb werden alle Unteraufgaben auch 
% extra in Ordner angelegt. Die einzelnen Dateien über den input Befehl einfügbar.
% Bilder und andere Grafik werden im Ordner Grafik abgelegt 


% Packages
\documentclass[paper=a4,bibliography=totoc,BCOR=10mm,twoside,numbers=noenddot,fontsize=11pt]{scrreprt}
\usepackage[ngerman]{babel}
\usepackage[T1]{fontenc}
\usepackage[latin1, utf8]{inputenc}
\usepackage{lmodern}
\usepackage{graphicx}
\usepackage{nicefrac}
\usepackage{fancyvrb}
\usepackage{amsmath,amssymb,amstext}
\usepackage{siunitx}
\usepackage{url}
\usepackage{natbib}
\usepackage{microtype}
\usepackage[format=plain]{caption}
\usepackage{physics}
\usepackage{titleref}

% Zusätzliche Packages
\usepackage{geometry}
\usepackage{anyfontsize}
\usepackage[table]{xcolor}
\usepackage{ifthen}
\usepackage[absolute,overlay]{textpos}
\usepackage{amsfonts}
\usepackage{xstring}
\usepackage{tikz}
\usepackage{pdfpages}

% Abschnittseinrückung und -abstand
% Die folgenden Zeilen sollen möglichst nicht verändert werden
\parindent 0.0cm
\parskip 0.8ex plus 0.5ex minus 0.5ex

% Anzahl und Größe von Gleitobjekten
% maximal 2 Objekte oben und unten
% erlaubt auch größere Bilder, welche die ganze Seite benötigen
% Die folgenden Zeilen sollen möglichst nicht verändert werden
\setcounter{bottomnumber}{2}
\setcounter{topnumber}{2}
\renewcommand{\bottomfraction}{1.}
\renewcommand{\topfraction}{1.}
\renewcommand{\textfraction}{0.}

%\sc und \bc veraltet. Daher: (20.09.2018)
\DeclareOldFontCommand{\sc}{\normalfont\scshape}{\@nomath\sc}
\DeclareOldFontCommand{\bf}{\normalfont\scshape}{\textbf}

% Verschiedenes
\pagestyle{headings}          % Der Seitenstil sollte möglichst nicht verändert werden
\graphicspath{{./bilder/}}    % Der Pfad für die Abbildungen Abbildungen wird gesetzt
\VerbatimFootnotes            % \verb etc. auch in \footnotes mφglich

% Funktionen
\newcommand\tab[1][1cm]{\hspace*{#1}}
\newcommand{\vect}[1]{\boldsymbol{\mathbf{#1}}}
\newcolumntype{g}{>{\columncolor[rgb]{ .741,  .843,  .933}}l}
% Tiefgestellte Zahlen nicht kursiv
\catcode`_=\active
\newcommand_[1]{\ensuremath{\sb{\mathrm{#1}}}}

\begin{document}

    \nonfrenchspacing

    % 0. Kapitel Cover
    % Charlotte Geiger - Manuel Lippert - Leonard Schatt
% Physikalisches Praktikum

% 0. Cover
% Noch abänderbar nur ein Vorschlag
\newgeometry{top=30mm, bottom=20mm, inner=20mm, outer=20mm}
\thispagestyle{empty}

% Colors
\definecolor{Notablue}{HTML}{3498DB}		%Theoretische Physik
\definecolor{Notared}{HTML}{CF366C}			%Mathematik
\definecolor{Notagreen}{HTML}{19B092}		%Experimentalphysik
\definecolor{Notaorange}{HTML}{FA9D00}		%Chemie/Wahlfach nicht physikalisch
\definecolor{Notagrey}{HTML}{969696}		%Praktikum
\definecolor{Notalavendel}{HTML}{9DBBD8}	%Wahlfächer physikalisch

% Boolean by default false
\newboolean{twoRows}
\newboolean{symbol}

% Funktions
\makeatletter
   \def\vhrulefill#1{\leavevmode\leaders\hrule\@height#1\hfill \kern\z@}
\makeatother
\newcommand*\ruleline[1]{\par\noindent\raisebox{.8ex}{\makebox[\linewidth]{\vhrulefill{\linethickness}\hspace{1ex}\raisebox{-.8ex}{#1}\hspace{1ex}\vhrulefill{\linethickness}}}}

% Variables
\def\schriftgrosse{70}
\def\linethickness{1,5pt}

\def\farbe{Notagrey}
\def\fach{PPBphys1}
\def\name{Charlotte Geiger - Manuel Lippert - Leonard Schatt}
\def\uberschrift{Protokoll Titel} % Absatz mit \\[0,5cm]; u = Übung, k = Klausur; s = Skript, e = Ergebnis
\def\bottom{SS2021}
\def\datum{28.04.2021}
\def\platz{NWII | RaumNr.}
\def\betreuer{Moritz Heindl}
\def\auswertp{Manuel Lippert}
\def\messp{Leonard Schatt}
\def\protop{Charlotte Geiger}
\def\groupnr{4}

\begin{titlepage}
			
	\centering
	{\LARGE \sffamily {\textbf{\bottom}\par}}
	\vspace{2,5cm}
    {\fontsize{40}{0}\sffamily\ruleline{\textcolor{\farbe}{\textbf{\fach}}}\par}
    \vspace{6cm}
	{\Large\sffamily \ruleline{\name}\par}
	
	
	% Choose Text
	\ifthenelse{\equal{\uberschrift}{s}} {\def\titel{Skript}}	
		{\ifthenelse{\equal{\uberschrift}{k}} {\def\titel{Klausur}}
			{\ifthenelse{\equal{\uberschrift}{u}} {\def\titel{Übung}}
				{\ifthenelse{\equal{\uberschrift}{e}} {\def\titel{Klausur \\[0,5cm] Ergebnis}}
					{\def\titel{\uberschrift}}
				}
			}
		}
	
	\begin{textblock*}{21cm}(0cm,9cm) % {block width} (coords), centered		
		{\fontsize{\schriftgrosse}{0}\sffamily\textcolor{\farbe}{\textbf{\titel}}\par}
	\end{textblock*}
	
	% Choose Logo
	\ifthenelse {\equal{\farbe}{Notared}} {\def\logo{Bilder/Logo/UniBTNotared}}
		{\ifthenelse {\equal{\farbe}{Notagreen}} {\def\logo{Bilder/Logo/UniBTNotagreen}}
			{\ifthenelse {\equal{\farbe}{Notablue}} {\def\logo{Bilder/Logo/UniBTNotablue}}
				{\ifthenelse {\equal{\farbe}{Notaorange}} {\def\logo{Bilder/Logo/UniBTNotaorange}}
					{\ifthenelse {\equal{\farbe}{Notagrey}} {\def\logo{Bilder/Logo/UniBTNotagrey}}
						{\ifthenelse {\equal{\farbe}{Notalavendel}} {\def\logo{Bilder/Logo/UniBTNotalavendel}}	
							{\ifthenelse {\equal{\farbe}{black}} {\def\logo{Bilder/Logo/UniBT}}	
								{\def\logo{noLogo}}
							}
						}
					}
				}
			}
		}	

	\IfSubStr{\logo}{noLogo}{\setboolean{symbol}{false}}{\setboolean{symbol}{true}}
	
	% Gruppe
	\vspace{10cm}
	{\large\sffamily{Gruppe \groupnr}}
	
	%Logo
	\vfill

	\ifthenelse{\boolean{symbol}}
		{
			\begin{figure}[h]
			\begin{center}
				
				\includegraphics[width=2cm]{\logo}
				
			\end{center}
			\end{figure}
		}
	
\end{titlepage}

\restoregeometry

% Information
\chapter*{Informationen}
\label{chap:info}

\begin{tabular}{l l}

	{\textbf{Versuchstag}} \hspace{1cm} & \hspace{1cm} {\datum}\\[0,2cm]
	{\textbf{Versuchsplatz}} \hspace{1cm} & \hspace{1cm} {\platz}\\[0,2cm]
	{\textbf{Betreuer}} \hspace{1cm} & \hspace{1cm} {\betreuer}\\[1,2cm]
	{\textbf{Gruppen Nr.}} \hspace{1cm} & \hspace{1cm} {\groupnr}\\[0.2cm]
	{\textbf{Auswertperson}} \hspace{1cm} & \hspace{1cm} {\auswertp}\\[0.2cm]
	{\textbf{Messperson}} \hspace{1cm} & \hspace{1cm} {\messp}\\[0.2cm]
	{\textbf{Protokollperson}} \hspace{1cm} & \hspace{1cm} {\protop}\\[0.2cm]

\end{tabular}

    \thispagestyle{empty}
    \cleardoublepage
    \tableofcontents
    \cleardoublepage

    % 1. Kapitel Einleitung
    % Charlotte Geiger - Manuel Lippert - Leonard Schatt
% Physikalisches Praktikum

% 1. Kapitel Einleitung

\chapter{Einleitung}
\label{chap:einleitung}
Polarisiertes Licht ist in unserem Alltag überall zu bemerken. In unserem direkten Umfeld sind zum Beispiel Sonnenbrillen und Computerdisplays polarisiert, aber auch in der Tierwelt wird es viel genutzt. So nutzen Bienen, Heuschrecken und manchen Fledermausarten die Lichtpolarisation für die Orientierung. 
%www.br.de/wissen/polarisation-polsarisiertes-licht-100
Aber was kann man darunter überhaupt verstehen? Wie und wodurch entsteht die Polarisation? Mit diesen Fragen zu und den Eigenschaften von linear polarisiertem Licht bezüglich Reflexion und Streuung beschäftigen wir uns in diesem Versuch. 
% Text

    % 2.Kapitel Fragen zur Vorbereitung
    % Charlotte Geiger - Manuel Lippert - Leonard Schatt
% Physikalisches Praktikum

% 2.Kapitel Fragen zur Vorbereitung

\chapter{Fragen zur Vorbereitung}
\label{chap:fvz}

\section{Teilaufgabe 1}
\label{sec: }

    % 3.Kapitel Protokoll
    % Charlotte Geiger - Manuel Lippert - Leonard Schatt
% Physikalisches Praktikum

% 3.Kapitel  Protokoll

% Variables
\def\skalierung{0.65}

\chapter{Messprotokoll}
\label{chap:protokoll}

Das Messprotokoll wurde am Versuchstag handschriftlich erstellt und hier als
PDF-Datei eingefügt. Dabei wurden Durchführung und Aufbau schon vorher in dieses
Dokument beschrieben, je nachdem.

% Einbindung des Protokolls als pdf (mit Seitenzahl etc.)
% Erste Seite mit Überschrift
%\includepdf[pages = 1, landscape = false, nup = 1x1, scale = \skalierung , pagecommand={\thispagestyle{empty}\chapter{Protokoll}}]
%            {03-Protokoll/Protokoll.pdf}
% Restliche Seiten richtig skaliert
%\includepdf[pages = -, landscape = false, nup = 1x1, scale = \skalierung , pagecommand={}]
%            {03-Protokoll/Protokoll.pdf}

    % 4.Kapitel Versuchsauswertung
    % Charlotte Geiger - Manuel Lippert - Leonard Schatt
% Physikalisches Praktikum

% 4.Kapitel Versuchsauswertung

\chapter{Auswertung und Diskussion}
\label{chap:versuchsauswertung}

% Text

% Input der Teilauswertung je nach Produktion der Nebendateien ohne Ordner
% Charlotte Geiger - Manuel Lippert - Leonard Schatt
% Physikalisches Praktikum

% Teilauswertung X

\section{Teilauswertung X}

% etc.

    % 5.Kapitel Fazit
    % Charlotte Geiger - Manuel Lippert - Leonard Schatt
% Physikalisches Praktikum

% 5. Kapitel Einleitung

\chapter{Fazit}
\label{chap:fazit}


% Platz für Text

    % Anhang
    % Charlotte Geiger - Manuel Lippert - Leonard Schatt
% Physikalisches Praktikum

% Anhang

\appendix

% Text

% Charlotte Geiger - Manuel Lippert - Leonard Schatt
% Physikalisches Praktikum

% Anhang A

\chapter{Append A}
\label{chap:anhangA}

\section{Teilanhang X}


    % Literatur
    \bibliographystyle{Auswertung.bst}
    \nocite{*}
    \bibliography{Auswertung.bib}

\end{document}