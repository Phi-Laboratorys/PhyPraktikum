\section{Vergleich der Verhältnisse der gemessenen Schwingungsfrequenzen mit theoretisch zu erwartenden Werten}
Für die theoretische Schwingungsfrequenz gilt folgende Formel \citep[vgl.][]{Schwingungsfrequenz}:
\begin{equation}
    \nu=\frac{1}{2\pi}\sqrt{\frac{k}{\mu}}
\end{equation}
Hierbei steht $k$ die Federkonstante zwischen den Kohlenstoffatom und dem Schwingungspartner.\\
$\mu$ steht für die reduzierte Masse, welche wie folgt definiert ist:
\begin{align}
    \mu =\frac{m_1\cdot m_2}{m_1+m_2}
\end{align}
Für das Verhältnis von zwei Schwingungen (mit gleicher Federkonstante) gilt:
\begin{equation}
    \frac{\nu_1}{\nu_2}=\sqrt{\frac{\mu_2}{\mu_1}}=\frac{\tilde{\nu}_1}{\tilde{\nu}_2}
\end{equation}
Wir wollen nun die Schwingungsfrequenzen (bzw. Wellenzahlen) von Molekülen vergleichen, bei denen je eine Atomart ausgetauscht wurde.
Es werden hier der Unterschied zwischen den Schwingungen $CHCl_3$ und $CDCl_3$ \& $CHCl_3$ und $CHBr_3$ angeschaut.
Mit den Massen der einzelnen Atome (Chlor, Wasserstoff, Deuterium, Kohlenstoff) wurden folgende reduzierte Massen berechnet \citep[vgl.][]{Massen}:
\begin{align}
    \mu_{C-H}&=0,92974\,\text{u}\\
    \mu_{C-D}&=1,72464\,\text{u}\\
    \mu_{C-Br}&=10,4162\,\text{u}\\
    \mu_{C-Cl}&=8,93414\,\text{u}
\end{align}
Für die theoretischen Schwingungsfrequenzen folgt nun:
\begin{align}
    \frac{\tilde{\nu}_{C-H}}{\tilde{\nu}_{C-D}}&=1,362\\
    \frac{\tilde{\nu}_{C-Cl}}{\tilde{\nu}_{C-Br}}&=1,079762594
\end{align}
Um nun das Verhältnis unserer gemessenen Werte zu berechnen, werden die jeweils gemessenen Wellenzahlen für $0^\circ$-Polarisation, $90^\circ$-Polarisation, Stokes- und Anti-Stokes-Linie gemittelt und anschließend dann das Verhältnis bestimmt.\\
Somit erhält man beim Vergleich von $CHCl_3$ zu $CDCl_3$:
\begin{table}[h]
    \centering\begin{tabular}{c|c|c}
        $\tilde{\nu}_{CHCl_3}$ (cm$^{-1}$) & $\tilde{\nu}_{CDCl_3}$ (cm$^{-1}$) & $\tilde{\nu}_{CHCl_3}/\tilde{\nu}_{CDCl_3}$\\\hline
        $262,43\pm3,2$ &$258,73\pm3,2$&$1,014\pm0,018$\\
        $363,45\pm3,2$ &$362,80\pm3,2$&$1,002\pm0,012$\\
        $668,70\pm2,8$ &$651,87\pm2,8$&$1,026\pm0,006$\\
        $763,20\pm2,8$ &$745,95\pm2,3$&$1,023\pm0,005$\\
        $1219,57\pm2,8$&$922,60\pm2,8$&$1,322\pm0,005$%\\
        %3018,9&2250,0&1,342
    \end{tabular}
    \caption{Verhältnis der gemessenen Werte von $CHCL_3$ zu $CDCL_3$}
\end{table}\newpage
Und beim Vergleich von $CHCl_3$ zu $CHBr_3$:
\begin{table}[h]
    \centering\begin{tabular}{c|c|c}
        $\tilde{\nu}_{CHCl_3}$ (cm$^{-1}$) & $\tilde{\nu}_{CHBr_3}$ (cm$^{-1}$) & $\tilde{\nu}_{CHCl_3}/\tilde{\nu}_{CHBr_3}$\\\hline
        $262,43\pm3,2$ &$154,63\pm3,2$&$1,697\pm0,041$\\
        $363,45\pm3,2$ &$221,53\pm3,2$&$1,641\pm0,028$\\
        $668,70\pm2,8$ &$541,10\pm2,8$&$1,236\pm0,008$\\
        $763,20\pm2,8$ &$655,90\pm2,8$&$1,164\pm0,007$\\
        $1219,57\pm2,8$&$1138,90\pm2,3$&$1,071\pm0,003$
    \end{tabular}
    \caption{Verhältnis der gemessenen Werte von $CHCl_3$ zu $CHBr_3$}
\end{table}\\
Für die Literaturwerte folgt:
\begin{table}[h]
    \centering\begin{tabular}{c|c|c}
        $\tilde{\nu}_{{CHCl_3}_l}$ (cm$^{-1}$) & $\tilde{\nu}_{{CDCl_3}_l}$ (cm$^{-1}$) & $\tilde{\nu}_{{CHCl_3}_l}/\tilde{\nu}_{{CDCl_3}_l}$\\\hline
        260,0&262,0&0,992\\
        365,9&366,5&0,998\\
        668,3&650,8&1,027\\
        761,2&737,6&1,032\\
        1215,6&908,3&1,338\\
        3018,9&2250,0&1,342
    \end{tabular}
    \caption{Verhältnis der aus der Literatur entnommen Werte}
\end{table}
\begin{table}[h]
    \centering\begin{tabular}{c|c|c}
        $\tilde{\nu}_{{CHCl_3}_l}$ (cm$^{-1}$) & $\tilde{\nu}_{{CHBr_3}_l}$ (cm$^{-1}$) & $\tilde{\nu}_{{CHCl_3}_l}/\tilde{\nu}_{{CHBr_3}_l}$\\\hline
        260,0&153,8&1,691\\
        365,9&222,3&1,646\\
        668,3&538,5&1,241\\
        761,2&656,0&1,160\\
        1215,6&1142,0&1,064\\
        3018,9&3023,0&0,999
    \end{tabular}
    \caption{Verhältnis der aus der Literatur entnommen Werte}
\end{table}\\
Die gemessenen Werte für das Verhältnis passen mit den Literaturwerten überein.
Für beide Messreihen gilt, je größer die Wellenzahl ist, desto genauer ist man an dem theoretischen Verhältnis der Frequenz dran.
Beim Vergleich zwischen $CHCl_3$ und $CDCl_3$ steigt das Verhältnis der Wellenzahlen an und beim Vergleich zwischen $CHCl_3$ und $CHBr_3$ fällt dieses ab.
