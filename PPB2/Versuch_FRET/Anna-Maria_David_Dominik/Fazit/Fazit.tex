\chapter{Fazit}
Dieser Versuch hat uns einen kleinen Einblick in die Biophysik und ihre Methoden gegeben.
Wir haben uns hierzu den Förster-Resonanzenergietransfer und seine Theorie genauer angeschaut.
Bei den Messungen haben wir das erste Mal mit einem Konfokalmikroskop gearbeitet.\newline

Wir haben die Effizienz des FRET auf drei unterschiedliche Arten berechnet.
Hierbei erhielten wir einen Einblick in das Bleichen von Farbmolekülen und die Fluoreszenz-Messung.
Aus den erhaltenen Werten, kann man folgenden Vergleich anstellen:
\begin{equation}
    E_{SE}>\left|E_{FLIM}\right|\approx E_{BL}
\end{equation}
Die Messungen der Sensitized Emission wurden vor der Mittags-Pause gemacht.
Dies könnte eine Erklärung sein, warum dort die Effizienz viel größer ist, als bei den danach erfolgten Messungen.
Die bestimmte FRET-Effizienz bei der 'Lebenszeitmessung' liegt in der gleichen Größenordnung, wie die des 'Akzeptorbleichens'.
Warum aber die FRET-Effizienz bei der 'Lebenszeitmessung' negativ ist uns unklar.

