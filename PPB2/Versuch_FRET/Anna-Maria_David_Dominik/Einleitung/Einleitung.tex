\chapter{Einleitung}
In biologischen Proben ist es oftmals schwierig Abstände von Atomen oder Molekülen zu bestimmen. Die meisten Mikroskope haben
eine zu geringe Auflösung um diese Distanzen, die meist nur wenige Nanometer
groß sind, zu messen. 
Eine wichtige Messmethode zur Bestimmung von diesen Abständen ist der Förster Resonanzenergietransfer
(engl. Förster resonance energy transfer), kurz FRET.\\
Hierbei handelt es sich um einen Mechanismus zur strahlungslosen Energieübertragung zwischen zwei Farbstoffen.
Der Donor, in unserem Fall ein cyan-farbenes fluoreszentes Protein (CFP), wird angeregt und dessen Anregungsenergie
wird dann strahlungslos auf den zweiten Farbstoff, den sogenannten Akzeptor, übertragen. 
Der Akzeptor, in unserem Fall ein gelb-farbenes fluoreszentes Protein (YFP), kann diese Energie in Form von Strahlung 
wieder abgeben. \\
Damit FRET in einer biologischen Zelle beobachtbar ist, sind die Proteine über PH (Pleckstrin-Homologiedomäne) an
die Membran gebunden. Die beiden Farbstoffe kommen sich dementsprechend nahe genug, damit FRET stattfinden und die FRET-Effizienz gemessen werden kann. 
Durch diese können Rückschlüsse auf den Abstand gezogen werden, da die Effizienz abhängig vom Abstand ist. 
Zudem ist eine weitere Voraussetzung für FRET, dass es zu einer Überlappung zwischen dem Emissionsspektrum 
des Donors und dem Absorptionsspektrum des Akzeptors kommt. \\
In diesem Versuch wird die Theorie von FRET genauer untersucht, aber auch dessen praktische Anwendung wird betrachtet.
Die FRET-Effizienz wird im experimentellen Teil des Versuches auf drei Arten bestimmt, durch Sensitized Emission, Photobleaching und Lebenszeitmessung. 
Die verschiedenen Methoden werden im Folgenden genauer beleuchtet, verglichen und diskutiert.
