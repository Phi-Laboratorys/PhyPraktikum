%Matteo Kumar - Leonard Schatt
% Fortgeschrittenes Physikalisches Praktikum

% 1. Kapitel Einleitung

\chapter{Einleitung}
\label{chap:einleitung}

Die Frage, was auf kleinster Ebene geschieht, ist eine Frage, die die Menschheit schon seit mindestens zweieinhalb Jahrtausenden 
quält. Dabei ist das menschliche Auge in seiner Fähigkeit Objekte aufzulösen sehr begrenzt. Ab dem 1. Jahrhundert nach Christus ist belegt, 
dass sich die Menschen mit technischen Mitteln zu helfen wissen. Anfangs handelt es sich um rein optische Vergrößerungen. Diese sind jedoch in ihrem Vergrößerungsvermögen sehr 
begrenzt. Im 20. Jahrhundert wurde das Rasterelektronenmikroskop (REM) erfunden - ein Durchbruch in der Mikroskopie. Leider ist auch dieses unfähig Objekte von unter einem Nanometer 
aufzulösen, da auch dieses Mikroskop, wie das optische, durch die Wellenlänge des verwendeten Teilchens limitiert ist. Dieses Problem wird dann in der zweiten Hälfte des 20. 
Jahrhunderts durch das Rasterkraftmikroskop (AFM) gelöst. Hier tastet eine Cantilever die Probe ab und umgeht somit die Limitation durch die Wellenlänge des Informationsträgers. 
Um dieses AFM wird es in diesem Versuch gehen. Es werden dabei mit einem faustgroßen Gerät Aufnahmen im Nanometerbereich gemacht. Beispielsweise werden Nanoröhrchen, ein 
Gebilde aus Kohlenstoff, mikroskopiert. 