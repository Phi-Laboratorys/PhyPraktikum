\chapter{Fazit}
In diesem Versuch wurde die Fourier-Transformations-Spektroskopie verwendet,
um verschiedene Linien unterschiedlicher Elemente zu untersuchen. \\
Nach Beendigung des Versuches mithilfe der FT Spektroskopie kann durchaus 
gesagt werden, dass die FTS sich gegen die 'Standard' Spektroskopie Methoden, 
welche häufig einen Monochromator verwenden, durchsetzt. \\
Vor allem während der Auswertung konnten man sich ein Bild über die Vorteile der
FT-Spektroskopie machen. Ein großer Vorteil ist die kurze Messzeit. 
Während bei anderen Versuchen die Messzeit mehrere Minuten dauerte, waren es  
in diesem Versuch lediglich 100 Sekunden. Zusätzlich hat
die FT-Spektroskopie eine gute Auflösung, welche die Daten-Analyse erleichterte.\\
Im Versuch wurden Interferogramme und deren Einhüllende einer Natrium-Dampflampe, einer 
Quecksilberhochdrucklampe und einer Quecksilberniederdrucklampe aufgenommen. Anschließend
wurden die Wellenlängen der Lampen bestimmt, diese stimmten auch mit den Literaturwerten
überein. Es wurden noch die Kohärenzlänge und die Linienbreite bestimmt, aber 
auch welcher Verbreiterungsmechanismus vorliegt wurde untersucht. Die Natriumdampflampe und 
die Quecksilberniederdrucklampe verbreitern sich durch die Dopplerverbreiterung, wohin
gegen die Quecksilberhochdrucklampe sich per Druck verbreitert. 
