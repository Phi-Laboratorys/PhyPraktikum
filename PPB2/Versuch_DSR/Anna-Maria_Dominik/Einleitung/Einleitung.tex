\chapter{Introduction}
Hyperfine structures are defined by small shifts in degenerate energy levels, so basically it's the 
smallest energy distance in atoms, molecules, or ions. 
The hyperfine structure arises from the energy of the nuclear magnetic dipole moment interacting with the magnetic field
 and the nuclear electric quadrupole moment in the electric field, at least in atoms.\\
One Possibility to measure such small distances as the hyperfine structures is the Doppler free spectroscopy
which we will use in our experiment. 
During our experiment we will investigate rubidium and its hyperfine structure in more detail.\\
Rubidium is the chemical element with the symbol Rb and atomic number 37, so it’s a alkali metal. 
Natural rubidium comprises two isotopes: \ce{^{85}Rb} (72\%, stable), and 28\% is slightly radioactive \ce{^{87}Rb}.
Furthermore, rubidium can be handled as one electron system, which is useful for the following data analysis. 
However, in this experiment we will have a closer look at the fine and hyperfine structure of 
\ce{^{87}Rb} and \ce{^{85}Rb}, as well as the transition energy and the ratio its isotopes. 
