\section{Linienbreiten und Verbreiterung}
Die \textbf{natürliche Linienbreite} ist die minimalste Linienbreite.
Die endliche Ausdehnung der Linienbreite geht auf die quantenmechanische Energie-Zeitunschärfe zurück.
Die Bestimmung der Lebensdauer und der exakten Energie eines 
angeregten Zustandes ist nicht möglich. 
Dies führt zu einer statischen Verbreiterung der Linienbreite, die kleinste mögliche Breite ist die natürliche Linienbreite.\\
\begin{equation}
    \Delta E \cdot \Delta t \leq \frac{\hbar}{2}
\end{equation}

Unter \textbf{Verbreiterungsmechanismen} versteht man die Vergrößerung der Linienbreite
über die natürliche Linienbreite hinaus. Hierbei unterscheidet man zwischen zwei Arten:
Die \textit{homogene Verbreiterung} tritt auf, wenn die Emissionswahrscheinlichkeit für eine bestimmte Frequenz 
für alle Teilchen gleich groß ist. Hierzu zählen z.B. Druckverbreiterung und Sättigungsverbreiterung. 
Die \textit{inhomogene Verbreiterung} tritt auf, wenn die Emissionswahrscheinlichkeit für eine bestimmte Frequenz 
nicht für alle Teilchen gleich groß ist. Hierzu zählt z.B. die Doppelverbreiterung.\\

\textbf{Druckverbreiterung}\\
Wenn die Atome freibeweglich sind, kann es zu Stößen zwischen zwei Atomen kommen. Bei einem elastischen Stoß verschieben sich die Energieniveaus kurzzeitig, da sie in den Wirkungsbereich der Coulombwechselwirkung kommen. Wenn es während dieser Zeit zu einer Photonen-Emission kommt, so hat dieses emittierte Photon eine andere Energie als bei der 'normalen' Emission.
Dies führt zu einer Verbreiterung der Spektrallinie.\\

\textbf{Doppelverbreiterung} \\
Dieser Verbreiterungsmechanismus beruht auf dem optischen Dopplereffekt.
Ein bewegtes Atom strahlt beim Übergang von einem höheren energetischen Zustand in einen niedrigeren energetischen Zustand ein Photon ab, dessen Frequenz bzw. Wellenlänge ist geschwindigkeitsabhängig. Die Wellenlänge wird größer, wenn sich das Atom entgegen der Emissionsrichtung des Photons bewegt und kleiner im umgekehrten Fall. 
Es ergeben sich Doppelverschiebungen in beide Richtungen und aufgrund der unterschiedlichen Geschwindigkeiten, kommt es zu einer Verbreiterung der Spektrallinie. \citep[vgl.][]{linienbreite}
