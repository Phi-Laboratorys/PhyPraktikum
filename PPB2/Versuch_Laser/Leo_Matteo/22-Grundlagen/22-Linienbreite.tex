
\section{Linienbreite}

\subsection{Homogene Linienbreite}

Die homogene Linienbreite ist die Linienbreite, die jedes Atom von selbst aufweist. Diese lässt sich nicht verringern.
Sie kann beispielsweise durch die Energieunschärfe 

\begin{equation}
    \Delta E \cdot \Delta t  \geq   \frac{\hbar}{2}
\end{equation}

gegeben sein. Da ein Zustand nicht unendlich lange existieren kann, 
muss die Energie und damit die Frequenz immer unscharf sein.


\subsection{Inhomogene Linienbreite}

Die inhomogene Linienbreite hängt von der Auswahl der Emitter ab. Dabei könnte diese
theoretisch durch die richtige Auswahl der Emitter in der Messung verringert werden. Das 
klassische Beispiel ist die Dopplerverschiebung im Spektrum eines Lasers. Die Atome in dem Gas
bewegen sich alle unterschiedlich schnell. Daher ist auch die Verbreiterung der Linie, die
durch die Addition der einzelnen Linien zustande kommt, von diesen abhängig. Sollte man es durch einen
geschickten Aufbau schaffen nur gleich schnelle Atome zu beobachten, wie man 
das im Versuch dopplerfreie Spektroskopie macht, kann man die Linienbreite reduzieren.