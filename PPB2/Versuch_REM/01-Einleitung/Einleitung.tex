% 1. Einleitung

\chapter{Einleitung}
\label{chap:einleitung}

Die Rasterelektronenmikroskopie hat sich als unentbehrliches Hilfsmittel zur Untersuchung von Oberflächen bewiesen.\\

In diesem Versuch wird uns die Funktionsweise und die Verwendung eines Rasterelektronenmikroskop (REM) der Marke Jeol JSM 6510 näher gebracht. Weiterhin werden wir durch Beobachtung mehrerer Proben ein Gefühl für die jeweiligen Darstellungen und ihre Bedeutung entwickeln. Unter den Proben fallen eine Pfennig-Münze, Fliege (als biologische Probe), Zinnstandard und gebrochene Schraube. Durch eine EDX-Analyse werden wir dann auch in der Lage sein die Bruchursache der Schraube zu ermitteln und unserer davor gesammeltes Wissen in der Praxis anzuwenden.\\

Am Schluss betrachten wir einen Micro-Chip, da dessen Aufbau und Struktur unser Intresse gewegt hat. 