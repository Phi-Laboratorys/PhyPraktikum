% Charlotte Geiger - Manuel Lippert - Leonard Schatt
% Physikalisches Praktikum

% Teilauswertung 1

\section{Umkehrverstärker}
Bei der Teilaufgabe 1 haben wir $R_1=10k\Omega$ und $R_{2_1}=1M\Omega$ bzw. $R_{2_2}=4,7M\Omega$ verwendet. Nun sollten wir sowohl den Frequenzgang der Verstärkung $\upsilon$ von 10Hz ab messen, als auch die Flankenabfallszeit $\tau$ bei der Übertragung einer Rechteckspannung.\\
Zuerst berechnen wir die Verstärkung $\upsilon$ und den Eingangswiderstand $R_e$. \\
Dabei bezeichnet $\upsilon_{t1}$ die theoretische Berechnung der Verstärkung bei Nutzung des Widerstandes $R_2=1M\Omega$, und $\upsilon_{t2}$ die theoretische Berechnung der Verstärkung mit dem Widerstand $R_2=4,7M\Omega$. Der Eingangswiderstand ergibt sich folgendermaßen: \\
$R_e=\frac{U_e}{I_1}=R_1=10k\Omega$
\begin{equation}
\upsilon=-\frac{R_2}{R_1} \tab \Rightarrow \tab \upsilon_{t1}= -100, \tab \upsilon_{t2}=-470
\end{equation}


Nun folgt die Fehlerberechnung mit dem Fehlerfortpflanzungsgesetz:
\begin{equation}
u_{\upsilon}=\sqrt{\bigl(-\frac{1}{R_1}\cdot u_{R_2}\bigr)^2 + {\bigl(\frac{R_2}{R_1^2}\cdot u_{R_1}\bigr)^2}}
\end{equation}
\begin{equation}
u_{R_e}=u_{R_1}
\end{equation}
Den Fehler für die Widerstände nehmen wir als kleiner 3 Prozent an. Daraus ergeben sich folgende theoretischen Werte:\\
\begin{align}
R_e=R_1=(10\pm 0,3)k\Omega\\
R_{2_1}=(1\pm 0,03)M\Omega\\
R_{2_1}=(4,71\pm 0,14)M\Omega\\
\upsilon_{t1}=-(100\pm4)\\
\upsilon_{t2}=-(470\pm20)
\end{align}
Bei der Berechnung von den Fehlern $s_{U_a}$ und $s_{U_e}$ nehmen wir den Ablesefehler $u_a$= 0,1 und Restfehler $u_r$ = 3 Prozent des Oszilloskops an. Somit folgt:
\begin{equation}
s_{U_a}=\sqrt{u_a^2+u_r^2}=\sqrt{(0,03 \cdot U_a)^2+(U_a\cdot 0,1)^2} \\
\end{equation}
\begin{equation}
s_{U_e}=\sqrt{u_a^2+u_r^2}=\sqrt{(0,03 \cdot U_e)^2+(U_e \cdot 0,1)^2} 
\end{equation}
\begin{equation}
\upsilon=\frac{U_a}{U_e} \Rightarrow u_\upsilon=\sqrt{(\frac{u_{U_a}}{U_e})^2+(\frac{U_a \cdot u_{U_e}}{U_e^2})^2}
\end{equation}
Mit den gemessenen Werten und berechneten Fehlern ergeben sich folgende Tabellen:
%Tabelle R_2=1MOhm und Tabelle R_2=4,7MOhm

Beim Versuch, die gemessene Verstärkung $\upsilon$ mit den theoretisch berechneten Werten der Verstärkung  $\upsilon_{t1}$ bzw $\upsilon_{t2}$ fällt auf, dass die theoretischen Werte nicht von der Frequenz abhängen. Beim Vergleich der Werte wird deutlich, dass die theoretisch berechnete Verstärkung die Verstärkung sehr kleiner Frequenzen beschreibt. Daher nehmen wir die gemessene Verstärkung von kleinen Frequenzen als Vergleich. 
\begin{equation}
\upsilon_1(f=10Hz)=(98\pm 14,5)
\end{equation}
\begin{equation}
\upsilon_2(f=10Hz)=(460\pm 48,5)
\end{equation}

Beim Vergleich erkennt man, dass die Beträge der Werte sich in den Fehlerbereichen überschneiden und daher im Rahmen der Messgenauigkeit übereinstimmen. Jedoch fällt auch auf, dass die theoretischen Werte negativ, die gemessenen jedoch positiv sind. Der Grund dafür ist die Phasenverschiebung von Eingangs und Ausgangssignal. \\





Nun beschäftigen wir uns mit der Bandbreite und der Flankenabfallszeit:\\
Die theoretisch erwartete Bandbreite liegt in dem Bereich, in dem die Verstärkung auf das konstante $\frac{\sqrt{2} }{2}$-fache des Maximalwertes gefallen ist. Das heißt für:\\
\begin{align}
R_2=1M\Omega: \upsilon_B=\frac{\sqrt{2}}{2}\cdot \upsilon_{max}=\frac{\sqrt{2} }{2}\cdot 98=69,296\approx 69\\
R_2=4,7M\Omega: \upsilon_B=\frac{\sqrt{2}}{2}\cdot \upsilon_{max}=\frac{\sqrt{2}}{2}\cdot460 =325,27\approx 325\\
\end{align}
Den Wert für die Bandbreite und daher auch die Grenzfrequent ermitteln wir rechnerisch bzw. graphisch. Dafür tragen wir den Frequenzgang doppellogarithmisch in ein Diagramm ein. Wie in den Fragen zur Vorbereitung schon beschrieben bleibt da das Verstärkung–Bandbreite–Produkt näherungsweise konstant $v\cdot B = v \cdot f_{gr} = \text{constant}$. Der konstante Bereich wird dabei durch die Grenzfrequenz $f_{gr}$ begrenzt, wobei bei einer Verstärkung $v=0$ die sogenannte Transitfrequenz $f_T$ liegt. Daher betrachten wir nur den linearen Teil der Kurve. Nun multiplizieren wir die jeweiligen Messwerte mit der Verstärkung und bilden daraus den Mittelwert, wodurch wir das Verstärkung-Bandbreite-Produkt $B\cdot \upsilon$ bekommen. 
%Graph einfügen

Nun soll auch die Beziehung zwischen der Flankenabfallszeit und der Bandbreite B betrachtet werden. \\
Dafür nehmen wir das oben berechnete Verstärkung–Bandbreite–Produkt und setzen dies mit der Transitfrequenz gleich, da die folgende Relation gilt:\\
\begin{equation}
f_g=\frac{f_T}{\upsilon_{theo}}
\end{equation}
Dadurch bekommen wir folgende Werte für die zwei Widerstände, wobei wir die Transitfrequenz aus einem Datenblatt entnommen haben:\\
\begin{equation}
\text{Für R} =1 \text{M} \Omega: f_{g1}=\frac{f_T}{\upsilon_{t1}}=\frac{4}{100}=0,04\cdot 10^6=40000Hz\\
\end{equation}
\begin{equation}
\text{Für R} =4,7 \text{M} \Omega: f_{g2}=\frac{f_T}{\upsilon_{t2}}=\frac{4}{470}=0,0085\cdot 10^6=8500Hz
\end{equation}
Da wir den Fehler für die Transitfrequenz aus dem Datenblatt wegen fehlender Angaben weglassen, berechnet sich der Fehler folgendermaßen:\\
\begin{equation}
u_{f_{g1}}=\sqrt{(\frac{f_T \cdot u_{\upsilon_{t1}}}{\upsilon_{t1}^2})^2}=0,0016\cdot 10^6=1600Hz
\end{equation}
\begin{equation}
u_{f_{g2}}=\sqrt{(\frac{f_T \cdot u_{\upsilon_{t2}}}{\upsilon_{t2}^2})^2}=0,00036\cdot 10^6=360Hz
\end{equation}
\begin{equation}
\Rightarrow f_{g1}=(40\pm1,6)kHz
\end{equation}
\begin{equation}
\Rightarrow f_{g2}=(8,5\pm0,4)kHz
\end{equation}
Zusätzlich können wir die Grenzfrequenz auch durch folgende Relation berechnen: Wir nehmen an, dass die Beziehung zwischen B und $\tau$ indirekt proportional ist. Dadurch können wir mit folgender Beziehung nach $f_g$ auflösen:\\
\begin{equation}
B=2\pi\cdot f_g \Rightarrow \frac{1}{\tau}=B=2\pi\cdot f_g \Leftrightarrow  f_g=\frac{1}{2\pi \tau}
\end{equation}
 Dies werden wir nun durch unsere und gemessenen Werte versuchen zu bestätigen und vergleichen dann die verschiedenen Werte für die Grenzfrequenz. \\
 Die gemessenen Flankenabfallszeiten haben folgende Werte: \\
\begin{equation}
\text{Für R} =1\text{M}\Omega:\tau_1=\Delta t =4,8div \cdot 0,1\frac{\mu s}{div}=4,8\cdot10^{-6}s=4,8\mu s
\end{equation}
\begin{equation}
\text{Für R} =4,7\text{M} \Omega:\tau_2=\Delta t =4,0div \cdot 0,2\frac{\mu s}{div}=0,8\cdot10^{-6}s= 0,8 \mu s
\end{equation}
Daraus folgt:\\
\begin{equation}
\text{Für R} =1 \text{M} \Omega: f_{g1}=\frac{1}{2\pi \tau_1}= \frac{1}{2\pi \cdot 4,8\cdot 10^{-6}}=33,157kHz
\end{equation}
Im Vergleich mit mehreren Gruppen fällt auf, dass wir uns bei $\tau$ von R=4,7$M\Omega$ verschrieben haben. Die Skalierung lag wahrscheinlich bei 5$\frac{\mu s}{div}$. Daher rechne ich nun mit diesem Wert weiter. 
\begin{equation}
\text{Für R} =4,7 \text{M} \Omega: f_{g2}=\frac{1}{2\pi \tau_2}= \frac{1}{2\pi \cdot 0,8 \cdot 10^{-6}}=7957,747Hz
\end{equation}
\begin{equation}
u_{\tau_1}=\sqrt{0,5^2+(4,8\cdot0,03)^2}\cdot10^{-6}=\cdot10^{-6}=0,52\cdot10^{-6}
\end{equation}
\begin{equation}
u_{\tau_2}=\sqrt{0,1^2+(20\cdot0,03)^2}\cdot10^{-6}=0,608\cdot10^{-6}
\end{equation}
\begin{equation}
u_{f_{g1}}=\sqrt{(\frac{u_{\tau_1}}{\tau^2 \cdot 2\pi})^2}= 3592,04
\end{equation}
\begin{equation}
u_{f_{g2}}=\sqrt{(\frac{u_{\tau_2}}{\tau^2 \cdot 2\pi}}= 241,9
\end{equation}
Daraus folgt, dass:
\begin{equation}
f_{g1}=(33\pm3,6)kHz
\end{equation}
\begin{equation}
f_{g2}=(8,0\pm0,2)kHz
\end{equation}
Beim Vergleichen der auf zwei Arten berechneten Grenzfrequenzen $f_g$ bemerkt man, dass sie deutlich in der gleichen Größenordnung sind und vor allem die Werte für den Widerstand R=4,7M$\Omega$ sich in ihren Fehlern deutlich überschneiden. Auch bei dem anderen Widerstand merkt man, dass sich die Werte stark annähern, sich aber nicht in den Fehlern überschneiden. Dafür könnten mehrere Gründe die Ursache sein. Zum Einen könnten es Messunsicherheiten, zu lange Kabel oder ähnliche Schalttechnische Fehler sein. Dies würde jedoch auch einen Fehler bei dem anderen Widerstand hervorrufen, was es jedoch wie gerade beschrieben nicht macht. Ein anderer Grund könnten fehlerhafte Mitschriften der Protokollperson sein. Da dies auch bei dem anderen Widerstand vorgekommen ist, ist dies auch hier nicht ausgeschlossen. Jedoch ist die Abweichung sehr gering, weshalb man es auch der generellen Ableseungenauigkeit zu schieben könnte. \\
Im Allgemeinen erkennt man jedoch, dass beide Rechenmethoden zu einem sehr ähnlichen Ergebnis führen, wodurch unsere theoretische Vermutung der Indirekten Proportionalität von B zu $\tau$ bestätigt wurde. 





















































