% Charlotte Geiger - Manuel Lippert - Leonard Schatt
% Physikalisches Praktikum

% Teilauswertung 1

\section{Umkehrverstärker}
Bei der Teilaufgabe 1 haben wir $R_1=10k\Omega$ und $R_{2_1}=1M\Omega$ bzw. $R_{2_2}=4,7M\Omega$ verwendet. Nun sollten wir sowohl den Frequenzgang der Verstärkung $\upsilon$ von 10Hz ab messen, als auch die Flankenabfallszeit $\tau$ bei der Übertragung einer Rechteckspannung.\\
Zuerst berechnen wir die Verstärkung $\upsilon$ und den Eingangswiderstand $R_e$. \\
Dabei bezeichnet $\upsilon_{t1}$ die theoretische Berechnung der Verstärkung bei Nutzung des Widerstandes $R_2=1M\Omega$, und $\upsilon_{t2}$ die theoretische Berechnung der Verstärkung mit dem Widerstand $R_2=4,7M\Omega$. Der Eingangswiderstand ergibt sich folgendermaßen: \\
$R_e=\frac{U_e}{I_1}=R_1=10k\Omega$
\begin{equation}
\upsilon=-\frac{R_2}{R_1} \tab \Rightarrow \tab \upsilon_{t1}= -100, \tab \upsilon_{t2}=-470
\end{equation}


Nun folgt die Fehlerberechnung mit dem Fehlerfortpflanzungsgesetz:
\begin{equation}
u_{\upsilon}=\sqrt{\bigl(-\frac{1}{R_1}\cdot u_{R_2}\bigr)^2 + {\bigl(\frac{R_2}{R_1^2}\cdot u_{R_1}\bigr)^2}}
\end{equation}
\begin{equation}
u_{R_e}=u_{R_1}
\end{equation}
Den Fehler für die Widerstände nehmen wir als kleiner 3 Prozent an. Daraus ergeben sich folgende theoretischen Werte:\\
$R_e=R_1=(10\pm 0,3)k\Omega$\\
$R_{2_1}=(1\pm 0,03)M\Omega$\\
$R_{2_1}=(4,71\pm 0,14)M\Omega$\\
$\upsilon_{t1}=-(100\pm4)$\\
$\upsilon_{t2}=-(470\pm20)$\\
Bei der Berechnung von den Fehlern $s_{U_a}$ und $s_{U_e}$ nehmen wir den Ablesefehler $u_a$= 0,1 und Restfehler $u_r$ = 3 Prozent des Oszilloskops an. Somit folgt:
\begin{equation}
s_{U_a}=\sqrt{u_a^2+u_r^2}=\sqrt{(0,03 \cdot U_a)^2+(U_a\cdot 0,1)^2} \\
\end{equation}
\begin{equation}
s_{U_e}=\sqrt{u_a^2+u_r^2}=\sqrt{(0,03 \cdot U_e)^2+(U_e \cdot 0,1)^2} 
\end{equation}
\begin{equation}
\upsilon=\frac{U_a}{U_e} \Rightarrow u_\upsilon=\sqrt{(\frac{u_{U_a}}{U_e})^2+(\frac{U_a \cdot u_{U_e}}{U_e^2})^2}
\end{equation}
Mit den gemessenen Werten und berechneten Fehlern ergeben sich folgende Tabellen:
%Tabelle R_2=1MOhm und Tabelle R_2=4,7MOhm

Beim Versuch, die gemessene Verstärkung $\upsilon$ mit den theoretisch berechneten Werten der Verstärkung  $\upsilon_{t1}$ bzw $\upsilon_{t2}$ fällt auf, dass die theoretischen Werte nicht von der Frequenz abhängen. Beim Vergleich der Werte wird deutlich, dass die theoretisch berechnete Verstärkung die Verstärkung sehr kleiner Frequenzen beschreibt. Daher nehmen wir die gemessene Verstärkung von kleinen Frequenzen als Vergleich. 
\begin{equation}
\upsilon_1(f=10Hz)=(98\pm 14,5)\\
\upsilon_2(f=10Hz)=(460\pm 48,5)
\end{equation}

Beim Vergleich erkennt man, dass die Beträge der Werte sich in den Fehlerbereichen überschneiden und daher im Rahmen der Messgenauigkeit übereinstimmen. Jedoch fällt auch auf, dass die theoretischen Werte negativ , die gemessenen jedoch positiv sind. Der Grund dafür ist die Phasenverschiebung von Eingangs und Ausgangssignal. \\


Nun werden wir das Verstärkung-Bandbreite-Produkt $B\cdot \upsilon$ bestimmen. \\
Dieses VBP ist ein Kennwert von Operationsverstärkern.  Um das Produkt zu bestimmen, tragen wir den Frequenzgang doppellogarithmisch in ein Diagramm ein. Wie in den Fragen zur Vorbereitung schon beschrieben bleibt das Verstärkung–Bandbreite–Produkt  näherungsweise konstant $v\cdot B = v \cdot f_{gr} = \text{constant}$. Der konstante Bereich wird dabei durch die Grenzfrequenz $f_{gr}$ begrenzt, wobei bei einer Verstärkung $v=0$ die sogenannte Transitfrequenz $f_T$ liegt. Daher betrachten wir nur den linearen Teil der Kurve. 
%Graphen einfügen
Der lineare Bereich liegt in dem Bereich\\
Für $R_2=1$ M $\Omega$\\
Für $R_2=4,7$ M $\Omega$ \\


Nun soll noch die Beziehung von der Bandbreite B und der Flankenabfallszeit betrachtet werden.\\
Wir haben folgende Werte gemessen:\\
Die Flankenabfallszeit beschreibt die Zeit, nach dem die verstärkte Spannung von 10V auf $\frac{1}{e} \cdot 10V \approx 3,7V$ fällt. \\
Dabei entspricht bei $R_2=1$ M $\Omega$ : $\Delta t =4,8div \cdot 0,1\frac{\mu s}{div}=4,8\cdot10^{-6}s$\\
Bei $R_2=4,7$ M $\Omega$ : $\Delta t =4,0div \cdot 0,2\frac{\mu s}{div}=0,8\cdot10^{-6}s$ 














