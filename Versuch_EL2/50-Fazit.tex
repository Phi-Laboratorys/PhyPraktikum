% Charlotte Geiger - Manuel Lippert - Leonard Schatt
% Physikalisches Praktikum

% 5. Kapitel Einleitung

\chapter{Fazit}
\label{chap:fazit}

Dieser Versuch hat uns demonstrativ gezeigt, dass, trotz komplexen Innenaufbau des Operationsverstärkers, jener leicht in der Bedienung ist 
bei der Durchführung mehrfacher Rechenoperationen. Dabei hat uns die Tatsache überrascht, wie einfache bestimmte Schaltungen 
in der Lage waren komplexe Rechenoperationen durchzuführen und diese am Oszilloskop visuell darzustellen. Insbesondere das Differenzieren der 
Dreiecksspannung war sehr beeindruckend. Dabei haben wir auch die Grenzen dessen, was die Schalungen können kennengelernt.\\
Im Gesamten haben wir mit diesem Versuch unsere Elektronikkenntnisse vertieft und hatten hatten generell viel Spaß 
und Freude in der Auswertung. Insbesondere wurde uns klar, dass, wenn man niedrigfrequente Rechtecksspanungen am Oszilloskop betrachten will, man das 
dieses in den Wechselstrommodus stellen muss, da sonst Versucht wird den Gleichspannungsanteil heraus zu filtern. 
Desweiteren haben die Studierenden haben definitiv einiges zu Operationsverstärker dazugelernt.