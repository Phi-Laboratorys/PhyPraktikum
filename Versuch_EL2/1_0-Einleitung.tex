% Charlotte Geiger - Manuel Lippert - Leonard Schatt
% Physikalisches Praktikum

% 1. Kapitel Einleitung

\chapter{Einleitung}
\label{chap:einleitung}

Ideale Gleichspannungsverstärker mit großer Verstärkung werden Operationsverstärker genannt. Der Name leitet sich von der Funktionsweise dieses elektrischen Bauteils ab. Mit einem Operationsverstärker kann gezielt der \textbf{Verstärkungsfaktor} eingestellt und mathematische \textbf{Operationen} an der Eingangsspannung(en) ausgeführt werden. Bestandteile des Bauteils sind Dioden, Transistoren, Widerstände und Kondensatoren, welche häufig in einer integrierten Schaltung auf einem winzigen, gekapselten Siliziumchip verbaut sind. In der Praxis wird von der hohen Leerlaufspannung kaum Gebrauch gemacht sondern eine Rückkopplung (Eigenschaft der Schaltung). \\
 \\
In diesem Versuch lernen die Studenten den Umgang mit der sogenannten Rückkopplungsschaltung. Dabei werden Schaltungen zum Verstärken, Differenzieren, und Integrieren der Eingangsspannung aufgebaut und deren Eigenschaften gemessen. Weiterhin lernen die Studierenden einen effektiven Impedanzwandler kennen in Form des Elektrometerverstärkers.