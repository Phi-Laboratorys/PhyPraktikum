% Charlotte Geiger - Manuel Lippert - Leonard Schatt
% Physikalisches Praktikum

% 2.Kapitel Fragen zur Vorbereitung

\chapter{Fragen zur Vorbereitung}
\label{chap:fvz}

% Text

% Input der Teilaufgaben je nach Produktion der Nebendateien ohne Ordner

%1
\section{Allgemeines}

\subsection*{Polarisation durch Reflexion}
Bei Auftreffen von unpolarisiertem Licht - zum Beispiel von einer Glühbirne oder von der Sonne - auf eine Glasplatte teilt sich das Licht in einen reflektierten und einen gebrochenen Lichtstrahl auf. 
Der reflektierte Strahl ist  linear polarisiert, wenn der Winkel zwischen reflektiertem in gebrochenem Strahl dem Brewster-Winkel entspricht. Dieser beträgt 90 Grad. Ist diese Voraussetzung nicht erfüllt, so entsteht nur teilweise Polarisation. 
%http://www.chemgapedia.de/vsengine/vlu/vsc/de/ph/14/ep/einfuehrung/wellenoptik/polarisation_b.vlu.html

\subsection*{Polarisation durch Brechung und Doppelbrechung}
Eine Möglichkeit für die Polarisation durch Brechung ist bei Eintritt von Lichtstrahlen in bestimmte Kristalle, beispielsweise Quarze. Licht wird hierauf je nach seiner Schwingungsrichtung in verschiedener Weise gebrochen, sind aber immer linear polarisiert. 
Kristalle weisen die Eigenschaft auf, dass sie in unterschiedliche Richtungen unterschiedlich das Licht brechen. Dieses Verhalten nennt man optisch anisotrop. Bei solchen Materialien tritt Doppelbrechung auf. Aber auch bei optisch isotropen Medien (beispielsweise Glas oder Wasser) kann Doppelbrechung vorkommen. Dieses Phänomen ist  bei Verformung des Materials durch Zug- und/oder Druckkräfte zu beobachten, was man als Spannungsdoppelbrechung bezeichnet. 
%https://www.lernhelfer.de/schuelerlexikon/physik-abitur/artikel/polarisation-von-licht-durch-reflexion-und-brechung

\subsection*{Polarisation durch Streuung}
Auch durch Streuung kann polarisiertes Licht erzeugt werden. Je nach Polarisationsrichtung wird Licht an einem sogenannten Streuzentrum unterschiedlich stark in unterschiedliche Richtungen gestreut. Das Streuzentrum ist beispielsweise ein Atom oder ein Molekül. Fällt unpolarisiertes Licht auf solch ein Streuzentrum, so wird es angeregt und das Licht wird gestreut. Da die Streurichtung ungleich der Einfallsrichtung ist, so kann man folgendes Phänomen erkennen:  Das in y-Richtung gestreute Licht wird vollständig linear in z-Richtung, das in z-Richtung abgestrahlte Licht in y-Richtung polarisiert. Daher erkennt man in bestimmten Raumrichtungen linear polarisiertes Licht.
%https://www.leifiphysik.de/optik/polarisation/ausblick/polarisation-durch-streuung

\subsection*{Funktionsweise von Polarisationsfiltern}
Die Filter bestehen aus langen, parallel ausgerichteten Kettenmolekülen. Durch das Eintreten von unpolarisiertem Licht werden die Kettenmoleküle durch die dazu parallelen elektromagnetischen Wellen angeregt, wodurch die Welle absorbiert wird. Die zu den Kettenmolekülen senkrecht schwingende Wellen können die Moleküle nicht anregen, wodurch nur parallele Wellen absorbiert werden. Daher filtern Polarisationsfilter nur Licht bestimmter Polarisationsrichtung aus. 
Bei senkrechter Verdrehung von zwei identischen Polarisationsfiltern erkennt man eine deutliche Verdunklung, da dadurch zwei senkrecht zueinander stehende Wellen absorbiert wurden. 
%https://www.leifiphysik.de/optik/polarisation/grundwissen/polarisation-von-licht-einfuehrung

\subsection*{Definition Polarisationsgrad}
Der Polarisationsgrad entspricht dem Verhältnis der Intensität des polarisierten Lichtanteils zur Gesamtintensität des Lichtes.
%direktes Zitat: https://www.spektrum.de/lexikon/optik/polarisationsgrad/2622

%2
\section{Zusammenhang von Intensität und elektrischem Feld}
Intensität wird in der Physik die Größe genannt, die die Flächenleistungsdichte beim senkrecht zur Ausbreitungsrichtung zeigende Transport von Energie beschreibt. 
Die Intensität wird im Versuch durch eine Photodiode gemessen. 
\begin{equation}
I=c\epsilon_0 E^2
\end{equation}
Mit dem elektrischen Feld E, beschrieben durch eine zeitlich periodische Funktion mit der Amplitude: $E=\sqrt{E_{\parallel}^2 +E^2_{\perp}}$, der Intensität $I=\frac{\epsilon_0\epsilon_mc_0}{2n_m}E^2$ und Medium m.
%Demtroder S. 188, 7.6



%3
\section{Fresnel'sche Formeln für  Spezialfälle }
\subsection*{$\varphi=0\rightarrow\psi=0$}
Für die Fresnelschen Formeln gilt:
\begin{align*}
\frac{E_{r,\parallel}}{E_{e,\parallel}}=\frac{n_2-n_1}{n_2+n_1}\\
\frac{E_{t,\parallel}}{E_{e,\parallel}}=\frac{2n_1}{n_2+n_1}\\
\frac{E_{r,\perp}}{E_{e,\perp}}=\frac{n_1-n_2}{n_2+n_1}\\
\frac{E_{t,\perp}}{E_{e,\perp}}=\frac{2n_1}{n_2+n_1}\\
\end{align*}
Betragsmäßig unterscheiden sich die senkrechte und parallele Richtung des reflektierten und transmittierten Anteil bezüglich des Einfallsstrahls nicht und verhalten sich gleich. Der Grund dafür ist, dass die rotationssymmetrische Aufbauweise keine Vorzugsrichtung aufweist. \\
Nun betrachten wir die Energieerhaltung, die über folgende Formeln gegeben ist:
\begin{align*}
E_{r,\parallel}+\frac{n_2}{n_1}E_{t,\parallel}=E_{e,\parallel}\\
E_{r,\perp}+\frac{n_2}{n_1}E_{t,\perp}=E_{e,\perp}\\
\end{align*}
Der Faktor $\frac{n_2}{n_1}$ wird durch den Medienwechsel notwendig. Zu erkennen ist nun, dass der reflektierte Anteil steigt, je größer der Unterschied zwischen den Brechindices ist. \\
Bei mitbetrachten des Winkels, muss man den Faktor $\frac{n_2}{n_1}$  um den Faktor $\frac{cos\psi}{cos\varphi} erweitern. 
Beide Richtungen der Polarisation verhalten sich gleich, wodurch man feststellt, dass sich durch senkrechten Durchgang keine Polarisationsgradänderung ergibt bzw. diese nicht möglich ist. Dadurch folgt: $P=const$

\subsection*{$\varphi+psi=\frac{\pi}{2}$}
Nun gilt mit den trigonometrischen Umformungen $sin(\psi)=cos(\varphi)$ und analog $sin(\varphi)=cos(\psi)$:
\begin{align*}
\frac{E_{r,\parallel}}{E_{e,\parallel}}=\frac{n_2cos(\varphi)-n_1sin(\varphi)}{n_2cos(\varphi)+n_1sin(\varphi)}= \frac{n_2cos(\varphi)-n_2cos(\varphi)}{2n_2cos(\varphi)}=0\\
\frac{E_{r,\perp}}{E_{e,\perp}}=\frac{n_1cos(\varphi)-n_2sin(\varphi)}{n_2cos(\varphi)+n_1sin(\varphi)}=\frac{n_1-\frac{n_2^2}{n_1}}{n_1+\frac{n_2^2}{n_1}}=\frac{n_1^2-n_2^2}{n_1^2+n_2^2}\\
\frac{E_{t,\parallel}}{E_{e,\parallel}}=\frac{2n_1cos(\varphi)}{2n_2cos(\varphi)}=\frac{n_1}{n_2}\\
\frac{E_{t,\perp}}{E_{e,\perp}}=\frac{2n_1cos(\varphi)}{n_2sin(\varphi)+n_1cos(\varphi)}=\frac{2n^2_1}{n_2^2+n_1^2}
\end{align*}
Man sieht aus den Ergebnissen, dass der parallele Anteil komplett transmittiert und überhaupt gar nicht reflektiert wird. Daher ist der reflektierte Strahl komplett polarisiert. Somit gilt: $P_r=1$




%4
\section{ Intensität der Streustrahlung}
Für die Rayleight-Streuung muss man sich zuerst der Schwingungseigenschaft von Luftmolekülen auseinandersetzen. Wie oben schon angedeutet können Moleküle nur in bestimmte Richtungen schwingen. Bei Anregung von Luftmolekülen entstehen induzierte elektrische Dipolschwingungen, die sogenannten Hertzschen Dipole. 
Die zur Polarisationsrichtung im Abstand r abgestrahlte Feldstärke kann man folgendermaßen berechnen:
\begin{equation}
E=\frac{\omega^2p}{4pi\varepsilon_0c^2r}
\end{equation}
mit Frequenz des abgestrahlten Lichts = $\omega$\\
Anzahldichte der Moleküle = N
E-Feldvektor des angeregten Lichts $E_{in}=E_0sin(\omega t)$\\
Polarisation = P = $(\varepsilon_r-1)\cdot\varepsilon_0E{in}$\\
induzierter Dipolmoment des Moleküls = $p=\frac{P}{N}=\frac{\varepsilon_r-1}{N}\cdot\varepsilon_0E_{in}$\\
gestreute Intensität (von Volumen mit N Molekülen) = I$=\sim NV\cdot\overline{E^2}$\\
$I_{in}\sim\overline{E_{in}}$, $\omega=\frac{2pic}{\lambda}$, $\varepsilon_r=n^2$, $\overline{E^2_{in}=\frac{E_0^2}{2}}$  \\\\
\begin{equation}
\frac{I}{I_{in}}=\frac{NV\overline{E^2}}{S^2_{in}}=\frac{\pi^2V(n^2-1)sin^2(\vartheta)}{Nr\lambda^4}
\end{equation}
Dadurch erkennt man, dass die Intensität indirekt proportional zur vierten Potenz der Wellenlänge ist. Je größer die Wellenlänge, desto kleiner die abgestrahlte Intensität. Da Blau eine kleine Wellenlänge hat von ca. 400 nm, so ist die abgestrahlte Intensität deutlich gößer, als beispielsweise rot mit der Wellenlänge von 700nm.  Daher erscheint der Himmel blau. 
% https://home.uni-leipzig.de/pwm/teaching/ExPhys3_WS0809/script/EP3_jan22_09.pdf


%5
\section{Winkelabhängigkeit der Rayleigh-Streustrahlung}
Die Lichtstreuung beruht allgemein darauf, dass durch die Lichtwelle im Dipole mit Ausrichtung in Polarisationsrichtung des Lichtes induziert werden. Die induzierten Dipole wiederum strahlen ein Lichtfeld der gleichen Frequenz ab. Die theoretische Winkelabhängigkeit der Streuintensität in der Streuebene ist je nach Polarisation des anregenden Lichts durch das Rayleigh-Gesetz vorgegeben. 
Nun diskutieren wir die Winkelabhängigkeit der Intensität der Rayleigh Streustrahlung anhand von Gleichung (7): 
\begin{equation}
Intensität = S(\delta)=AIsin^2(\delta)
\end{equation}\
Mit der Quelle Demtröder S.321 der 7.Auflage folgt folgender Zusammenhang:
%Demtröder
\begin{equation}
\overline{P_S}=N\sigma_S \overline{I}
\end{equation}
$\overline{P_S}$ als Streuleistung, $A=N\sigma(\omega)$ als Frequenzabhängigkeit, $\sigma$ als Streuquerschnitt und $\delta$ als betrachteter Winkel.\\
Daraus folgt:
\begin{equation}
I=I_0 sin^2(\delta)
\end{equation}
Dadurch wird deutlich, dass der Winkel $\delta$ je nach Polarisationsrichtung variiert. \\
Nun wird die Abhängigkeit der Polarisationsrichtung zur Streuebene betrachtet. 
Zuerst die Parallele Richtung:
%Bild 1 einfügen mit Polarisation parallel zur Streuebene
Aus der Abbildung erkennt man die geometrischen Zusammenhänge von $\delta=\frac{\pi}{2}-\Theta$. Damit folgt:
\begin{equation}
I(\Theta)=I_{0,\shortparallel}\cdot sin^2 (\frac{\pi}{2}-\Theta)=I_{0,\shortparallel}\cdot cos^2(\Theta)
\end{equation}
\begin{equation}
\Rightarrow S_{0,\shortparallel}\cdot cos^2(\Theta)
\end{equation}
Nun betrachten wir die senkrechte Polarisationsrichtung im Bezug zur Streuebene
%Bild 1 einfügen mit Polarisation senkrecht zur Streuebene
Aus dieser Abbildung kann man sehen, dass $\delta$ für jedes $\Theta$ immer $\frac{\pi}{2}$ bleibt. Damit folgt:
\begin{equation}
I(\delta=\frac{\pi}{2}) = I_{0,\perp}
\end{equation}
\begin{equation}
\Rightarrow S_\perp (\Theta) \sim AI_{0,\perp}
\end{equation}
Man erkennt, dass die Folgerungen (parallel, bzw senkrecht) den Gleichungen Gl. 8 und Gl.9 im Skript entsprechen. 
Durch die Szstem-Symmetrie haben Rückwerts- und Vorwärtsstreuung die gleichen Werte.
%www.spektrum.de/lexikon/optik/rayleigh-streuung/2774


%6
\section{starke Vorwärtsstreuung bei der Miestreuung}
Bei Lichtstreuung an festen Mikropartikeln beispielsweise Staub oder Rauch erkennt man eine elastische, kohärente Streuung. Diese Streuung von Licht an Teilchen mit einem ähnlich großen Durchmesser wie die Wellenlänge nennt man Mie-Lorentz Streuung. Die Intensität der Mie Streuung ist proportional zum quadrierten Durchmesser des streuenden Teilchens. Die gesamte gestreute Lichtintensität berechnet sich dann folgendermaßen:
\begin{equation}
I\propto \biggl| \sum_{K=1}^N A^2_K \biggl|
\end{equation}
$A_K$ ist die Streuamplitude des K-ten Moleküls im Partikel mit N Molekülen
Zustandekommen der starken Vorwärtsstreuung der Miestrahlung: 

Für die Veranschaulichung der Mie-Streuung kann folgendes System betrachtet werden:\\
%Graphik 1 der Miestreuung einfügen
Durch geometrische Überlegungen werden folgende Zusammenhänge deutlich:
\begin{align*}
\Delta_{vor}=\mid d-dcos(\Theta)\mid=d\mid 1-cos(\Theta) \mid\\
\Delta_{rück}=\mid d+dcos(\pi-\Theta_r)\mid=d\mid 1-cos(\Theta_r) \mid\\
\rightarrow \Delta = d\mid 1-cos(\Theta) \mid
\end{align*}
Der Phasenversatz ist also eine Funktion von $1-cos(\Theta)$.
Man erkennt, dass man durch diese theoretische Vorüberlegung keinen Unterschied zwischen Vorwärts und Rückwärts-Streurichtung ausmachen kann. \\
Bei den bisherigen Überlegungen haben wir den Abstand d als konstant angenommen. Dies gilt jedoch nur in der Theorie, da sich Teilchen in der Theorie durch thermische Bewegungen bzw. Schwingungen immer bewegen und sich so der Abstand dauerhaft verändert. Dies ist einer der Gründe, weshalb die Miestrahlung vor allem nach vorne streut. 
Für die "wirkliche" Beschreibung der starken Vorwärtsstreuung der Miestrahlung und anderer Effekte bei Streuung an Partikeln nutzt man die relativ komplizierte Lorenz-Mie-Theorie. Diese Theorie ist eine mathematisch genaue Beschreibung verschiedenster Streuphänomene von elektromagnetischer Wellen an Teilchen und die genaue Lösung der Maxwell-Gleichungen für die Streuung einer ebenen elektromagnetischen Welle an einem beliebig großen Teilchen. Es werden dadurch Lösungen gefunden, die verschiedene Geometrien und Formen haben (ähnlich zu den Legendre - Polynomen der Quantenmechanik). FDiese Formen zeigen, dass die Vorwärtsstreuung meist größer ist, als die Rückwärtsstrahlung.

%https://physik.cosmos-indirekt.de/Physik-Schule/Mie-Streuung
%Demtröder, Auflage 7, Kapitel 10.9.3, S.322
%webdoc.sub.gwdg.de/ebook/diss/2003/fu-berlin/1998/13/kap3.pdf
















































