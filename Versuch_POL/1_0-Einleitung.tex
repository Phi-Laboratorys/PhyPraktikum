% Charlotte Geiger - Manuel Lippert - Leonard Schatt
% Physikalisches Praktikum

% 1. Kapitel Einleitung

\chapter{Einleitung}
\label{chap:einleitung}
Polarisiertes Licht ist in unserem Alltag überall zu bemerken. In unserem direkten Umfeld sind zum Beispiel Sonnenbrillen und Computerdisplays polarisiert, aber auch in der Tierwelt wird es viel genutzt. So nutzen Bienen, Heuschrecken und manchen Fledermausarten die Lichtpolarisation für die Orientierung. 
%www.br.de/wissen/polarisation-polsarisiertes-licht-100
Aber was kann man darunter überhaupt verstehen? Wie und wodurch entsteht die Polarisation? Mit diesen Fragen zu und den Eigenschaften von linear polarisiertem Licht bezüglich Reflexion und Streuung beschäftigen wir uns in diesem Versuch. 
% Text