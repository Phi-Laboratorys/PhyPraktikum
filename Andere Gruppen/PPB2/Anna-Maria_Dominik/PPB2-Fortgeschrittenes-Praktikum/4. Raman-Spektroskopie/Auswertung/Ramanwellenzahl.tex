\section{Ramanwellenzahlen}
\subsection{Bestimmung der Ramanwellenzahlen}
Bereits während des Versuches, wurden die Wellenlängen der einzelnen Peak bestimmt.
Die kompletten aufgenommenen Spektren sind nochmal im Anhang zu finden.
Dabei wurde darauf geachtet, dass für die Wellenlängenbestimmung der 'Cursor' mittig in der Flanke war.
Die gemessenen Wellenlängen lassen sich in Wellenzahlen wie folgt umrechnen:
\begin{equation}
    \tilde{\nu}=\frac{1}{100\cdot\lambda}
\end{equation}
Um nun die Raman-Verschiebung zu berechnen, wird die Wellenzahl der Rayleigh Linie von der der (Anti-)Stokes-Linie abgezogen:
\begin{equation}
    \Delta\tilde{\nu}=\left|\tilde{\nu}_\text{Raman}-\tilde{\nu}_\text{Rayleigh}\right|
\end{equation}
Als Fehler wurde ein Ablesefehler von $0,2\,\text{nm}$ veranschlagt.
Für den Fehler von der Ramanverschiebung folgt:
\begin{equation}
    \Delta\tilde{\nu}=\left|0,2\cdot10^{-9}\,\text{m}\cdot\frac{1}{100\cdot\left(\lambda_\text{Raman}\right)^2}\right|
\end{equation}
Der Fehler für die kleinsten gemessenen Wellenlängen liegt bei $6,4\,\text{cm}^{-1}$ und für die größten bei $4,2\,\text{cm}^{-1}$.
Somit wird als Abschätzung nach oben, ein Fehler von $6,4\,\text{cm}^{-1}$ verwendet.\newpage
\subsection{Messwerte und Vergleich mit der Literatur}
Bei der Benennung, steht der erste Index für den Stokes (s)/Anti-Stokes (as) Bereich und der zweite Index für die Polarisation.
Die Literaturangaben, wurden aus dem Auszug des Buches 'Molecular spectra \& and molecular structure' aus der Zusatzliteratur entnommen \citep{zusatzliteratur}.
\subsubsection{CCl$_4$}
Folgende Wellenzahlen wurden für Tetrachlormethan bestimmt:
\begin{table}[h]
    \centering\begin{tabular}{cccc|c}
        $\tilde{\nu}_{s,0}$ (cm$^{-1}$)&$\tilde{\nu}_{as,0}$ (cm$^{-1}$)&$\tilde{\nu}_{s,90}$ (cm$^{-1}$)&$\tilde{\nu}_{as,90}$ (cm$^{-1}$)&$\tilde{\nu}_{l}$ (cm$^{-1}$)\\\hline
        &&&&145,0\\
        $\left(287,3\pm6,4\right)$&$\left(221,6\pm6,4\right)$&$\left(210,0\pm6,4\right)$&$\left(228,9\pm6,4\right)$&217,9\\
        $\left(305,5\pm6,4\right)$&$\left(318,1\pm6,4\right)$&$\left(305,5\pm6,4\right)$&$\left(327,7\pm6,4\right)$&314,0\\
        &&&&434,0\\
        &$\left(463,0\pm6,4\right)$&$\left(446,8\pm6,4\right)$&$\left(470,1\pm6,4\right)$&455,1/458,4/461,5\\
        &$\left(776,5\pm6,4\right)$&$\left(737,1\pm6,4\right)$&$\left(790,0\pm6,4\right)$&762,0/790,5\\
        &&$\left(888,9\pm6,4\right)$&&\\
        &&$\left(933,6\pm6,4\right)$&&\\
        &&$\left(1035,1\pm6,4\right)$&&\\
        &&$\left(1126,3\pm6,4\right)$&&\\
        &&$\left(1212,7\pm6,4\right)$&&\\
        &&$\left(1279,6\pm6,4\right)$&&\\
        &&$\left(1376,4\pm6,4\right)$&&\\
        &&$\left(1540,3\pm6,4\right)$&&1539,0\\
        &&$\left(1591,5\pm6,4\right)$&&\\
        &&$\left(1679,7\pm6,4\right)$&&\\
        &&$\left(1762,7\pm6,4\right)$&&\\
        &&$\left(1868,2\pm6,4\right)$&&        
    \end{tabular}
    \caption{Vergleich der gemessenen Ramanlinien mit Literatur für $0^\circ$ und $90^\circ$-Polarisation\\im (Anti-)Stokes-Bereich für CCl$_4$}
\end{table}\\
Speziell zu Tetrachlormethan ist zu sagen, dass in der Literatur mehrere nahe beieinander liegende Wellenzahlen angegeben wurden.
Diese konnten wir aber nicht auflösen und haben diese deswegen nur als eine Linie gemessen.\newpage
\subsubsection{CHCL$_3$}
Folgende Wellenzahlen wurden für Trichlormethan (Chlorophorm) bestimmt:
\begin{table}[h]
    \centering\begin{tabular}{cccc|c}
        $\tilde{\nu}_{s,0}$ (cm$^{-1}$)&$\tilde{\nu}_{as,0}$ (cm$^{-1}$)&$\tilde{\nu}_{s,90}$ (cm$^{-1}$)&$\tilde{\nu}_{as,90}$ (cm$^{-1}$)&$\tilde{\nu}_{l}$ (cm$^{-1}$)\\\hline
        $\left(100,5\pm6,4\right)$&&$\left(90,4\pm6,4\right)$&$\left(194,9\pm6,4\right)$&\\
        $\left(256,3\pm6,4\right)$&$\left(270,0\pm6,4\right)$&$\left(248,6\pm6,4\right)$&$\left(274,8\pm6,4\right)$&260,0\\
        $\left(360,1\pm6,4\right)$&$\left(370,7\pm6,4\right)$&$\left(352,3\pm6,4\right)$&$\left(370,7\pm6,4\right)$&365,9\\
        &&$\left(537,1\pm6,4\right)$&&\\
        &$\left(674,2\pm6,4\right)$&$\left(655,4\pm6,4\right)$&$\left(676,5\pm6,4\right)$&668,3\\
        &$\left(767,4\pm6,4\right)$&$\left(748,0\pm6,4\right)$&$\left(774,2\pm6,4\right)$&761,2\\
        &&$\left(911,2\pm6,4\right)$&&\\
        &&$\left(1026,6\pm6,4\right)$&&\\
        &&$\left(1126,3\pm6,4\right)$&&\\
        &$\left(1221,3\pm6,4\right)$&$\left(1209,8\pm6,4\right)$&$\left(1227,6\pm6,4\right)$&1215,6\\
        &&$\left(1276,6\pm6,4\right)$&&\\
        &&$\left(1376,4\pm6,4\right)$&&\\
        &&$\left(1585,5\pm6,4\right)$&&\\
        &&$\left(1673,6\pm6,4\right)$&&\\
        &&$\left(1765,7\pm6,4\right)$&&\\
        &&$\left(1861,9\pm6,4\right)$&&\\
        &&&&3018,9
    \end{tabular}
    \caption{Vergleich der gemessenen Ramanlinien mit der Literatur für $0^\circ$ und $90^\circ$-Polarisation\\im (Anti-)Stokes-Bereich für CHCl$_3$}
\end{table}
\subsubsection{CDCL$_3$}
Folgende Wellenzahlen wurden für deuteriertes Trichlormethan (deuteriertes Chlorophorm) bestimmt:
\begin{table}[h]
    \centering\begin{tabular}{cccc|c}
        $\tilde{\nu}_{s,0}$ (cm$^{-1}$)&$\tilde{\nu}_{as,0}$ (cm$^{-1}$)&$\tilde{\nu}_{s,90}$ (cm$^{-1}$)&$\tilde{\nu}_{as,90}$ (cm$^{-1}$)&$\tilde{\nu}_{l}$ (cm$^{-1}$)\\\hline
        $\left(100,5\pm6,4\right)$&&$\left(92,9\pm6,4\right)$&&\\
        $\left(253,7\pm6,4\right)$&$\left(265,2\pm6,4\right)$&$\left(246,0\pm6,4\right)$&$\left(270,0\pm6,4\right)$&262,0\\
        $\left(357,5\pm6,4\right)$&$\left(368,3\pm6,4\right)$&$\left(352,3\pm6,4\right)$&$\left(373,1\pm6,4\right)$&366,5\\
        &&$\left(534,4\pm6,4\right)$&&\\
        &$\left(658,2\pm6,4\right)$&$\left(639,2\pm6,4\right)$&$\left(658,2\pm6,4\right)$&650,8\\
        &$\left(744,8\pm6,4\right)$&&$\left(747,1\pm6,4\right)$&737,6\\
        &&$\left(891,7\pm6,4\right)$&&\\
        &$\left(915,2\pm6,4\right)$&$\left(930,8\pm6,4\right)$&$\left(921,8\pm6,4\right)$&908,3\\
        &&$\left(1123,4\pm6,4\right)$&&\\
        &&$\left(1212,7\pm6,4\right)$&&\\
        $\left(1218,5\pm6,4\right)$&&$\left(1279,6\pm6,4\right)$&&\\
        $\left(1379,3\pm6,4\right)$&&$\left(1370,5\pm6,4\right)$&&\\
        &&$\left(1537,3\pm6,4\right)$&&\\
        &&&&2250,0
    \end{tabular}
    \caption{Vergleich der gemessenen Ramanlinien mit der Literatur für $0^\circ$ und $90^\circ$-Polarisation\\im (Anti-)Stokes-Bereich für CDCl$_3$}
\end{table}\newpage
\subsubsection{CHBr$_3$}
Folgende Wellenzahlen wurden für Tribrommethan (Bromophorm) bestimmt:\\
\begin{table}[h]
    \centering\begin{tabular}{cccc|c}
        $\tilde{\nu}_{s,0}$ (cm$^{-1}$)&$\tilde{\nu}_{as,0}$ (cm$^{-1}$)&$\tilde{\nu}_{s,90}$ (cm$^{-1}$)&$\tilde{\nu}_{as,90}$ (cm$^{-1}$)&$\tilde{\nu}_{l}$ (cm$^{-1}$)\\\hline
        $\left(146,2\pm6,4\right)$&$\left(163,1\pm6,4\right)$&$\left(143,6\pm6,4\right)$&$\left(165,6\pm6,4\right)$&153,8\\
        $\left(215,6\pm6,4\right)$&$\left(231,3\pm6,4\right)$&$\left(210,3\pm6,4\right)$&$\left(228,9\pm6,4\right)$&222,3\\
        &&$\left(368,0\pm6,4\right)$&&\\
        &$\left(549,6\pm6,4\right)$&$\left(526,4\pm6,4\right)$&$\left(547,3\pm6,4\right)$&538,5\\
        &$\left(660,4\pm6,4\right)$&$\left(644,6\pm6,4\right)$&$\left(662,7\pm6,4\right)$&656,0\\
        &&$\left(916,8\pm6,4\right)$&&\\
        &&$\left(1023,7\pm6,4\right)$&&\\
        &$\left(1148,6\pm6,4\right)$&$\left(1129,2\pm6,4\right)$&&1142,0\\
        &&$\left(1209,8\pm6,4\right)$&&\\
        &&$\left(1279,6\pm6,4\right)$&&\\
        &&$\left(1373,4\pm6,4\right)$&&\\
        &&$\left(1417,8\pm6,4\right)$&&\\
        &&$\left(1537,3\pm6,4\right)$&&\\
        &&$\left(1588,5\pm6,4\right)$&&\\
        &&$\left(1673,6\pm6,4\right)$&&\\
        &&$\left(1768,8\pm6,4\right)$&&\\
        &&$\left(1868,2\pm6,4\right)$&&\\
        &&&&3023,0
    \end{tabular}
    \caption{Vergleich der gemessenen Ramanlinien mit der Literatur für $0^\circ$ und $90^\circ$-Polarisation\\im (Anti-)Stokes-Bereich für CHBr$_3$}
\end{table}\\
Generell liegen die gemessenen Werte in der Größenordnung und überschneiden sich auch in den meisten Fällen mit den theoretisch zu erwartenden Werten.
Die meisten Werte wurden in der $90^\circ$-Polarisation für die Stokes-Verschiebung gemessen.
Die wenigsten für die Stokes-Verschiebung bei $0^\circ$-Polarisation.
Wie auch in den angehängten Plots ersichtlich ist, ist bei $0^\circ$-Polarisation und im Stokes-Bereich das Hintergrundrauschen am stärksten.
Dadurch ist es wahrscheinlich, dass das die Peaks, die in diesem Bereich eine geringe Intensität aufweisen, als Hintergrundrauschen angesehen und somit nicht gemessen wurden.\\
Die Werte, welche keinen Literaturwert zugeordnet werden können sind mit aller Wahrscheinlichkeit äußere Störeinflüsse, welche sich als Peak geäußert haben und somit mitaufgenommen worden sind.
Einige Literaturwerte konnten gar nicht gemessen werden, da diese außerhalb des Messbereiches lagen.