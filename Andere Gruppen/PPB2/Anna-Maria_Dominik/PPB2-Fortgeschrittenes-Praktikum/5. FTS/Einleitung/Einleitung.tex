\chapter{Einleitung}
Die Spektroskopie basiert auf der Absorption von Licht und diese absorbierte Wellenlänge ist charakteristisch
für die zu untersuchende Verbindung. \\
Eine Art der Spektroskopie ist die \textbf{Fourier-Transformspektroskopie} (FTS).
Dieses Verfahren ist modern und wurde erst ab 1950 entwickelt. Hierbei 
wird ein Zweistrahl-Interferogramm der zu untersuchenden Probe aufgenommen und anschließend
erhält man das entsprechende Spektrum mithilfe der Fourier Transformation. Die Interferogramm-Intensitäten 
enthalten die gesamten Informationen des unbekannten Spektrums. \\

In unserem Versuch wird ein Michelson-Interferometer im sichtbaren Bereich als Zweistrahl
Interferogramm verwendet. Mit dessen Hilfe wird während dem Versuch eine Natrium Dampflampe,
eine Quecksilberhochdrucklampe und -niederdrucklampe untersucht. Anschließend kann man mithilfe
des aufgenommenen Spektrums die Kohärenzlänge, Wellenlänge und Linienbreite der untersuchten 
Linien bestimmen. 
