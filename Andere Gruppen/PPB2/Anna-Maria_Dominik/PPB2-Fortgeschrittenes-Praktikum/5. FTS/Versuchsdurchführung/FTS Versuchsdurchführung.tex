\documentclass[a4paper, twoside]{article}
\usepackage{geometry} \geometry{top=30mm, bottom=25mm, inner=20mm, outer=20mm}
\usepackage[linktoc=all]{hyperref}
\hypersetup{
	bookmarksopen=true,
	bookmarksdepth=3,
	colorlinks=true,
	citecolor=blue,
	linkcolor=[rgb]{0.6,0,0}}
\usepackage{graphicx}
\setlength{\abovecaptionskip}{0pt}
\usepackage{amssymb,amsmath,physics}
\begin{document}

\title{FTS: Versuchsdruchführung}
\author{Dominik Müller}
\date{\today}
\maketitle
\section{Justage des Spektrometers}
Motoren einschalten und Messprogramm starten $\to$ Spiegel fahren auf Weißlichtposition
\subsection{Strahlengang Spektrallampen}
\begin{itemize}
	\item Interferometerspiegel mit Deckeln abdecken (Mitte lokalisieren)
	\item Strahlhöhe beträgt 9,5$\,$cm
	\item Kollimator entfernen
	\item Justage des HeNe-Laserstrahls parallel zur Tischebene (mit Lochblende)
	\item Kollimator wieder einbauen und justieren, dass der Strahl mittig durch die Linsen gelangt (Lochblende)
	\item Justage der Periskop-Spiegel
	\item Immer zuerst die \textbf{rechte} Rändelschraube (Arretierung) lösen, dann die linke zur Vertikalenverstellung benutzen
	\item Hinter unterem Spiegel, Strahl auf 9,5$\,$cm Abstand zur Tischplatte parallel justieren
	\item Überprüfung mit vorhandener Lehre
	\item Strahlenteiler und Ausgleichsplatte bereits vorjustiert \textbf{Finger weg!!!}
	\item Interferometerspiegel unabhängig voneinander (Nur eine Abdeckung abnehmen) in Autokollimation (Senkrechtes Reflektieren $\leftrightarrow|$) zur Lochblende im Kollimator justieren
	\item Güte der Justierung am Ort des Detektros kontrollieren (Interferenzringe, keine -streifen)
	\item Öffnung des Detektors in Zentrum der Ringe schieben, Bandpassfilter davorsetzen 
\end{itemize}
\subsection{Strahlengang HeNe-Laser}
\begin{itemize}
	\item Ein- und austretenen Strahl am Eingang des Interferometers überlagern
	\item Dazu P1 \textbf{vorsichtig} verstellen
	\item Mit P2 Strahl auf die Photodiode richten (voll ausleuchten)
	\item Danach Spektrometer mit beiden schwarzen Hauben abdecken
	\item SiPM-Detektor und beide Transimpedanzwandler einschalten
	\item Betreuer Justage überprüfen lassen
\end{itemize}
\section{Messung atomarer Spektrallinien}
Folgende Filterkomibationen verwenden!!!!!
\begin{itemize}
\item Na-Dampflampe: OD 1.6 / Bandpassfilter: orange
\item He-Lampe: OD - / Bandpassfilter: orange
\item Hg Niederdrucklampe: OD 0.3 / Bandpassfilter: grün
\item Hg Hochdrucklampe: OD 2.0 / Bandpassfilter: grün
\end{itemize}
Beim Wechsel darauf achten, dass das Entladungsgefäß in der Austrittsöffnung komplett sichtbar ist. Lampe muss dazu \textbf{ausgeschaltet} sein!
\subsection{Na-D Linie}
Interferogramm aufnehmen:
\begin{itemize}
	\item Spiegel M1 verfahren
	\item Länge von $100\,\mu$m
	\item Vorschubgeschwindigkeit $v=1\mu$ms$^{-1}$
\end{itemize}
Komplette Schwebung aufnehmen:
\begin{itemize}
	\item Vorschubgeschwindigkeit $v=5\mu$ms$^{-1}$
\end{itemize}
Einhüllende aufnehmen:
\begin{itemize}
	\item Spiegel M2 verfahren
	\item $v=100\mu$ms$^{-1}$
\end{itemize}
\subsection{He-Linie}
Interferogramm aufnehmen:
\begin{itemize}
	\item Spiegel M1 verfahren
	\item Länge von $100\,\mu$m
	\item Vorschubgeschwindigkeit $v=1\mu$ms$^{-1}$
\end{itemize}
Einhüllende aufnehmen:
\begin{itemize}
	\item Spiegel M2 verfahren
	\item $v=100\mu$ms$^{-1}$
\end{itemize}
\section{Auswertung}
\subsection{Na-D Linie, Hg-Linie}
\begin{itemize}
\item Ermitteln Sie aus der Einhüllenden Na-Dupletts den Korrekturfaktor fur Motor M2.
\item Bestimmen Sie die Wellenlänge, die Kohärenzlänge und die Linienbreite der Linie.
\item Bestimmen Sie das Intensitätsverhältnis und Abstand der beiden Natrium-D Linien. Aus welcher Messung erhält man dafür die genauesten Messwerte?
\item Welcher Verbreiterungsmechanismus liegt bei dieser Lampe vor?
\end{itemize}
\subsection{He-Linie, Hg-Linie}
\begin{itemize}
\item Bestimmen Sie die Wellenlänge, die Kohärenzlänge und die Linienbreite der Linie.
\item Welcher Verbreiterungsmechanismus liegt bei dieser Lampe vor?
\end{itemize}
\end{document}