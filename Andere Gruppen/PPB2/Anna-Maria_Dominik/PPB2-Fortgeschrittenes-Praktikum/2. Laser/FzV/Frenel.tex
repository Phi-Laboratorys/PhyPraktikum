\section{Fresnel'sche Gleichungen}
Die Fresnel'schen Formeln beschreiben die Reflexion und Transmission einer ebenen elektromagnetischen Welle an einer Grenzschicht.
Sie sind wie folgt definiert:
\begin{align}
    \left(\frac{E_T}{E_E}\right)_\perp&=\frac{2\sin(\beta)\sin(\alpha)}{\sin(\alpha+\beta)}\\
    \left(\frac{E_R}{E_E}\right)_\perp&=-\frac{\sin(\alpha-\beta)}{\sin(\alpha+\beta)}\\
    \left(\frac{E_T}{E_E}\right)_\parallel&=\frac{2\sin(\beta)\cos(\alpha)}{\sin(\alpha+\beta)}\\
    \left(\frac{E_R}{E_E}\right)_\parallel&=\frac{\tan(\alpha-\beta)}{\tan(\alpha+\beta)}\\    
\end{align}
Sie beschreiben das Verhältnis der reflektierter $E_R$ bzw. transmittierter Feldstärke $E_T$ zu der einfallenden Feldstärke $E_E$. \newpage
Hierbei wird zusätzlich zwischen dem senkrecht $\perp$ und dem parallel $\parallel$ polarisierten Teil (zur Einfallsebene) der Feldstärke unterschieden. 
Bei den oben verwendeten Formeln wurde ein nicht-magnetisches Material verwendet, d.h. $\mu_1=\mu_2 = 1$.
Der Winkel $\alpha$ ist der Einfallswinkel und der Winkel $\beta$ ist der Ausfallswinkel.\\
Diese Darstellungsart ist nur eine von mehreren Möglichkeiten die Fresnel`schen Formeln darzustellen, man kann sie beliebig umschreiben mithilfe 
von Additionstheoremen und Brechungsgesetzen.\\
Ein besonderer Winkel sollte noch erwähnt werden, der sogenannte \textbf{Brewster-Winkel $\alpha_B$}. Fällt Licht unter diesen besonderen Winkel 
ein, so wird nur der senkrecht zur Einfallsebene polarisierten Anteil reflektiert. Das reflektierte Licht ist dann linear polarisiert. Der parallel 
zur Einfallsebene einfallende Anteil des Lichtstrahls wird transmittiert. \citep[vgl.][]{Kohler}\\
Es gilt:
\begin{equation}
    \alpha_B = \arctan(\frac{n_2}{n_1})
\end{equation}
