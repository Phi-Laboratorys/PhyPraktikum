% Charlotte Geiger - Manuel Lippert - Leonard Schatt
% Physikalisches Praktikum

% Teilaufgabe 7

\section{Senkrechte Ausrichtung eines rotierenden Bierfilzes}
Wird ein Bierfilz, welcher um seine Figurenachse rotiert, in die Luft geworfen, erfährt dieser ein Drehmoment. Dieses wird verursacht durch die Kraftwirkung des Luftwiderstand auf den Anteil der Fläche des Bierfilz, welche senkrecht zur Luftströmung steht. Dieser Umstand hat die Folge, dass sich der Drehimpuls des Bierfilzes ändert und sich dieser damit kippt bis die Richtung des Drehimpulses senkrecht zur Strömungrichtung der Luft steht. Dabei hat der Bierfilz in dieser Position ein Minimum an Reibung durch die Luft, wodurch kein weiteres Drehmoment auf den Bierfilz ausgewirkt wird und damit die Bewegung sich stabilisiert.\\
Bei einem Diskus ändert sich dieser Vorgang, da dieser ein sehr großes Trägheitsmoment bzgl. seiner Figurenachse aufweißt, was durch seine Konstruktion (Hauptmasse möglichst weit von der Figurenachse entfernt) bedingt ist. Durch diese Konstruktion muss auf einen Diskus hohe Drehmomente wirken, damit dieser seine Ausrichtung ändert. Die von der Luftströmung verursachten Drehmomente sind dafür aber zu schwach um eine signifikante Änderung zu verursachen, bevor der Diskus auf dem Boden aufschlägt.