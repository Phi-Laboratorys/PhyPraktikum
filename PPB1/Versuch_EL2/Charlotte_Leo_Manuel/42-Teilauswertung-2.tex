% Charlotte Geiger - Manuel Lippert - Leonard Schatt
% Physikalisches Praktikum

% Teilauswertung 2

\def\weite{4cm}

\section{Umkehrintegrator}


\subsection*{Diskussion der Beobachtungen am Oszilloskop}
\begin{center}
    \includegraphics[scale = 0.4]{intOPVverbessert.PNG}
    \captionof{figure}{Modifizierte Version eines Umkehrintegrators}
\end{center}
Bei der Betrachtung von Abbildung \ref{abb:int} wird deutlich, dass der Umkehrintegrator bei zunehmender Frequenz immer besser integriert und die dadurch entstehende Dreiecksspannung immer besser sichbar wird.
\begin{center}
    \begin{tabular}{c c c}
        \includegraphics[width = \weite]{42-10Hz.jpg} & \includegraphics[width = \weite]{42-100Hz.jpg} & \includegraphics[width = \weite]{42-1000Hz.jpg}
    \end{tabular}
    \captionof{figure}{Umkehrintegrator bei zunehmender Frequenz}
    \label{abb:int}
\end{center}
Dabei ist zu bemerken, dass die entstehende Dreiecksspannung einen sichtbaren Phasenunterschied $\Delta\varphi$ vorweist. Das deckt sich mit der Erwartung der Integrationsschaltung, da die Ausgangsspannung $U_a$ der Eingangsspannung $U_e$ aufgrund des Kondensators $C_2$ bei höheren Frequenzen um $\Delta\varphi = \frac{\pi}{2}$ nachhinkt. Dieser Umstand lässt sich auch mit Gleichung (2.19) herleiten, wobei daraus folgt:
\begin{gather}
    \begin{aligned}
        U_a(t) &= - \frac{R_2}{R_1} \frac{1}{1+i\omega R_2 C_2} U_e(t)\\[0.5cm]
        \Leftrightarrow \frac{U_a}{U_e} &= v_{i,2} = - \frac{R_2}{R_1} \frac{1}{1+i\omega R_2 C_2} = - \frac{R_2}{R_1} \frac{1 - i\omega R_2 C_2}{1+(\omega R_2 C_2)^2}
    \end{aligned}\\[0.5cm]
    \Rightarrow \Delta\varphi = \arctan(\frac{\imaginary(v_{i,2})}{\real(v_{i,2})}) = \arctan(\omega R_2 C_2) \xrightarrow{\omega \rightarrow \infty} \frac{\pi}{2} 
\end{gather}
Betrachtet man nun die Gleichung \ref{eq:wi2} ergibt sich für diese Schaltung mit den Widerständen 
\begin{gather*}
    R_1 = (10.0 \pm 0.3)~\text{k}\Omega,~R_2 = (1.00 \pm 0.03)~\text{M}\Omega~\text{und}~C_2 = (10.0 \pm 0.3)~\text{nF},
\end{gather*}
wobei dabei der Fehler der Widerstände mit $3\%$ des Messwerts abgeschätzt und der Kondensator einen vom Hersteller angegebenen Fehler von $2.5\%$ des angegeben Wertes hat, für die Grenzfrequenz $f_{i,gr}$ mit dem Fehler $s_{f_{i,gr}}$ (ermittelt durch das Fehlerfortpflanzungsgesetz):
\begin{gather}
    f_{i,gr} = \frac{1}{2\pi R_2 C_2} = 15.91549~\text{Hz}\\[0.5cm]
    s_{f_{i,gr}} = \frac{1}{2\pi}\sqrt{\left(\frac{s_{R_2}}{R_2^2 C_2}\right)^2 + \left(\frac{s_{C_2}}{R_2 C_2^2}\right)^2} = 0.67523~\text{Hz}\\[0.5cm]
    \Rightarrow \boxed{f_{i,gr} = (15.9 \pm 0.7)~\text{Hz}}
\end{gather}
Dieses Ergebnis deckt sich mit der Messung, da bei einer Frequenz von $100~\text{Hz}$ erkennt man in Abbildung \ref{abb:int} schon die Dreiecksspannung, währendessen bei einer Frequenz von $10~\text{Hz}$ das Signal der Ausgangsspannung $U_a$ die Ladekurve des Kondensators zeigt. Dies hat den Grund, dass der OPV ungefähr ab der Grenzfrequenz $f_{i,gr}$ linear arbeitet und korrekt integriert. Weiterhin lässt sich durch Erhöhung der Frequenz der Spannungsquelle durch die Einstellungen am Oszilloskop (siehe S.22) gut erkennen, dass die Verstärkung frequenzabhängig ist.
\newpage


\subsection*{Frequenzgang}
Für den theoretischen Verlauf der Verstärkung $v_{i,2}$ betrachten wir Gleichung (2.20) mit $\omega = 2\pi f$.
%, woraus für dessen Fehler mit dem Fortpflanzungsgesetz folgt:
%\begin{gather}
%    s_{v_{i,2}} = v_{i,2} \sqrt{\left(\frac{s_{R_2}}{R_2(1+(\omega R_2C_2)^2)}\right)^2+\left(\frac{s_{R_1}}{R_1}\right)^2+\left(\frac{\omega^2 R_2^2 C_2 s_{C_2}}{1+(\omega R_2 C_2)^2}\right)^2}
%\end{gather}
Die Fehlerrechnung für die Messwerte des Oszilloskop wird analog zu Kapitel 4.1 durchgeführt.  
Mit einsetzen der Messwerte folgt dann folgende Tabelle \ref{tab:int}, wobei der Messwert 1 ausgesondert wird, da dieser nicht korrekt ablesbar war und ab Messwert 19 ein Ablesefehler $s_a = 0.5~\text{Div}$ verwendet wird, da diese Werte bei der Messung sehr schwer abzulesen waren.
\begin{center}
    \captionof{table}{Messreihe Umkehrintegrator}
    \begin{tabular}{l | c c c c c | c c}
        {} &   $f$/Hz &  $U_a$/V &  $s_{U_a}$/V &  $U_e$/V &  $s_{U_e}$/V 
        &      $v$ &   $s_v$ \\
        \hline
        1  &      5 &  4.000 &      0.156 &   0.050 &      0.002 &  80.00 &  4.25 \\
        2  &     10 &  4.000 &      0.156 &   0.050 &      0.002 &  80.00 &  4.25 \\
        3  &     14 &  3.800 &      0.152 &   0.050 &      0.002 &  76.00 &  4.09 \\
        4  &     17 &  3.500 &      0.145 &   0.050 &      0.002 &  70.00 &  3.84 \\
        5  &     21 &  3.000 &      0.135 &   0.050 &      0.002 &  60.00 &  3.45 \\
        6  &     28 &  2.500 &      0.090 &   0.050 &      0.002 &  50.00 &  2.55 \\
        7  &     37 &  2.000 &      0.078 &   0.050 &      0.002 &  40.00 &  2.13 \\
        8  &     52 &  1.500 &      0.067 &   0.050 &      0.002 &  30.00 &  1.73 \\
        9  &     56 &  1.400 &      0.047 &   0.050 &      0.002 &  28.00 &  1.37 \\
        10 &     66 &  1.200 &      0.041 &   0.050 &      0.002 &  24.00 &  1.19 \\
        11 &     78 &  1.000 &      0.036 &   0.050 &      0.002 &  20.00 &  1.02 \\
        12 &    100 &  0.800 &      0.031 &   0.050 &      0.002 &  16.00 &  0.85 \\
        13 &    110 &  0.700 &      0.023 &   0.050 &      0.002 &  14.00 &  0.69 \\
        14 &    130 &  0.600 &      0.021 &   0.050 &      0.002 &  12.00 &  0.60 \\
        15 &    160 &  0.500 &      0.018 &   0.050 &      0.002 &  10.00 &  0.51 \\
        16 &    210 &  0.400 &      0.016 &   0.050 &      0.002 &   8.00 &  0.43 \\
        17 &    280 &  0.300 &      0.013 &   0.050 &      0.002 &   6.00 &  0.35 \\
        18 &    440 &  0.200 &      0.050 &   0.050 &      0.002 &   4.00 &  1.02 \\
        19 &    800 &  0.100 &      0.025 &   0.050 &      0.002 &   2.00 &  0.51 \\
        20 &   2100 &  0.040 &      0.005 &   0.050 &      0.002 &   0.80 &  0.11 \\
        21 &   3100 &  0.030 &      0.005 &   0.050 &      0.002 &   0.60 &  0.10 \\
        22 &   5800 &  0.020 &      0.005 &   0.050 &      0.002 &   0.40 &  0.10 \\
        23 &   7500 &  0.015 &      0.003 &   0.050 &      0.002 &   0.30 &  0.05 \\
        24 &  10000 &  0.012 &      0.003 &   0.050 &      0.002 &   0.24 &  0.05 \\
    \end{tabular}
    \label{tab:int}
\end{center}
Trägt man die Verstärkung $v$ doppellogarithmisch gegen die Frequenz $f$ auf erhält man Abbildung \ref{abb:intMess}, wobei der theoretische Verlauf von $v$ gegeben ist durch die Gleichung (2.20). Dabei ist deutlich zu erkennen, dass die Messung in Rahmen der Messgenauigkeit gut getroffen wurde. Auch ist die Grenzfrequenz $f_{i,gr}$ durch die Messung nochmal bestätigt und es lässt sich gut der lineare Bereich des OPVs erkennen, indem dieser korrekt integriert.
\newpage
\begin{center}
    \captionof{figure}{Frequenzgang des Umkehrintegrator}
    \includegraphics[scale = 0.75]{intOPVlogMess.png}
    \label{abb:intMess}
\end{center}