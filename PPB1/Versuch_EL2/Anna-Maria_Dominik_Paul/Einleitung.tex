\chapter{Einleitung}
Der Operationsverstärker ist ein, wie der Name schon verrät, Verstärker. Sein Name geht auf den mathematischen Begriff des Operators zurück. Er ist ein gleichspannungsgekoppelter Verstärker, dessen Faktor der Verstärkung sehr hoch ist. 
Diese Art von Verstärker besteht aus vielen Einzelteilen, wie beispielweise Dioden, Widerständen und Kondensatoren. Sie sind vor allem als Bauteile für 'Analog-Technik' bekannt, hier werden sie als integrierter Schaltkreis, auf einem winzigen Silizium-Chip verbaut. 
Das Verhalten einer Schaltung, eines Operationsverstärker, wird hauptsächlich durch das Rückkopplungsvermögen bestimmt. Mithilfe diesem können der Operationsverstärkungsfaktor eingestellt und die Schaltung aufgebaut werden.

In diesem Versuch ist es Hauptziel, das Rückkopplungsvermögen des Operationsverstärker zu untersuchen. Um dies zu erreichen, soll eine einfache Schaltung zum Verstärken, Differenzieren und Integrieren der Eingangsspannung aufgebaut werden und die Eigenschaften dieser untersucht werden.