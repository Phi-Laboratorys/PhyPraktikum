% Charlotte Geiger - Manuel Lippert - Leonard Schatt
% Physikalisches Praktikum

% 5. Kapitel Einleitung

\chapter{Fazit}
\label{chap:fazit}


% Platz für Text
Polarisation ist ein Phänomen, mit dem wir in unserem gesamten Leben umgeben sind. Ob durch das Eindringen der Sonnenstrahlen in das Hausinnere oder im Kino durch die Brillen für den 3D Film. \\
In diesem Versuch haben wir uns mit diesem Phänomen auseinandergesetzt und sowohl den Brewsterwinkel, den Brechungsindex als auch die Größe kugelförmiger Polystyrol-Teilchen gemessen bzw. berechnet.\\
Durch das Vergleichen mit den Theoretisch ermittelten Werten ist uns aufgefallen, dass selbst die kleinste Lichtverschmutzung die Messwerte verfälschen kann. Daher ist der Versuchsaufbau wahrscheinlich noch besser, wenn es sowohl eine bessere Abschirmung gegen Licht, als auch evtl. einen Schrittmotor für eine genauere Winkelmessung gäbe. Aber für das Kennenlernen für die Messung von polarisierenden Eigenschaften des Licht ist es auf jeden Fall eine gute Konstruktion. 
% Text