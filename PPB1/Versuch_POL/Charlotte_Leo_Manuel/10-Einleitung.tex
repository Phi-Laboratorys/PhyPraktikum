% Charlotte Geiger - Manuel Lippert - Leonard Schatt
% Physikalisches Praktikum

% 1. Kapitel Einleitung

\chapter{Einleitung}
\label{chap:einleitung}
Licht ist etwas, das uns tagtäglich umgibt-sowohl im Freien oder hinter Fenstern durch das Sonnenlicht oder in der Nacht durch die Glühbirnen in den Lampen. Das uns dadurch auch die Phänomene der Brechung, Beugung und Polarisation des Lichts begegnen ist uns meistens nicht bewusst. Und doch ist bieten sie  ziemlich wichtige Erkenntnisse. Beispielsweise kann man dadurch Spannungsanalyse an transparenten Kunststoffen durchführen, was unter anderem auch bei Spannungsanalysen bei älteren Brücken verwendet wird, wodurch die Stabilität gemessen wird. Auch kommt man damit in Kinos in Berührung, durch die 3D-Brillen, wenn man sich einen Film anschaut mit 3D Effekten oder beim Aufsetzen von Sonnenbrillen. \\
Dad wir recht häufig mit diesen Phänomenen in Berührung kommen, ist es sinnvoll, sich mit diesem Phänomen näher auseinanderzusetzen. In diesem Versuch lernen wir die Eigenschaften linear polarisiertes Licht eines Lasers bezüglich Reflexion und Streuung kennen. Wir messen den Brewsterwinkel und berechnen daraus die Intensitätsdifferenzen der Einfallenden im Gegensatz zur reflektierten Intensität. Auch berechnen wir den Brechungsindex des Glases und berechnen mit Hilfe der Miestreuung die Größe kugelförmiger Polystyrol-Teilchen. 
% Text