% Charlotte Geiger - Manuel Lippert - Leonard Schatt
% Physikalisches Praktikum

% Teilaufgabe X

\section{Vom kontinuierlichen zum diskreten Spektrum}

Wenn man die Wärmestrahlung betrachtet, sieht man kontinuierliche Spekteren. Dies ändert sich, wenn man nur einzelne Atome, beziehungsweise Gase von Atomen betrachtet. Bei diesen kann man im
Spektrum klar voneinander getrennte Linien erkennen. Diese Linien nennt man Spektrallinien.\\
Sie kommen zustande, weil in Atomen die Absorbtion und Emission von elektronagnetischen Wellen nicht kontinuierlich erfolgt. Die Energienniveaus der Atome sind diskret und 
somit können auch nur Wellen bestimmter Energien absorbiert und emitiert werden.\\
Das die Spektrallinien nicht monochromatisch sind lässt sich mit unterschiedlichen Argumenten erklären. Mit Hilfe der Quantenmachanik lassen sich folgende Aussagen treffen.\\
Die Energie-Zeit-Unschärfe folgt aus der Unschärferelation der Quantenmechanik. Da die Operatoren für Energie und Zeit nicht kommutieren, muss das Folgende gelten.\\
\begin{equation}
    \Delta E  \Delta t \eqslantgtr \frac{1}{2} [ \hat{H},\hat{T}] =  \frac{\hbar }{2} 
\end{equation}
Nehmen wir nun Gleichheit der linken und rechten Seite an und nennen $$ \Delta E =\frac{\Gamma}{2}$$ die Halbwertsbreite. Dann folgt: \newline
\begin{equation}
    \Delta E =\frac{\Gamma}{2} = \frac{\hbar }{2\mathcal{T} } 
\end{equation}
wobei die Lebensdauer des Teilchens $$\mathcal{T} = \Delta t$$ hier folgendermaßen angenommen wird.\\
Man sieht sehr schön dass eine unendlich scharfe Spektrallinie zu Widersprüchen führen würde, wie beispielsweise:
\begin{equation}
    \lim_{\Delta E \to 0}  \implies \mathcal{T} \longrightarrow \infty 
\end{equation}
Bei unendlich scharfen Spektrallinien müsste diese unendlich lange stabil bleiben. Es sollte also keine Emission geben.\\
Diese Behauptung wiederspricht jedoch dem Experiment und ist somit falsch.