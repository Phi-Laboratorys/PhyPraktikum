\section{Diskussion der Übersichtsmessungen}

Nach dem justieren der Apperatur mit der Halogenlampe verwenden wir die Quecksilberdampflampe, um deren Spektrum auszumessen.
Dabei stellen wir fest, dass das Spektrum nicht kontinuierlich ist, wie man es vieleicht erwarten würde.
Stattdessen stechen bestimmte Peaks klar heraus. Diese sind zum Teil charakteristisch für die Quecksilberdampflampe, was in 
Kapitel "`\titleref{Messlit}"' auf Seite S.\pageref{Messlit} genauer behandelt wird.\newline
In der Übersichtsmessung des Spektrums stechen vorallem die Peaks bei 5460\r{A} und 4360\r{A}.
Beim Vergleich der Messung mit und ohne dem Gelbfilter fällt auf, dass manche Linien bei der Messung mit Filter verschwinden. Das liegt an der Eigenschaft des 
Farbglases die Beugungen höherer Ordnung herauszufiltern. Diese werden in der Intensität stark geschwächt. Um dies zu verdeutlichen haben wir bei der Darstellung \ref{Übersichtsmessungen}  die Intensität
jeweils durch den Maximalwert geteilt. Aus diesem Grund lässt sich das "`Abdämpfen"' der Beugungen höherer Ordnung besser  beobachten.

\begin{figure}[h]
    \centering
    \includegraphics[width= \linewidth]{Übersichtsmessungen.png}
    \caption{Messung mit und ohne Gelbfilter im Vergleich}
    \label{Übersichtsmessungen}
\end{figure}

Unter berücksichtung von Grafik \ref{Übersichtsmessungen} wird deutlich, dass die Peaks bei 4360\r{A}, 4045\r{A} und 3650\r{A} von der Beugung 2.Ordnug des Spektrometers kommen.

