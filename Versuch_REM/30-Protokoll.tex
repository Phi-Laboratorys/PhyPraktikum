% Charlotte Geiger - Manuel Lippert - Leonard Schatt
% Physikalisches Praktikum

% 3.Kapitel  Protokoll

% Variables
\def\skalierung{0.65}

\chapter{Messprotokoll}
\label{chap:protokoll}

Für jede Aufnahme werden im Folgenden die Einstellungen des Mikroskops dokumentiert.

\section{a. Einpfennig- bzw. Eincentstücke mit Bohrungen}

\subsection{Aufnahme 1: Lorem ipsum...}

\paragraph{Bemerkungen:}
\begin{center}
    \begin{tabular}{|l|r|}
    \hline
     Dateiname &  \\
     \hline
        
    \end{tabular} 
 \end{center}

\paragraph{Einstellungen:}

\begin{center}
   \begin{tabular}{l|r}
    \rowcolor{lightgray}\textbf{Parameter} & \textbf{Eingestellter Wert} \\
    \hline\hline
    Vergrößerung &  \\
    \hline
    Arbeitsabstand (WD) & \\
    \hline
    Strahldurchmesser &  \\
    \hline
    Beschleunigungsspannung & \\
       
   \end{tabular} 
\end{center}



\section{b. Fliege}

\subsection{Aufnahme 1: Lorem ipsum...}

\paragraph{Bemerkungen:}
\begin{center}
    \begin{tabular}{|l|r|}
    \hline
     Dateiname &  \\
     \hline
        
    \end{tabular} 
 \end{center}

\paragraph{Einstellungen:}

\begin{center}
   \begin{tabular}{l|r}
    \rowcolor{lightgray}\textbf{Parameter} & \textbf{Eingestellter Wert} \\
    \hline\hline
    Vergrößerung &  \\
    \hline
    Arbeitsabstand (WD) & \\
    \hline
    Strahldurchmesser &  \\
    \hline
    Beschleunigungsspannung & \\
       
   \end{tabular} 
\end{center}



\section{c. Zinnstandart}

\subsection{Aufnahme 1: Lorem ipsum...}

\paragraph{Bemerkungen:}
\begin{center}
    \begin{tabular}{|l|r|}
    \hline
     Dateiname &  \\
     \hline
        
    \end{tabular} 
 \end{center}

\paragraph{Einstellungen:}

\begin{center}
   \begin{tabular}{l|r}
    \rowcolor{lightgray}\textbf{Parameter} & \textbf{Eingestellter Wert} \\
    \hline\hline
    Vergrößerung &  \\
    \hline
    Arbeitsabstand (WD) & \\
    \hline
    Strahldurchmesser &  \\
    \hline
    Beschleunigungsspannung & \\
       
   \end{tabular} 
\end{center}



\section{d. Gebrochene Schraube}

\subsection{Aufnahme 1: Lorem ipsum...}

\paragraph{Bemerkungen:}
\begin{center}
    \begin{tabular}{|l|r|}
    \hline
     Dateiname &  \\
     \hline
        
    \end{tabular} 
 \end{center}

\paragraph{Einstellungen:}

\begin{center}
   \begin{tabular}{l|r}
    \rowcolor{lightgray}\textbf{Parameter} & \textbf{Eingestellter Wert} \\
    \hline\hline
    Vergrößerung &  \\
    \hline
    Arbeitsabstand (WD) & \\
    \hline
    Strahldurchmesser &  \\
    \hline
    Beschleunigungsspannung & \\
       
   \end{tabular} 
\end{center}



\section{e. Selbstgewählt}

\subsection{Aufnahme 1: Lorem ipsum...}

\paragraph{Bemerkungen:}
\begin{center}
    \begin{tabular}{|l|r|}
    \hline
     Dateiname &  \\
     \hline
        
    \end{tabular} 
 \end{center}

\paragraph{Einstellungen:}

\begin{center}
   \begin{tabular}{l|r}
    \rowcolor{lightgray}\textbf{Parameter} & \textbf{Eingestellter Wert} \\
    \hline\hline
    Vergrößerung &  \\
    \hline
    Arbeitsabstand (WD) & \\
    \hline
    Strahldurchmesser &  \\
    \hline
    Beschleunigungsspannung & \\
       
   \end{tabular} 
\end{center}