% 2. Fragen zur Vorbereitung

\chapter{Theoretischer Hintergrund}
\label{chap:fvz}

\section{Arten und Vergleich der Mikroskope}
\label{sec:artenEM}

\subsection*{Raster-Elektronenmikroskop (REM)}
Bei einem REM wird durch einen Elektronenstrahl die zu untersuchende Probe zeilenförmig abgerastert. Dabei wird die Topografie (Oberfläche), die Kristallstruktur un Materialunterschiede der Probe auf einen Bildschirm mittels Sekundärelektronen (SE, inelastische Stöße) und Rückstoßelektronen (RE, elastische Stöße) mit entsprechenden Detektoren abgebildet. Weiterhin lässt ein REM eine Röntgenanalyse zu, wodurch auch eine Elementanalyse der Probe möglich ist. \citep{RasterEM}
\begin{itemize}
    \item Auflösung: $\sim$ \SI{10}{\nano\metre}\\
    Das Auflösungvermögen ist dabei von dem Strahlendurchmesser und dem Abbildungsignal abhängig und beträgt zwischen \SI{1}{\nano\metre} $\sim$ \SI{2}{\nano\metre} in günstigen Verhältnissen. \citep{WikiREM}
    \item Eindringtiefe: $\sim$ \SI{1}{\micro\metre}
    \item Probe: Vakuumstabil und trocken mit leitender Oberflächenschicht 
\end{itemize}
Das REM und seine Funktionsweise wird in weiteren Kapitel noch genauer betrachtet.

\subsection*{Transmission-Elektronenmikroskop (TEM)}
Bei einem TEM werden dünne Probe wird mit Elektronen durchstrahlt, welche durch die Streuung ihre Bewegungsrichtung ändern und ihre Energie durch inelastische Stöße verlieren. Die Elektronen, welche das Material durch elastische Stöße unter Erhaltung des Eintrittswinkels verlassen, werden in der hinteren Brennebene fokussiert. Die gestreuten Elektronen werden mit einer Blende abgeschirmt. Entweder wird dann das \textit{Zwischenbild} (vergrößertes Lichtbild) oder das \textit{Elektronenbeugungsbild} (Fokusebene) betrachtet. 
\begin{itemize}
    \item Auflösung: einige \si{\nano\metre} bis \si{\micro\metre}\\
    Dabei hängt die Auflösung von der Beschleuigungsspannung (80 $\sim$ 400\si{\kilo\volt}) und der Materialdicke ab. 
    \item Probe: Ultradünne Schnitte notwendig (10–\SI{100}{\nano\metre}) \citep{WikiTEM}
\end{itemize}
\newpage
\subsection*{Lichtmikroskop (LM)}
Beim LM werden stark vergrößerte Bilder von kleinen Strukturen oder Objekten mit Hilfe von Licht und optischen System aus Linsen erzeugt.
\begin{itemize}
    \item Auflösung: \SI{0,2}{\micro\metre} $\sim$ \SI{0,3}{\micro\metre}\\
    Wird durch die physikalischen Gesetzmäßigkeiten bestimmt und hängt somit von der Wellenlänge ab. 
    \item Probe: Für gut erkennbare Strukturen im Bild muss die Probe ausreichend Kontrast enthalten. \citep{WikiLM}
\end{itemize}

\subsection*{(REM,TEM) vs LM}
\begin{itemize}
    \item[\textcolor{green}{\textbf{+}}] Licht mit viel größerer Wellenlänge (\SI{380}{\nano\metre}) als Elektronen (Welle-Teilchen-Dualismus)(\SI{5}{\nano\metre}) $\Rightarrow$ Erheblich bessere Auflösung
    \item[\textcolor{red}{\textbf{-}}] (REM,TEM) benötigt Vakuum im Gang des Elektronenstrahl imd elktromagnetische Linsen, während eine LM \enquote{nur} Glaslinsen benötigt $\Rightarrow$ Höherer technischer Aufwand \citep{RuppelEM}
\end{itemize}

\subsection*{REM vs TEM}
\begin{itemize}
    \item[\textcolor{green}{\textbf{+}}] Probenpräpartion einfacher, da keine ultradünnen Schnitte der Probe erzeugt werden müssen
    \item[\textcolor{green}{\textbf{+}}] 3D-Abbildung der Oberfläche des Objekts $\Rightarrow$ leicht verständliche Bilder 
    \item[\textcolor{red}{\textbf{-}}] geringere Vergrößerung und Auflösung
    \item[\textcolor{red}{\textbf{-}}] Keine Aussage über innere Struktur der Probe \citep{RuppelEM} 
\end{itemize}

\section{Aufbau eines Raster-Elektronenmikroskops}
\label{sec:aufbau}
