% 2. Fragen zur Vorbereitung

\chapter{Theoretischer Hintergrund}
\label{chap:fvz}

\section{Arten und Vergleich der Mikroskope}
\label{sec:artenEM}

\subsection*{Raster-Elektronenmikroskop (REM)}
Bei einem REM wird durch einen Elektronenstrahl die zu untersuchende Probe zeilenförmig abgerastert. Dabei wird die Topografie (Oberfläche), die Kristallstruktur un Materialunterschiede der Probe auf einen Bildschirm mittels Sekundärelektronen (SE, inelastische Stöße) und Rückstoßelektronen (RE, elastische Stöße) mit entsprechenden Detektoren abgebildet. Weiterhin lässt ein REM eine Röntgenanalyse zu, wodurch auch eine Elementanalyse der Probe möglich ist. \citep{RasterEM}\\
Das REM und seine Funktionsweise wird in weiteren Kapitel noch genauer betrachtet.

\subsection*{Transmission-Elektronenmikroskop (TEM)}
Bei einem TEM werden dünne Probe wird mit Elektronen durchstrahlt, welche durch die Rutherford-Streuung ihre Bewegungsrichtung ändern und ihre Energie durch inelastische Stöße verlieren. Die Elektronen, welche das Material durch elastische Stöße unter Erhaltung des Eintrittswinkels verlassen, werden in der hinteren Brennebene fokussiert. Die gestreuten Elektronen werden mit einer Blende abgeschirmt. Entweder wird dann das \textit{Zwischenbild} (vergrößertes Lichtbild) oder das \textit{Elektronenbeugungsbild} (Fokusebene) betrachtet. % !Quelle Wikipedia

\subsection*{Lichtmikroskop (LM)}
Beim LM werden stark vergrößerte Bilder von kleinen Strukturen oder Objekten mit Hilfe von Licht und optischen System aus Linsen erzeugt. Für gut erkennbare Strukturen im Bild muss die Probe ausreichend Kontrast enthalten. % !Quelle Wikipedia

\subsection*{(REM,TEM) vs LM}
\begin{itemize}
    \item[\textcolor{green}{\textbf{+}}] Licht mit viel größerer Wellenlänge (380 nm) als Elektronen (Welle-Teilchen-Dualismus)(5nm) $\Rightarrow$ Erheblich bessere Auflösung \\
    \item[\textcolor{red}{\textbf{-}}] (REM,TEM) benötigt Vakuum im Gang des Elektronenstrahl imd elktromagnetische Linsen, während eine LM \enquote{nur} Glaslinsen benötigt $\Rightarrow$ Höherer technischer Aufwand
\end{itemize}

\subsection*{REM vs TEM}
