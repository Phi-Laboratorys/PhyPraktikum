% 3.Kapitel Protokoll

% Variables
\def\skalierung{0.65}

\chapter{Messprotokoll}
\label{chap:protokoll}

Für jede Aufnahme werden die Einstellungen des Mikroskops auf dem abgespeicherten Bild dokumentiert.
\begin{itemize}
   \item SEI: Sekundär und Rückstreuelektronen (Everhart-Thronley-Detektor)
   \item REF: Rückstreuelektronen (Everhart-Thronley-Detektor)
   \item BEC: Combo-Mode (A+B) (Halbleiterdetektor)
   \item BET: Topo-Mode (A-B) (Halbleiterdetektor)
   \item BES: Shadow-Mode (Halbleiterdetektor)
\end{itemize}
\section*{Pfennig}
\label{sub:pfennig}
Untersuchung der wichtigsten Einstellungsparameter des REM auf die Abbildungseigenschaft
\begin{itemize}
   \item[1)] Abhängigkeit der Darstellung auf Beschleunigungsspannung:\\
   Wir verändern mehrfach die Beschleunigungsspannung und dokumentieren dieselbe Bildstelle. Dabei wechseln wir zw. den Modi\\
   Datei: \textit{a\_Pfennig\_0005.tif} bis \textit{a\_Pfennig\_0017.tif}\\
   Bemerkung:
   \begin{itemize}
      \item Bei BEC konnte bei kleiner Beschleunigungsspannung kein Bild erfasst werden
      \item Höhe bleibt bei 39,5 cm über den ganzen Versuchsteil unverändert
   \end{itemize}
   \item[2)] EDX-Messung:\\
   Aufnahme des Röntgenspektrums auf glatter Oberfläche der Münze und im angebohrten Loch.
   Datei: \textit{a\_Pfennig\_0018.tif}, \textit{Oberfläche, Loch + Dateiendung}
   \item[2)] Abhängigkeit der Darstellung Strahldurchmesser:\\
   Wir verändern den Strahlendurchmesser von SS10 bis SS75.\\
   Datei: \textit{a\_Pfennig\_0019.tif} bis \textit{a\_Pfennig\_0021.tif}
\end{itemize}
\newpage
\section*{Fliege}
Untersuchung einer Fliege bei Beschleunigungsspannung 10 kV.
\begin{itemize}
   \item[1)] Abbildung gesamtes Objekt:\\
   Mehr Tiefenschärfe wird durch Vergrößerung des Arbeitsabstands erreicht. 
   Datei: \textit{b\_Fliege\_0022.tif}
   \item[2)] Segment Auge:\\
   Datei: \textit{b\_Fliege\_0023.tif} bis \textit{b\_Fliege\_0026.tif}, \textit{26-1.PNG,26-2.PNG} (Screenshots mit Maße)
   \item[3)] Freie Bereichsauswahl:\\
   Datei: \textit{b\_Fliege\_0027.tif} bis \textit{b\_Fliege\_0031.tif}
\end{itemize}

\section*{Zinnstandard}
Grobe Aufnahme eines kleinen Bereichs:\\
Datei: \textit{c\_Zinn\_0032.tif}
\begin{itemize}
   \item[1)] Variation des Strahlendurchmessers bei 20kV:\\ 
   Datei: \textit{c\_Zinn\_0033.tif} bis \textit{c\_Zinn\_0035.tif} 
   \item[2)] Variation der Beschleunigungsspannung bei SS30:\\
   Datei: \textit{c\_Zinn\_0035.tif} bis \textit{c\_Zinn\_0037.tif}  
   \item[3)] Variation Arbeitsabstands bei 20kV und SS30:\\
   Die Arbeitsabstände werden dabei auf dem erstellten Foto abgespeichert.\\
   Messfehler Skala: Ablesefehler $s_a$=0,05 mm Restfehler wird mit Ablesefehler abgeschätzt.\\ 
   Zwei Messreihen, da kurzzeitig Orientierung verloren wurde.\\
   Datei: \textit{c\_Zinn\_0038.tif} bis \textit{c\_Zinn\_0046.tif}      
\end{itemize}

\section*{Schraube}
Untersuchung von Bruchursache und Material der Schraube. Dabei nehmen wir Aufnahmen bei unterschiedlichen Modi auf zur Identifizierung der Bruchursache. Auffällig sind dabei drei Bereiche mit unterschiedlichen Materialien, welche mit dem EDX analysiert werden.\\
Datei: \textit{d\_Schraube\_0039.tif} bis \textit{d\_Schraube\_0055.tif}, \textit{Dunkel1, Dunkel2, Hell1, Normal1 + Dateiendung}

\section*{Wunschprobe}
Wahl fiel auf einen Microchip.\\
Datei: \textit{e\_Chip\_0056.tif} bis \textit{e\_Chip\_0063.tif}, \textit{Chip + Dateiendung}