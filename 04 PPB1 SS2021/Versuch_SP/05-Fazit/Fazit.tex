% Charlotte Geiger - Manuel Lippert - Leonard Schatt
% Physikalisches Praktikum

% 5. Kapitel Einleitung

\chapter{Fazit}
\label{chap:fazit}
Das Ziel dieses Versuches war die Leistungsfähigkeit und Funktionsweise des Spektrometers kennenzulernen. Dies wurde erreicht, indem wir Spektrallinien vermessen haben und das Transmissionsverhalten eines Filterglases untersucht haben. Außerdem haben wir eine Methode zur Ermittlung des Auflösungsvermögens gelernt. Der Umgang mit einem Messprogramm wurde außerdem nochmal geübt. Dabei begegneten uns Probleme, welche später bei eigenständigem Arbeiten auch auftreten. Die von dem Programm ausgegebenen Dateien sind beispielsweise im "`.txt"'-Format, wobei die Kommazahlen mit Kommata abgetrennt sind. Software zum Plotten erwartet jedoch csv-Dateien. Bei diesen ist die Zeilentrennung ein Semikolon und das Komma ein Punkt. Diese Schwierigkeiten zu Lösen war gutes Training für zukünftige Auswertungen.

% Platz für Text