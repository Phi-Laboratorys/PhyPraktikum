% Charlotte Geiger - Manuel Lippert - Leonard Schatt
% Physikalisches Praktikum

% Anhang A

\chapter{Berechnungen Fourier-Reihe und Effektivspannungen}
\label{app:Berechnung}

\section*{Sinusschwingung}
Effektivspannung:
\begin{gather}
    U_{eff} = \sqrt{\frac{1}{T}\int^T_0 U_0^2 \sin^2\left(\frac{2\pi}{T}\right) dt} = \sqrt{\frac{U_0^2}{T} \left[\frac{t}{2} - \frac{T\sin(\frac{4\pi}{T}t)}{8\pi}\bigg \vert^T_0 \right]} = \sqrt{\frac{U_0^2}{T}\frac{T}{2}} = \frac{U_0}{\sqrt{2}}
\end{gather}

\section*{Rechteckschwingung}
$b_k$ der Fourier-Reihe:
\begin{gather}
    \begin{aligned}
        b_k &= \frac{2}{T} \int^{T}_{0} f(t)\sin(k \frac{2\pi}{T} t)dt\\
            &= \frac{2}{T} \left[ \int^{\frac{T}{2}}_{0} U_0\sin(k \frac{2\pi}{T} t)dt - \int^{T}_{\frac{T}{2}} U_0\sin(k \frac{2\pi}{T} t)dt\right]\\
            &= \frac{2}{T}\frac{U_0T}{k2\pi} \left[-\cos(k \frac{2\pi}{T} t) \bigg \vert^{\frac{T}{2}}_{0} + \cos(k \frac{2\pi}{T} t) \bigg \vert^{T}_{\frac{T}{2}} \right]\\
            &= \frac{U_0}{k\pi} \left[-\cos(k\pi)+1 + 1 - \cos(k\pi)\right]\\
            &= \frac{2U_0}{k\pi}\left[1-\cos(k\pi)\right]
    \end{aligned}\\[0,5cm]
    \Rightarrow b_k =
    \begin{cases}
        0, & k~\text{gerade}\\
        \frac{4U_0}{\pi}\frac{1}{2k-1}, & k~\text{ungerade}\\
    \end{cases}
\end{gather}

Effektivspannung:
\begin{gather}
    U_{eff} = \sqrt{\frac{1}{T}\left[\int^{\frac{T}{2}}_0 U_0^2dt + \int^T_{\frac{T}{2}} U_0^2 dt\right]} = \sqrt{\frac{U_0^2}{T}\left[\frac{T}{2}+T-\frac{T}{2}\right]} = U_0
\end{gather}

\section*{Dreiecksschwingung}
$b_k$ der Fourier-Reihe:
\begin{gather}
    \begin{aligned}
        b_k &= \frac{2}{T} \int^{\frac{3T}{4}}_{-\frac{T}{4}} f(t)\sin(k \frac{2\pi}{T} t)dt\\
            &= \frac{2}{T} \left[\int^{\frac{T}{4}}_{-\frac{T}{4}} at\sin(k \frac{2\pi}{T} t)dt+ \int^{\frac{3T}{4}}_{\frac{T}{4}} a\left(\frac{T}{2}-t\right)\sin(k \frac{2\pi}{T} t)dt\right]\\
            &= \frac{2a}{T} \left[\int^{\frac{T}{4}}_{-\frac{T}{4}} t\sin(k \frac{2\pi}{T} t)dt - \int^{\frac{3T}{4}}_{\frac{T}{4}}t\sin(k \frac{2\pi}{T} t)dt\right] + \int^{\frac{3T}{4}}_{\frac{T}{4}}a\sin(k \frac{2\pi}{T} t)dt\\
            &= (\text{I}) + (\text{II})\\[0,5cm]
        (\text{I}) &= \frac{2a}{T} \left[\int^{\frac{T}{4}}_{-\frac{T}{4}} t\sin(k \frac{2\pi}{T} t)
            dt - \int^{\frac{3T}{4}}_{\frac{T}{4}}t\sin(k \frac{2\pi}{T} t)dt\right] \Rightarrow~\text{Partielle Integration}\\
            &= \frac{2a}{T}\frac{T}{k2\pi} \left[-t\cos(k \frac{2\pi}{T} t) \bigg \vert^{\frac{T}{4}}_{-\frac{T}{4}} + t\cos(k \frac{2\pi}{T} t) \bigg \vert^{\frac{3T}{4}}_{\frac{T}{4}}\right]\\
            &\tab+\frac{2a}{T}\frac{T}{k2\pi}\left[\int^{\frac{T}{4}}_{-\frac{T}{4}} \cos(k \frac{2\pi}{T} t)dt - \int^{\frac{3T}{4}}_{\frac{T}{4}} \cos(k \frac{2\pi}{T} t)dt\right]\\
            &= \frac{2a}{T}\frac{T}{k2\pi} \left[-\frac{T}{4}\cos(k \frac{\pi}{2}) -\frac{T}{4}\cos(- k \frac{\pi}{2}) + \frac{3T}{4} \cos(k \frac{3\pi}{2}) - \frac{T}{4} \cos(k \frac{\pi}{2}) \right]\\
            &\tab+\frac{2a}{T}\frac{T}{k2\pi}\left[\int^{\frac{T}{4}}_{-\frac{T}{4}} \cos(k \frac{2\pi}{T} t)dt - \int^{\frac{3T}{4}}_{\frac{T}{4}} \cos(k \frac{2\pi}{T} t)dt\right]\\
            &= \frac{2a}{T}\frac{T}{k2\pi}\left[\int^{\frac{T}{4}}_{-\frac{T}{4}} \cos(k \frac{2\pi}{T} t)dt - \int^{\frac{3T}{4}}_{\frac{T}{4}} \cos(k \frac{2\pi}{T} t)dt\right]\\
            &= \frac{2a}{T}\left(\frac{T}{k2\pi}\right)^2 \left[\sin(k \frac{2\pi}{T} t) \bigg \vert^{\frac{T}{4}}_{-\frac{T}{4}} - \sin(k \frac{2\pi}{T} t) \bigg \vert^{\frac{3T}{4}}_{\frac{T}{2}}\right]\\
            &= \frac{2a}{T}\left(\frac{T}{k2\pi}\right)^2 \left[\sin(k\frac{\pi}{2}) - \sin(-k\frac{\pi}{2}) - \sin(k\frac{3\pi}{2}) + \sin(k\frac{\pi}{2})\right]\\[0,5cm]
        (\text{II}) &= \int^{\frac{3T}{4}}_{\frac{T}{4}}a\sin(k \frac{2\pi}{T} t)dt = \frac{aT}{2\pi} \left[\cos(k\frac{2\pi}{T}t)\bigg \vert^{\frac{3T}{4}}_{\frac{T}{4}}\right]\\
                    &= \frac{aT}{2\pi} \left[\cos(k\frac{3\pi}{2}) - \cos(k\frac{\pi}{2})\right] = 0
    \end{aligned}\\[0,5cm]
    \Rightarrow b_k =
    \begin{cases}
        0, & k~\text{gerade}\\
        \frac{8aT}{4\pi^2}\frac{(-1)^{k-1}}{(2k-1)^2} = \frac{8U_0}{\pi^2}\frac{(-1)^{k-1}}{(2k-1)^2}, & k~\text{ungerade}\\
    \end{cases}
\end{gather}
Effektivspannung:
\begin{gather}
    \begin{aligned}
        U_{eff} &= \sqrt{\frac{1}{T}\left[\int^{\frac{T}{4}}_{-\frac{T}{4}} (at)^2dt + \int^{\frac{3T}{4}}_{\frac{T}{4}} \left(a\left(\frac{T}{2}-t\right)\right)^2 dt\right]}\\
                &= \sqrt{\frac{a^2}{T}\left[\int^{\frac{T}{4}}_{-\frac{T}{4}} t^2dt + \int^{\frac{3T}{4}}_{\frac{T}{4}} \left(\frac{T}{2}-t\right)^2 dt\right]}
                = \sqrt{\frac{a^2}{3T}\left[t^3 \bigg \vert^{\frac{T}{4}}_{-\frac{T}{4}} - \left(\frac{T}{2}-t\right)^3 \bigg \vert^{\frac{3T}{4}}_{\frac{T}{4}}\right]}\\
                &= \sqrt{\frac{a^2}{3T}\left[\left(\frac{T}{4}\right)^3 + \left(\frac{T}{4}\right)^3 - \left(\frac{T}{2}- \frac{3T}{4} \right)^3 +  \left(\frac{T}{2}- \frac{T}{4} \right)^3\right]}\\
                &= \sqrt{\frac{a^2}{3T}\left(\frac{T^3}{16}\right)} = \sqrt{\frac{1}{3}\left(\frac{aT}{4}\right)^2} = \sqrt{\frac{U_0^2}{3}} = \frac{U_0}{\sqrt{3}}
     \end{aligned}
\end{gather}