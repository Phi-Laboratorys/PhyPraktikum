% 1. Einleitung

\chapter{Einleitung}
\label{chap:einleitung}

Durch elektronische Messung ist jede Messung eines Signals einem gewissen Anteil von Rauschen behaftet. Um die Messung so präzise wie möglich durchführen zu können muss man zu den Mitteln der Signal/Rausch-Verbesserung greifen. Dafür ist wichtig die jeweiligen Störquellen zu identifizieren und diese bestenfalls zu eliminieren oder in praktischsten Fall zu unterdrücken.\\

In diesem Versuch werden die Methoden und die auswirkung der Signal/Rausch-Verbesserung diskutiert. Dabei werden unterschiedliche zeitliche Signalformen mit überlagertem Rau-
schen über die „Fast Fourier Transformationsmethode“ (FFT) und der Mittlung der Signal diskutiert. Zudem werden die grundlegenden Arten von elektronischen Filter und deren Effekt in der Praxis angewendet und analysiert. Auch das Lock-In Verfahren wird Anhand eines Lock-In Verstärkers näher betrachtet.