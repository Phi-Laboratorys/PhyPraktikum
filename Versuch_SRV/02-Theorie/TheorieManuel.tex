% Teilaufgabe 11
\newpage
\section{Fourier-Reihe und Effektivspannungen}
\label{sec:fourierseries}
\subsection*{Allgemeines zur Fourier-Reihe und Effektivspannungen}
\label{sub:fourierseriesAllgemein}
Eine Fourier-Reihe zerlegt eine gegeben periodische Funktion in ihre jeweiligen Sinus und Cosinusanteile. Die reelle Fourier-Reihe einer bestimmten $T$-periodischen Funktion lässt sich mit den folgenden Formeln berechnen:
\begin{gather}
    f(t) = \frac{a_0}{2} + \sum^{\infty}_{k=1} \left[a_k \sin(k\frac{2\pi}{T} t) +b_k \cos(k\frac{2\pi}{T} t)\right]\\
    a_k = \frac{2}{T} \int^{\frac{T}{2}}_{-\frac{T}{2}} f(t)\cos(k \frac{2\pi}{T} t)dt\\
    b_k = \frac{2}{T} \int^{\frac{T}{2}}_{-\frac{T}{2}} f(t)\sin(k \frac{2\pi}{T} t)dt
\end{gather}
Dabei sind die Grenzen der Integrale von $-\frac{T}{2}$ bis $\frac{T}{2}$ nicht fest, sie können verschoben werden. Es ist aber wichtig, dass über eine Periode integriert wird in diesem Fall über eine komplette Periodendauer $T$.\\

Um die effektiven Spannungswerte der jeweiligen Schwingungsform zu bestimmen, bildet man das sogenannte \enquote{Quadratische Mittel}. Dieses ist wie folgt definiert:
\begin{gather}
    U_{eff} = \sqrt{\frac{1}{T}\int^T_0 f(t)^2 dt}
\end{gather}
Hierbei ist es wieder zu erwähnen, dass die Grenzen der Integration nicht fest sind, aber die Integration über eine Periodenlänge erfolgen muss.

\subsection*{Sinusschwingung}
\label{sub:sinus}
Der Fall der Sinusschwingung ist besonders einfach, da wie vorangegangen erwähnt, die Fourier-Reihe eine periodische Funktion in ihre Sinus und Cosinusanteile zerlegt. Daraus folgt die Fourier-Reihe der Sinusschwingung ist die Sinusschwingung selbst und kann somit trivial angegeben werden als:
\begin{gather}
    \boxed{f(t) = U_0\sin(\frac{2\pi}{T} t)}
\end{gather}
Die Effektivspannungen der Sinusschwingung ergibt sich wie folgt:
\begin{gather}
    U_{eff} = \sqrt{\frac{1}{T}\int^T_0 U_0^2 \sin^2\left(\frac{2\pi}{T}\right) dt} = \sqrt{\frac{U_0^2}{T} \left[\frac{t}{2} - \frac{T\sin(\frac{4\pi}{T}t)}{8\pi}\bigg \vert^T_0 \right]} = \frac{U_0}{\sqrt{2}}\\[0,5cm]
    \Rightarrow\boxed{U_{eff}=\frac{U_0}{\sqrt{2}}}
\end{gather}
\newpage

\subsection*{Rechteckschwingung}
\label{sub:square}
Als nächstes wird die Fourier-Reihe der Rechteckschwingung bestimmt. Diese hat die Form:
\begin{gather}
    f(t) = 
    \begin{cases}
        +U_0, & 0 \leq t \leq \frac{T}{2} \\
        -U_0, & \frac{T}{2} \leq t \leq T \\
    \end{cases}
\end{gather}
Da die Funktion der Rechteckschwingung punktsymmetrisch zum Ursprung ist, fallen alle Cosinusanteile weg, da $a_k = 0$. Somit muss nur $b_k$ wie folgt berechnet werden:
\begin{gather}
    \begin{aligned}
        b_k &= \frac{2}{T} \int^{T}_{0} f(t)\sin(k \frac{2\pi}{T} t)dt\\
            &= \frac{2}{T} \left[ \int^{\frac{T}{2}}_{0} U_0\sin(k \frac{2\pi}{T} t)dt - \int^{T}_{\frac{T}{2}} U_0\sin(k \frac{2\pi}{T} t)dt\right]\\
            &= \frac{2}{T}\frac{U_0T}{k2\pi} \left[-\cos(k \frac{2\pi}{T} t) \bigg \vert^{\frac{T}{2}}_{0} + \cos(k \frac{2\pi}{T} t) \bigg \vert^{T}_{\frac{T}{2}} \right]\\
            &= \frac{U_0}{k\pi} \left[-\cos(k\pi)+1 + 1 - \cos(k\pi)\right]\\
            &= \frac{2U_0}{k\pi}\left[1-\cos(k\pi)\right]
    \end{aligned}\\[0,5cm]
    \Rightarrow \boxed{b_k =
    \begin{cases}
        0, & k~\text{gerade}\\
        \frac{4U_0}{\pi}\frac{1}{2k-1}, & k~\text{ungerade}\\
    \end{cases}}
\end{gather} 
Es werden nur noch Terme mir ungeraden $k$ betrachtet und man erhält:
\begin{gather}
    \boxed{f(t) = \frac{4U_0}{\pi} \sum^{\infty}_{k=1} \frac{1}{2k-1} \sin((2k-1)\frac{2\pi}{T}t)}
\end{gather}
Die Effektivspannungen der Rechteckschwingung lässt sich trivial errechnen:
\begin{gather}
    U_{eff} = \sqrt{\frac{1}{T}\left[\int^{\frac{T}{2}}_0 U_0^2dt + \int^T_{\frac{T}{2}} U_0^2 dt\right]} = U_0\\[0,5cm]
    \Rightarrow\boxed{U_{eff} = U_0}
\end{gather}
\newpage

\subsection*{Dreiecksspannung}
\label{sub:triangle}
Als Letztes wollen wir die Dreieckschwingung betrachtet. Die Form dieser ist definiert wie folgt:
\begin{gather}
    f(t) = 
    \begin{cases}
        at, & -\frac{T}{4} \leq t \leq \frac{T}{4} \\
        a\left(\frac{T}{2}-t\right), & \frac{T}{4} \leq t \leq \frac{3T}{4} \\
    \end{cases}
    ~\text{mit}~U_0 = \frac{aT}{4}
\end{gather} 
Die Funktion ist erneut punktsymmetrisch zum Ursprung, wodurch wieder alle $a_k$-Koeffizienten wieder 0 sind. $b_k$ ergibt sich dann durch folgende Rechnung:
\begin{gather}
    \begin{aligned}
        b_k &= \frac{2}{T} \int^{\frac{3T}{4}}_{-\frac{T}{4}} f(t)\sin(k \frac{2\pi}{T} t)dt\\
            &= \frac{2}{T} \left[\int^{\frac{T}{4}}_{-\frac{T}{4}} at\sin(k \frac{2\pi}{T} t)dt+ \int^{\frac{3T}{4}}_{\frac{T}{4}} a\left(\frac{T}{2}-t\right)\sin(k \frac{2\pi}{T} t)dt\right]\\
            &= \frac{2a}{T} \left[\int^{\frac{T}{4}}_{-\frac{T}{4}} t\sin(k \frac{2\pi}{T} t)dt - \int^{\frac{3T}{4}}_{\frac{T}{4}}t\sin(k \frac{2\pi}{T} t)dt\right] + \int^{\frac{3T}{4}}_{\frac{T}{4}}a\sin(k \frac{2\pi}{T} t)dt\\
            &= (\text{I}) + (\text{II})\\
        (\text{I}) &= \frac{2a}{T} \left[\int^{\frac{T}{4}}_{-\frac{T}{4}} t\sin(k \frac{2\pi}{T} t)
            dt - \int^{\frac{3T}{4}}_{\frac{T}{4}}t\sin(k \frac{2\pi}{T} t)dt\right] \Rightarrow~\text{Partielle Integration}\\
            %&= \frac{2a}{T}\frac{T}{k2\pi} \left[-t\cos(k \frac{2\pi}{T} t) \bigg \vert^{\frac{T}{4}}_{-\frac{T}{4}} + t\cos(k \frac{2\pi}{T} t) \bigg \vert^{\frac{3T}{4}}_{\frac{T}{2}}\right]\\
            %&\tab+\frac{2a}{T}\frac{T}{k2\pi}\left[\int^{\frac{T}{4}}_{-\frac{T}{4}} \cos(k \frac{2\pi}{T} t)dt - \int^{\frac{3T}{4}}_{\frac{T}{4}} \cos(k \frac{2\pi}{T} t)dt\right]\\
            &= \frac{2a}{T}\frac{T}{k2\pi}\left[\int^{\frac{T}{4}}_{-\frac{T}{4}} \cos(k \frac{2\pi}{T} t)dt - \int^{\frac{3T}{4}}_{\frac{T}{4}} \cos(k \frac{2\pi}{T} t)dt\right]\\
            &= \frac{2a}{T}\left(\frac{T}{k2\pi}\right)^2 \left[\sin(k \frac{2\pi}{T} t) \bigg \vert^{\frac{T}{4}}_{-\frac{T}{4}} - \sin(k \frac{2\pi}{T} t) \bigg \vert^{\frac{3T}{4}}_{\frac{T}{2}}\right]\\
            &= \frac{2a}{T}\left(\frac{T}{k2\pi}\right)^2 \left[\sin(k\frac{\pi}{2}) - \sin(-k\frac{\pi}{2}) - \sin(k\frac{3\pi}{2}) + \sin(k\frac{\pi}{2})\right]\\
        (\text{II}) &= \int^{\frac{3T}{4}}_{\frac{T}{4}}a\sin(k \frac{2\pi}{T} t)dt = 0
    \end{aligned}\\[0,5cm]
    \Rightarrow \boxed{b_k =
    \begin{cases}
        0, & k~\text{gerade}\\
        \frac{8aT}{4\pi^2}\frac{(-1)^{k-1}}{(2k-1)^2}, & k~\text{ungerade}\\
    \end{cases}}
\end{gather}
Es werden wieder nur die Terme mit ungeraden $k$ betrachtet und $\frac{aT}{4}$ mit $U_0$ ersetzt. Somit erhält man:
\begin{gather}
    \boxed{f(t) = \frac{8U_0}{\pi^2} \sum^{\infty}_{k=1} \frac{(-1)^{k-1}}{(2k-1)^2} \sin((2k-1)\frac{2\pi}{T}t)}
\end{gather} 
Die Effektivspannungen der Dreieckschwingung wird wie folgt ermittelt:
\begin{gather}
    \begin{aligned}
        U_{eff} &= \sqrt{\frac{1}{T}\left[\int^{\frac{T}{4}}_{-\frac{T}{4}} (at)^2dt + \int^{\frac{3T}{4}}_{\frac{T}{4}} \left(a\left(\frac{T}{2}-t\right)\right)^2 dt\right]}\\
                &= \sqrt{\frac{a^2}{T}\left[\int^{\frac{T}{4}}_{-\frac{T}{4}} t^2dt + \int^{\frac{3T}{4}}_{\frac{T}{4}} \left(\frac{T}{2}-t\right)^2 dt\right]}
                = \sqrt{\frac{a^2}{3T}\left[t^3 \bigg \vert^{\frac{T}{4}}_{-\frac{T}{4}} - \left(\frac{T}{2}-t\right)^3 \bigg \vert^{\frac{3T}{4}}_{\frac{T}{4}}\right]}\\
                &= \sqrt{\frac{a^2}{3T}\left(\frac{T^3}{16}\right)} = \sqrt{\frac{1}{3}\left(\frac{aT}{4}\right)^2} = \sqrt{\frac{U_0^2}{3}} = \frac{U_0}{\sqrt{3}}
     \end{aligned}\\[0,5cm]
     \Rightarrow\boxed{U_{eff} = \frac{U_0}{\sqrt{3}}}
\end{gather}