% Teilaufgabe 11
\newpage
\section{Fourier-Reihe und Effektivspannungen}
\label{sec:fourierseries}
\subsection*{Allgemeines zur Fourier-Reihe und Effektivspannungen}
\label{sub:fourierseriesAllgemein}
Eine Fourier-Reihe zerlegt eine gegeben periodische Funktion in ihre jeweiligen Sinus und Cosinusanteile. Die reelle Fourier-Reihe einer bestimmten $T$-periodischen Funktion lässt sich mit den folgenden Formeln berechnen:
\begin{gather}
    f(t) = \frac{a_0}{2} + \sum^{\infty}_{k=1} \left[a_k \sin(k\frac{2\pi}{T} t) +b_k \cos(k\frac{2\pi}{T} t)\right]\\
    a_k = \frac{2}{T} \int^{\frac{T}{2}}_{-\frac{T}{2}} f(t)\cos(k \frac{2\pi}{T} t)dt\\
    b_k = \frac{2}{T} \int^{\frac{T}{2}}_{-\frac{T}{2}} f(t)\sin(k \frac{2\pi}{T} t)dt
\end{gather}
Dabei sind die Grenzen der Integrale von $-\frac{T}{2}$ bis $\frac{T}{2}$ nicht fest, sie können verschoben werden. Es ist aber wichtig, dass über eine Periode integriert wird in diesem Fall über eine komplette Periodendauer $T$.\\

Um die effektiven Spannungswerte der jeweiligen Schwingungsform zu bestimmen, bildet man das sogenannte \enquote{Quadratische Mittel}. Dieses ist wie folgt definiert:
\begin{gather}
    U_{eff} = \sqrt{\frac{1}{T}\int^T_0 f(t)^2 dt}
\end{gather}
Hierbei ist es wieder zu erwähnen, dass die Grenzen der Integration nicht fest sind, aber die Integration über eine Periodenlänge erfolgen muss.\\

Die detaillierteren Berechnungen zu jeder Schwingungsform lassen sich in Anhang \ref{app:Berechnung} nachlesen.

\subsection*{Sinusschwingung}
\label{sub:sinus}
Der Fall der Sinusschwingung ist besonders einfach, da wie vorangegangen erwähnt, die Fourier-Reihe eine periodische Funktion in ihre Sinus und Cosinusanteile zerlegt. Daraus folgt die Fourier-Reihe der Sinusschwingung ist die Sinusschwingung selbst und kann somit trivial angegeben werden als:
\begin{gather}
    \boxed{f(t) = U_0\sin(\frac{2\pi}{T} t)}
\end{gather}
Die Effektivspannungen der Sinusschwingung:
\begin{gather}
    \boxed{U_{eff}=\frac{U_0}{\sqrt{2}}}
\end{gather}
\newpage

\subsection*{Rechteckschwingung}
\label{sub:square}
Als nächstes wird die Fourier-Reihe der Rechteckschwingung bestimmt. Diese hat die Form:
\begin{gather}
    f(t) = 
    \begin{cases}
        +U_0, & 0 \leq t \leq \frac{T}{2} \\
        -U_0, & \frac{T}{2} \leq t \leq T \\
    \end{cases}
\end{gather}
Da die Funktion der Rechteckschwingung punktsymmetrisch zum Ursprung ist, fallen alle Cosinusanteile weg, da $a_k = 0$. Somit muss nur $b_k$ wie folgt berechnet werden:
\begin{gather}
    b_k =
    \begin{cases}
        0, & k~\text{gerade}\\
        \frac{4U_0}{\pi}\frac{1}{2k-1}, & k~\text{ungerade}\\
    \end{cases}
\end{gather} 
Es werden nur noch Terme mit ungeraden $k$ betrachtet und man erhält:
\begin{gather}
    \boxed{f(t) = \frac{4U_0}{\pi} \sum^{\infty}_{k=1} \frac{1}{2k-1} \sin((2k-1)\frac{2\pi}{T}t)}
\end{gather}
Die Effektivspannungen der Rechteckschwingung:
\begin{gather}
    \boxed{U_{eff} = U_0}
\end{gather}
\subsection*{Dreiecksspannung}
\label{sub:triangle}
Als Letztes wollen wir die Dreieckschwingung betrachtet. Die Form dieser ist definiert wie folgt:
\begin{gather}
    f(t) = 
    \begin{cases}
        at, & -\frac{T}{4} \leq t \leq \frac{T}{4} \\
        a\left(\frac{T}{2}-t\right), & \frac{T}{4} \leq t \leq \frac{3T}{4} \\
    \end{cases}
    ~\text{mit}~U_0 = \frac{aT}{4}
\end{gather} 
Die Funktion ist erneut punktsymmetrisch zum Ursprung, wodurch wieder alle $a_k$-Koeffizienten 0 sind. $b_k$ ergibt sich dann durch wie folgt:
\begin{gather}
    b_k =
    \begin{cases}
        0, & k~\text{gerade}\\
        \frac{8U_0}{\pi^2}\frac{(-1)^{k-1}}{(2k-1)^2}, & k~\text{ungerade}\\
    \end{cases}
\end{gather}
Es werden wieder nur die Terme mit ungeraden $k$ betrachtet. Somit erhält man:
\begin{gather}
    \boxed{f(t) = \frac{8U_0}{\pi^2} \sum^{\infty}_{k=1} \frac{(-1)^{k-1}}{(2k-1)^2} \sin((2k-1)\frac{2\pi}{T}t)}
\end{gather} 
Die Effektivspannungen der Dreieckschwingung:
\begin{gather}
     \boxed{U_{eff} = \frac{U_0}{\sqrt{3}}}
\end{gather}