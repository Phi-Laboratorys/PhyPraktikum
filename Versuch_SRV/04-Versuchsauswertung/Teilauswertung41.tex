% Teilauswertung 1

\section{Mittelung und Fouriertransformation}
\label{sec:mittelungAndTrafo}

\paragraph{a)}\textbf{Experimentelle Bestimmung der Fourierentwicklungskoeffizienten}\\
from scipy.signal import argrelextrema
Im Folgenden wird die Fourierentwicklungskoeffizienten der Sinus, Rechteck und Dreiecksschwingung experimentell bestimmt. Dazu werden die einzelnen Peaks im Fourierspektrum mit Hilfe von einen Python-Skript (argelextrma Modul aus scipy.signal Package) ermittelt, welche dann den Wert der Fourierentwicklungskoeffizienten für diese Frequenz darstellen. Beachtet werden nur ungerade Vielfache der eingestellten Frequenz. Dies folgt aus Kapitel \ref{sec:fourierseries} bei der nur ungerade Vielfache in der Reihenentwicklung von Rechteck und Dreieckschwingung vorkommen. Die Daten der Amplitude $A$ wurden dabei in dB aufgenommen. Für die Umrechnung der Amplitude in dB zu Volt wird die Gleichung (\ref{eq:spannungpegel}) aus Kapitel \ref{sec:pegel} nach der Spannung $U$ aufgelöst und erhält:
\begin{gather}
    L_U = 20 \log_{10}\left(\frac{U}{U_0}\right)~\Leftrightarrow~U = U_0 \cdot 10^{\frac{L_U}{20}}
\end{gather}
Zu beachten ist noch, dass die Spannung $U$ die gemessene Effektivspannung ist. Diese Effektivspannung muss mit dem Faktor für die Sinusschwingung aus Kapitel \ref{sec:fourierseries} umgewandelt werden, da die Fourierreihe der jeweiligen Schwingung aus Sinusschwingungen besteht. Die verwendeten Formeln sind dann:
\begin{gather}
    U = U_0 \cdot 10^{\frac{L_U}{20}} \cdot \sqrt{2}  ~\text{mit}~U_0=1\,\text{V}
    \label{eq:umrechnung}
\end{gather}
Die Auswertung ergibt dann Tabelle \ref{tab:fourierkoeff}, in welcher man gut erkennen kann, dass die gemessenen Werte nicht weit von der Theorie abweichen. Die Messung weist hierbei Abweichungen von $\pm0,003$\,V auf. Dies kann durch Rauschen in der Messaparatur verursacht worden sein.
\newpage
\begin{center}
    \textbf{Sinus}\\[0,2cm]
    \begin{tabular}{l | c | c c}
        $k$ & $f$/kHz &   $A_{mess}$/V & $A_{Theo}$/V \\
        \hline
        1 & 1 &  1,0024 & 1,0000 \\
    \end{tabular}\\[0,5cm]
    \begin{tabular}{c c}
        \textbf{Rechteck} & \textbf{Dreieck} \\[0,2cm]
        \begin{tabular}{l | c | c c}
            $k$ & $f$/kHz  &   $A_{mess}$/V & $A_{Theo}$/V \\
            \hline
            1  &       1 &  1,276204 &  1,273240 \\
            2  &       3 &  0,425216 &  0,424413 \\
            3  &       5 &  0,254976 &  0,254648 \\
            4  &       7 &  0,181976 &  0,181891 \\
            5  &       9 &  0,141379 &  0,141471 \\
            6  &      11 &  0,115513 &  0,115749 \\
            7  &      13 &  0,097574 &  0,097942 \\
            8  &      15 &  0,084397 &  0,084883 \\
            9  &      17 &  0,074300 &  0,074896 \\
            10 &      19 &  0,066309 &  0,067013 \\
            11 &      21 &  0,059820 &  0,060630 \\
            12 &      23 &  0,054445 &  0,055358 \\
            13 &      25 &  0,049915 &  0,050930 \\
            14 &      27 &  0,046045 &  0,047157 \\
            15 &      29 &  0,042696 &  0,043905 \\
            16 &      31 &  0,039764 &  0,041072 \\
            17 &      33 &  0,037185 &  0,038583 \\
            18 &      35 &  0,034885 &  0,036378 \\
            19 &      37 &  0,032824 &  0,034412 \\
            20 &      39 &  0,030967 &  0,032647 \\
            21 &      41 &  0,029284 &  0,031055 \\
            22 &      43 &  0,027747 &  0,029610 \\
            23 &      45 &  0,026338 &  0,028294 \\
            24 &      47 &  0,025044 &  0,027090 \\
            25 &      49 &  0,023854 &  0,025984 \\
            26 &      51 &  0,022747 &  0,024965 \\
            27 &      53 &  0,021718 &  0,024023 \\
            28 &      55 &  0,020760 &  0,023150 \\
            29 &      57 &  0,019863 &  0,022338 \\
            30 &      59 &  0,019024 &  0,021580 \\
            31 &      61 &  0,018230 &  0,020873 \\
            32 &      63 &  0,017489 &  0,020210 \\
            33 &      65 &  0,016786 &  0,019588 \\
            34 &      67 &  0,016126 &  0,019004 \\
            35 &      69 &  0,015501 &  0,018453 \\
        \end{tabular} \hspace{0,5cm} &
        \begin{tabular}{l | c | c c}
            $k$ & $f$/kHz &  $A_{mess}$/V & $A_{Theo}$/V \\
            \hline
            1  &       1 &  0,812606 &  0,810569 \\
            2  &       3 &  0,090246 &  0,090063 \\
            3  &       5 &  0,032489 &  0,032423 \\
            4  &       7 &  0,016573 &  0,016542 \\
            5  &       9 &  0,009979 &  0,010007 \\
            6  &      11 &  0,006700 &  0,006699 \\
            7  &      13 &  0,004805 &  0,004796 \\
            8  &      15 &  0,003604 &  0,003603 \\
            9  &      17 &  0,002806 &  0,002805 \\
            10 &      19 &  0,002247 &  0,002245 \\
            11 &      21 &  0,001853 &  0,001838 \\
            12 &      23 &  0,001530 &  0,001532 \\
            13 &      25 &  0,001298 &  0,001297 \\
            14 &      27 &  0,001116 &  0,001112 \\
            15 &      29 &  0,000963 &  0,000964 \\
            16 &      31 &  0,000846 &  0,000843 \\
            17 &      33 &  0,000737 &  0,000744 \\
            18 &      35 &  0,000654 &  0,000662 \\
            19 &      37 &  0,000591 &  0,000592 \\
            20 &      39 &  0,000521 &  0,000533 \\
            21 &      41 &  0,000480 &  0,000482 \\
            22 &      43 &  0,000433 &  0,000438 \\
            23 &      45 &  0,000395 &  0,000400 \\
            24 &      47 &  0,000362 &  0,000367 \\
            25 &      49 &  0,000335 &  0,000338 \\
            26 &      51 &  0,000312 &  0,000312 \\
            27 &      53 &  0,000280 &  0,000289 \\
            28 &      55 &  0,000264 &  0,000268 \\
            29 &      57 &  0,000255 &  0,000249 \\
            30 &      59 &  0,000237 &  0,000233 \\
            31 &      61 &  0,000218 &  0,000218 \\
            32 &      63 &  0,000205 &  0,000204 \\
            33 &      65 &  0,000194 &  0,000192 \\
            34 &      67 &  0,000181 &  0,000181 \\
            35 &      69 &  0,000170 &  0,000170 \\
        \end{tabular}
    \end{tabular}    
    \captionof{table}{Vergleich Fourierentwicklungskoeffizienten in Theorie und Praxis}
    \label{tab:fourierkoeff}
\end{center}
\paragraph{b)}\textbf{Fourierspektrum Sinusschwingung}\\
Dieser Teil behandelt das Fourierspektrum der Sinusschwingung bei 1\,V Amplitude.
\begin{center}
    \includegraphics[scale = 0.5]{Manuel/41/FourierSinus.pdf}
    \captionof{figure}{Fourierspektrum der Sinusschwingung}
    \label{image:fourierSinus}
\end{center}
Wie in Abbildung \ref{image:fourierSinus} ist deutlich ein ausgeprägter Peak zu erkennen. Dieser Peak ist wie in Tabelle \ref{tab:fourierkoeff} die Effektivspannung der Sinusschwingung in dB. Nach Kapitel \ref{sec:fourierseries} sollte aber das Fourierspektrum nur einen Peak bei der eingestellten Frequenz besitzen, aber in Abbildung \ref{image:fourierSinus} ist neben dem Peak deutlich noch weiter Peaks zu erkennen. Diese Peaks werden durch die Umgebungseinflüsse aus Kapitel \ref{sec:umwelt} und durch die Netzspannungsquelle selbst verursacht.
\newpage
\paragraph{c)}\textbf{Einfluss Mittelung auf Rechteckschwingung}\\
Nun wird der Einfluss der Anzahl der Mittelung $N$ auf eine Rechteckschwingung mit einer Amplitude $U_0=$ 1\,V mit einem Bandrauschen der Bandbreite $f_{Band}=$ 20\,MHz und einer AM Depth $d =$ 120\,\% betrachtet.
\begin{center}
    \includegraphics[scale = 0.5]{Manuel/41/Mittelung.pdf}
    \captionof{figure}{Einfluss der Mittelung auf Fourierspektrum einer Rechteckschwingung}
    \label{image:einflussMittelung}
\end{center}
In Abbildung \ref{image:einflussMittelung} ist zu erkennen, dass sich mit der Zunahme der Anzahl der Mittelungen $N$ die Ausprägung der Peaks zunimmt. Dies war zu erwarten, da schon wie in Kapitel \ref{sub:mittelung} erklärt, bei der Mittelung das Signal des Rauschens nur um den Faktor $\sqrt{N}$ zunimmt, während das Signal der Rechteckschwingung proportional zur Wiederholung anwächst.
\newpage
\paragraph{d)}\textbf{Einfluss Signal/Rausch-Abstand auf zeitliche Signalform}\\
Als weiteres wird der Einfluss des Signal/Rausch-Abstands $d$ auf die Signalform einer Rechteckschwingung mit einer Amplitude $U_0=$ 1\,V und einen Bandbreite\\$f_{Band}=$ 20\,MHz.
In Abbildung \ref{image:sigRauRechtKomplett} ist zuerst der Einfluss des Signal/Rausch-Abstands auf die komplette Rechteckschwingung dargestellt. Schon hier erkennt man, dass mit höherem Signal/Rausch-Abstands das \enquote{Ausschlagen} der Amplitude des Rauschens zunimmt. Noch deutlich ist dies zu erkennen in Abbildung \ref{image:sigRauRecht}, die ein Ausschnitt aus dem oberen Umkehrpunkt der Rechteckschwingung ist. Die Änderung des Signal/Rausch-Abstands hat aber keinen generellen Einfluss auf den Verlauf der Signalform der Rechteckschwingung.
\newpage
\begin{center}
    \includegraphics[scale = 0.5]{Manuel/41/SignalRauschAbstandKomplett.pdf}
    \captionof{figure}{Einfluss Signal/Rausch-Abstand auf komplette Rechteckschwingung}
    \label{image:sigRauRechtKomplett}
\end{center}
\begin{center}
    \includegraphics[scale = 0.5]{Manuel/41/SignalRauschAbstand.pdf}
    \captionof{figure}{Einfluss Signal/Rausch-Abstand im Ausschnitt vom obere Umkehrpunkt der Rechteckschwingung}
    \label{image:sigRauRecht}
\end{center}

\paragraph{e)}\textbf{Einfluss Bandbreite auf zeitliche Signalform}\\
Zuletzt in diesem Kapitel wird der Einfluss der Bandbreite auf die zeitliche Signalform besprochen. Dabei wird wieder eine Rechteckschwingung mit einer Amplitude\\ $U_0=$ 1\,V und einem Signal/Rausch Abstand $d=$ 100\,\% verwendet. In Abbildung \ref{image:bandbreiteRechtKomplett} wird erneut eine komplette Rechteckschwingung gezeigt. Darauf erkennt man, dass der Funktionsgenerator die Form des Rechtecksignals mit kleineren Bandbreiten $f_{Band}$ nicht mehr erzeugen kann. Vor allem bei Frequenzen unter 100\,kHz. Aber auch eine Zunahme des Rauschens ist zu erkennen, wenn die Bandbreite kontinuierlich abnimmt. In Abbildung \ref{image:bandbreiteRecht} kann man dies noch genauer betrachten. 
\newpage
\begin{center}
    \includegraphics[scale = 0.5]{Manuel/41/BandbreiteKomplett.pdf}
    \captionof{figure}{Einfluss Bandbreite auf komplette Rechteckschwingung}
    \label{image:bandbreiteRechtKomplett}
\end{center}
\begin{center}
    \includegraphics[scale = 0.5]{Manuel/41/Bandbreite.pdf}
    \captionof{figure}{Einfluss Bandbreite im Ausschnitt vom obere Umkehrpunkt der Rechteckschwingung}
    \label{image:bandbreiteRecht}
\end{center}