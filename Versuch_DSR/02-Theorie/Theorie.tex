\chapter{Theoretical background}

\section{Saturation spectroscopy in general}
With saturation spectroscopy enables us to see the hyperfine structure, because with this methode it is theoretical possible to bypass the dopplereffect. \\
By saturation spectroscopy a laser (pump beam) beam will pass a gas sample (with doppler widened absorption transitions). That caused a stimulation of specific atoms, depending from there speed. 

Vice versa to the pump beam another laser beam with the same frequency (probe beam) will also pass the gas sample.

If now a photon is absorbed, the atom must have a speed component ($v_z$ parallel to the laser beams) in the intervall: 
\begin{align}
    v_{\text{z}} \pm \Delta v_{\text{z}} = \frac{\omega -\omega_0 \pm \delta \omega_{\text{n}}}{k}
\end{align}
In this connection: 
\begin{itemize} 
    \item $\delta \omega_{\text{n}}$ is the homogeneous line width.
    \item $\Delta v_{\text{z}}$ is the width of the hole in the occupation distribution.
    \item $\omega$ is the frequency of the laser.
    \item $\omega_0$ is the resonance frequency of the wanted transition.
    \item $k$ 
\end{itemize}

Because many atoms are stimulated, the probe beam hits barley atoms that can be stimulated. For this reason we obtain the occupation distribution $N(v_{\text{z}})$ and the adsorption profile $\alpha(\omega)$. For this experiment the laser frequency is so chosen, that the fit with the resonance frequency of the wanted transition.

\section{Preparatory questions}

\paragraph{1. Nuclear spin}
Because the isotopes have a different nuclear structure and the spin depends on that structure, the spins are different. Below the quantum numbers of the basic condition and the fist two stimulated conditions: 