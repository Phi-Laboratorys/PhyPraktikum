\chapter{Theoretical background}

\section{Saturation spectroscopy in general}
With saturation spectroscopy enables us to see the hyper fine structure, because with this method it is theoretical possible to bypass the doppler effect. \\
By saturation spectroscopy a laser (pump beam) beam will pass a gas sample (with doppler widened absorption transitions). That caused a stimulation of specific atoms, depending from there speed. 

Vice versa to the pump beam another laser beam with the same frequency (probe beam) will also pass the gas sample.

If now a photon is absorbed, the atom must have a speed component ($v_z$ parallel to the laser beams) in the interval: 
\begin{gather}
    v_{\text{z}} \pm \Delta v_{\text{z}} = \frac{\omega -\omega_0 \pm \delta \omega_{\text{n}}}{k}
\end{gather}
In this connection: 
\begin{itemize} 
    \item $\delta \omega_{\text{n}}$ is the homogeneous line width.
    \item $\Delta v_{\text{z}}$ is the width of the hole in the occupation distribution.
    \item $\omega$ is the frequency of the laser.
    \item $\omega_0$ is the resonance frequency of the wanted transition.
    %\item $k$ 
\end{itemize}

Because many atoms are stimulated, the probe beam hits barley atoms that can be stimulated. For this reason we obtain the occupation distribution $N(v_{\text{z}})$ and the adsorption profile $\alpha(\omega)$. For this experiment the laser frequency is so chosen, that the fit with the resonance frequency of the wanted transition \citep{VA00}.

\newpage
\section{Preparatory questions}

\paragraph{1. Nuclear spin}
Because the isotopes have a different nuclear structure and the spin depends on that structure, the spins are different. Below the quantum numbers of the basic condition and the fist two stimulated conditions: 
\begin{table}[h]
    \centering
    \begin{tabular}{|c|c|c|c|c|}
        \hline
        isotope & basic condition & first stimulated condition & second stimulated condition \\
        \hline\hline
        Rb-85 & $5^2$S$_{\frac{1}{2}}$ $\Rightarrow$ F = 2, 3 & $5^2$P$_{\frac{1}{2}}$ $\Rightarrow$ F = 2, 3 & $5^2$S$_{\frac{3}{2}}$ $\Rightarrow$ F = 1, 2, 3, 4 \\
        \hline
        Rb-87 & $5^2$S$_{\frac{1}{2}}$ $\Rightarrow$ F = 1, 2 & $5^2$P$_{\frac{1}{2}}$ $\Rightarrow$ F = 1, 2 & $5^2$S$_{\frac{3}{2}}$ $\Rightarrow$ F = 0, 1, 2, 3 \\
        \hline
    \end{tabular}
    \caption{Quantum numbers of Rb-85 and Rb-87}
\end{table}
\paragraph{2. Important terms}

\subparagraph{a)}\textbf{Natural line width}\\
The natural line width is only the result of the finite beam duration. So it is the smallest line width you can measure \citep{Dem}.

\subparagraph{b)}\textbf{Doppler broadening}\\
If a stimulated atom moves with a speed component $v_z$ the light who is emitted by this atom and has a wave vector component $k_z$ is subject to the doppler effect. Because the atom movement is statistical and depends on the temperature of the probe the doppler broadening also depends on the temperature \citep{Dem}.
\begin{gather}
    \Delta \nu_D = \frac{2\nu_0}{c}\sqrt{\ln(2)\frac{2k_BT}{m}} = \frac{2\nu_0\hat{v}}{c}\sqrt{\ln(2)}~\text{with}~\hat{v}= \sqrt{\frac{2k_BT}{m}}, \\
    \Delta\nu_D = 2\sigma\sqrt{\ln(2)} \Rightarrow \sigma = \frac{2\nu_0\hat{v}}{c} \Leftrightarrow \hat{v} = \frac{\sigma c}{2 \nu_0}
\end{gather}

\subparagraph{c)}\textbf{Homogeneous/inhomogeneous widening}\\
Is the probability for emission and absorption equal, so is the line homogeneously widened. If not, it is inhomogeneously broadened \citep{LSDem}.

\subparagraph{d)}\textbf{Saturation widening}\\
The line profile will be changed caused by a change of the occupation number of the transition. The pump beam causes a saturation of the occupation density and this leads to a line widening \citep{LSDem}.

\paragraph{3. Cross-over resonances}
Cross-over resonances are additionally occurring resonance line between tow lamp dips. They are caused by an overlap of the doppler broadened signals. The following applies to the frequency: 
\begin{gather}
    f_c = \frac{f_1 + f_2}{2}
\end{gather}

\paragraph{4. Beam splitter and $\lambda$/2 plate}
The $\lambda$/2 plate and filter wheel in connection with the first beam polarizing splitter is for the adjustment of the intensity distribution between the Pump beam and the probe beam. The second polarizing beam splitter is used to decouple the probe beam and couple the pump beam.

\paragraph{5. Hyperfine structure}
The Hyperfine constant can be calculated with the following formular: 
\begin{gather}
    \Delta E_{HFS} = \frac{a}{2} [F(F+1) - J(J+1) - I(I+1)]
\end{gather}
In this con $a$ is the Hyperfine constant.
\begin{gather}
    a = \frac{g_I \mu_k}{\sqrt{J(J+1)}} \\
    \mu_k = \frac{eh}{2m_p} = 5.0510 \cdot 10^{-27}
\end{gather}
$g_I$ is the Lande-factor.\\
The possible transitions are determined by the selection rules of the hyperfine structure \citep{Dem}: 
\begin{itemize}
    \item $\Delta F = 0, \pm 1$
    \item $\Delta J = 0, \pm 1$
    \item $\Delta L = \pm 1$
\end{itemize}