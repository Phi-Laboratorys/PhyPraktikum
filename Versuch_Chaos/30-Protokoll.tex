% Manuel Lippert - Paul Schwanitz
% Physikalisches Praktikum

% 3.Kapitel  Protokoll

\chapter{Messprotokoll}
\label{chap:protokoll}


\section{Versuchsdurchführung invertiertes Pendel}
\label{sec:versuchPendel}
\subsection*{Bifurkationsdiagramm}
\label{sub:bifu}
Vermessung der Gleichgewichtslage $\theta_g$ in Abhängigkeit der Masse $M$. Dabei wird die Gleichgewichtslage über eine Spannung $U_a$, welche generiert wird durch in (\ref{sub:aufbauPendel}) beschriebenen Schaltkreis erzeugt wird. Dabei wird angenommen (siehe (\ref{sub:aufbauPendel})), dass $U_a$ proportional zu der Auslenkwinkel im Gleichgewicht $\theta_g$ ist und damit der generelle Verlauf der Graphen identisch ist.
\begin{itemize}
\item Messfehler Waage: $s_a = 0,005\, g = s_r$
\item Messfehler Multimeter: $s_a = 0,00005 = s_r$\\
      Durch starke Schwankungen am Multimeter verändert sich der Wert des Fehlers des Multimeters mit der Zunahme der Masse $M$
\item Länge des Pendels (mit Stahlmaßstab): $l = 37 \, cm$
\item Gewicht Feststellschraube: $m_s = 3,14 \,g$
\end{itemize}
Datei: BifurkationPendel.csv

\textbf{Verifikation Ergebnis:}\\
Freie Schwingung des Pendels bei einer Auslenkung bis ca. 1V. Messung der Schwingunsdauer $T$ (10fach) mit Smartphone (Google Pixel 5). Über die Schwingungsdauer wird dann die Federkonstante $k$ des Pendels berechnet.

Datei: Schwingunsdauer\_woMass.csv; Messung ohne Masse.\\
Datei: Schwingunsdauer\_wMass.csv; Messung mit 12,58 g.

Grobe Auswertung: Kritische Masse $M_k\approx19,3$ g wurde durch Überschlagsrechnung bestätigt.

\subsection*{Schwache Nichtlinearität}
\label{sub:weakLin}
Montage Dämpfungssegel, wobei dabei zu beachten ist, dass die kritsche Masse $M_k\approx19,3$ g nicht überschritten wird, damit man im monostabilen Zustand des Pendels bleibt.
\begin{itemize}
    \item Masse Dämfungssegel: $m=4,4$ g
    \item Zusätzlich montierte Masse: $m = 10,42$ g
\end{itemize}
Die Masse $M_{total}= 14,41$ g bleibt hierbei über den ganzen Versuchsteil unverändert und das Dämpfungssegel befestigt über den kompletten restlichen Versuchsverlauf.
\paragraph{a)}
Für die Amplitudenabhängigkeit des Pendels wird dieses einmal ausgelenkt und dessen Schwingung über das Messprogramm aufgezeichnet.

Datei: 06\_09\_2021\_14\_41\_30\_G11\_pendel\_0.dat

\paragraph{b)}
Messung der Hystereseschleife der Schwingung mit Messprogramm.

Einstellungen: 2000 Steps; Start: 0Hz; End: 1,1Hz\\
Datei: 06\_09\_2021\_14\_30\_52\_06\_09\_2021\_14\_30\_52\_G11\_pendel\_resonanz\_\_0.dat\\

\subsection*{Starke Nichtlinearität}
\label{sub:strongLin}
Veränderung der Masse über die kritische Masse für bistabilen Zustand. Beachte Dämpfungsblech aus (2a) immer noch mit befestigt.
\begin{itemize}
    \item Zusätzlich montierte Masse: $m = 19,44$ g
\end{itemize}
Diese Masse $M_{total}=23,84$ g bleibt auch über diesen Versuchsteil unverändert.

\paragraph{a)}
Variation der Antriebsfrequenz $\omega$ in kleinen Schritten zur Lokalisierung der Schwingungszustände. Aufnahme der Attraktoren und Leistungsspektren mit dem Messprogramm.

Labview Absturz bei 0,517 Hz 

Datei: Datum\_Uhrzeit\_G11\_pendel\_0.dat (Mehrere Files mit selben Namen in zugehörige Ordner gespeichert)

Neustart bei 0,52 Hz
quasichaotisch bei $\omega\approx0,411$ Hz

Datei: Datum\_Uhrzeit\_Richter\_pendel\_0.dat (Wegen Neustart Dateienbennung verändert gewesen)

Frequenzen den einzelnen Schwingungszuständen werden den Daten entnommen.

\paragraph{b)}
Verdeutlichung der Empfindlichkeit der Anfangsbedingung mit Ruhelage auf der linken Seite (Pendel hängt auf die linke Seite). Aufnahme von drei Zeitserien Trajektorien (eine mehr als benötigt) mittels des Messprogramms.
\begin{itemize}
    \item Anregungsfrequenz $\omega$: 0,411Hz
\end{itemize}
Strittmotor in Anfangspos. (Armstelleung ganz unten)

Datei: Datei: Datum\_Uhrzeit\_G11\_pendel\_0.dat (Mehrere Files mit selben Namen in zugehörige Ordner gespeichert)

% TODO: #32 Protokoll Shinriki überarbeiten @PaulSchwanitz
\section{Versuchsdurchführung Shinriki}
\label{sec:versuchShin}
\textbf{a) Phasendiagramm}\\
Wir stellen einzelne Paramterwerte für $R_2$ und $R_1$ ein und varieren je nach eingestellten Parameter mit $R_2$ oder $R_1$ bis das Phasendiagramm vollständig abgefahren ist ab. Widerstände schwer ablesbar, weswegen Angaben fehlerhaft sein können, $R_1$ konnte hierbei genauer bestimmt werden. Alle Werte werden in $\text{k}\Omega$ angeben.\\
\begin{tabular}{c  r| c  r  r  r  r  r  r  r}
    Par &  & Var & Per1 & Per2 & Per4 & Chaos1 & Per3 & Chaos2 & Double\\
    \hline
    $R_2$ & 16,50 & $R_1$ & 13,40 & 20,10 & 21,10 & 21,46 & 22,20 & 22,30 & 24,62\\
    $R_2$ & 13,00 & $R_1$ & 16,16 & 23,68 & 25,19 & 25,86 & 26,72 & 26,82 & 28,98\\
    $R_1$ & 55,00 & $R_2$ & 7,88  & 8,52 & 8,72 & 8,76 & 8,92 & 8,96 & 9,16\\
\end{tabular}\\

\textbf{b) Schnitt im Phasendiagramm}\\
Wir schneiden bei fixierten $R_2=8,4~\text{k}\Omega$ durch das Phasendiagramm.
Daten werden elektronisch erstellt.

\textbf{c) Bifurkationsdiagramm}\\
Nutzen oben verwendeten Schnitt für diese Aufgabe zum Erstellen eines Bifurkationsdiagramms. Dies geschieht wieder elektronisch.

\textbf{d) Großmann-Feigenbaum-Konstante}\\
Wir vermessen nun gesondert die einzelnen Bifurkationen durch Variation von $R_1$ mit gleichen $R_2$ aus (b). Alle Werte werden $\text{k}\Omega$ angegeben. Hierbei steht der Index $i$ in $r_i$ für die einzeln Bifurkationen.\\
\begin{tabular}{c c c}
    $r_1$ & $r_2$ & $r_3$\\
    \hline
    59 & 65,6 & 67,6
\end{tabular}

\textbf{e) Einbettungstheorem}\\
Aufnahme des Originalattraktors bei $R_2$ wie in (b) und $R_1=51~\text{k}\Omega$ und $\delta t=60$ n (?); Rekonstruktion mithilfe des entsprechenden Programmteils mit qualitativen Übereinstimmung der Form des rekonstruierten Attraktors mit dem Originalattraktor.


% Einbindung des Protokolls als pdf (mit Seitenzahl etc.)
% Erste Seite mit Überschrift
%\includepdf[pages = 1, landscape = false, nup = 1x1, scale = \skalierung , pagecommand={\thispagestyle{empty}\chapter{Protokoll}}]
%            {03-Protokoll/Protokoll.pdf}
% Restliche Seiten richtig skaliert
%\includepdf[pages = -, landscape = false, nup = 1x1, scale = \skalierung , pagecommand={}]
%            {03-Protokoll/Protokoll.pdf}