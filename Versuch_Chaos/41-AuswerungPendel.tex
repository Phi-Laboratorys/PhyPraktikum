% Manuel Lippert - Paul Schwanitz
% Physikalisches Praktikum

% Auswertung Teil 1

\section{invertierteres Pendel}
\label{sec:auswertungPendel}
\subsection{Bifurkationsdiagramm und kritische Masse}
\label{sub:bifuAndKritMass}
Über den Versuch wurde die Auslenkung $\theta$ bzw. die Spannung $U_a$ am Digitalmultimeter links ($U_{a,l}$) und rechts ($U_{a,r}$) gemessen. Dabei ergeben sich erst ab einer gewissen Masse, eine Auslenkung links oder rechts hier schematisch in dem Bifurkationsdiagramm dargestellt (Abb \ref{image:bifu}), welches auch symmetrisch gegenüber Periode 1 ist.
\begin{center}
    \includegraphics[scale=0.7]{Pendel/3.1/BifurkationMass.png}
    \captionof{figure}{Bifurkationsdiagramm in Abhängigkeit der Masse}
    \label{image:bifu}
\end{center}
Um die kritische Masse $M_k$ zu bestimmen, wird die Differenz $\Delta U_a=U_{a,l}-U_{a,r}$ bestimmt und diese quadriert, also $(\Delta U_a)^2$. Der Fehler ergibt sich dann aus dem Fehlerfortpflanzungsgesetz, wobei der Ablesefehler $s_a$ gleichzeitig als Restfehler $s_r$ abgeschätzt wird. Daraus folgt:
\begin{gather}
    (\Delta U_a)^2 = (U_{a,l}-U_{a,r})^2\\
    s_{U_a}=\sqrt{s_a^2+s_r^2}=\sqrt{2}s_a\\[0,5cm]
    s_{(\Delta U_a)^2}=\sqrt{\left(\frac{\partial((\Delta U_a)^2)}{\partial U_{a,l}}s_{U_a}\right)^2 + \left(\frac{\partial((\Delta U_a)^2)}{\partial U_{a,r}}s_{U_a}\right)^2}=2\sqrt{2}s_{U_a}|\Delta U_a|=4s_a |\Delta U_a|
\end{gather}
\begin{center}
    \begin{tabular}{r|cccc|cc}
        $M$/g &  $U_{a,l}$/V &  $U_{a,r}$/V & $s_a$/V & s_{U_a}/V & $(\Delta U_a)^2$/V$^2$ &  $s_{(\Delta U_a)^2}$/V$^2$ \\
        \hline
         0,00  &  1,78606 &  1,78606 &   0,00005 &  0,00007 &  0,0 &     0,0 \\
         3,14  &  1,78046 &  1,78046 &   0,00050 &  0,00071 &  0,0 &     0,0 \\
         6,78  &  1,77300 &  1,77300 &   0,00500 &  0,00707 &  0,0 &     0,0 \\
        10,75  &  1,76000 &  1,76000 &   0,00500 &  0,00707 &  0,0 &     0,0 \\
        14,42  &  1,72000 &  1,72000 &   0,05000 &  0,07071 &  0,0 &     0,0 \\
        15,72  &  1,70000 &  1,70000 &   0,05000 &  0,07071 &  0,0 &     0,0 \\
        18,10  &  1,57000 &  1,57000 &   0,05000 &  0,07071 &  0,0 &     0,0 \\
        19,36  &  1,33000 &  1,33000 &   0,05000 &  0,07071 &  0,0 &     0,0 \\
        21,19  &  1,10000 &  2,38000 &   0,05000 &  0,07071 &  1,6 &     0,3 \\
        21,67  &  1,10000 &  2,35000 &   0,05000 &  0,07071 &  1,6 &     0,3 \\
        23,05  &  0,86000 &  2,65000 &   0,05000 &  0,07071 &  3,2 &     0,4 \\
        24,88  &  0,71000 &  2,83000 &   0,05000 &  0,07071 &  4,5 &     0,4 \\
        28,45  &  0,45000 &  3,11000 &   0,05000 &  0,07071 &  7,1 &     0,5 \\
        32,12  &  0,23000 &  3,36000 &   0,05000 &  0,07071 &  9,8 &     0,6 \\
    \end{tabular}
    \captionof{table}{Messreihe Auslenkung Gleichgewichtslage}
    \label{tab:gleichgewichtslage}
\end{center}
Die Daten werden dann mit dem Numpy-Modul linear gefittet, dabei werden nur die letzten sieben Datensätze verwendet werden, da sich $(\Delta U_a)^2$ erst ab da Veränderung zeigt (siehe Abb. \ref{image:linFit}). Dabei ergibt sich die gefittete Funktion mit den jeweiligen Fehlern ermittelt:
\begin{gather}
    (\Delta U_a)^2 = c M + b = 0,77 \frac{\text{V}^2}{\text{g}} M - 14,73~\text{V}^2\\
    c = (0,77 \pm 0,02)~\frac{\text{V}^2}{\text{g}},~b = (-14,73 \pm 0,48)~\text{V}^2
\end{gather}
Daraus ergibt sich die kritische Masse $M_k$, wenn man $(\Delta U_a)^2=0$ setzt. Woraus widerrum mit Fehlerfortpflanzung folgt:
\begin{gather}
    M_k = \frac{b}{c} = 19,22~\text {g},~
    s_{M_k} = \sqrt{\left(\frac{s_b}{c}\right)^2+\left(\frac{b s_c}{c^2}\right)^2}=0,78~\text{g}\\[0,5cm]
    \Rightarrow\boxed{M_k = (19,22 \pm 0,78)~\text{g}}
\end{gather}
Die Federkonstante $k$ und dessen Fehler ergibt sich dann mit dem Fehler der Länge $L$ (gemessen mit Stahlmaßstab), wobei die Erdbeschleunigung fehlerfrei angenommen wird:
\begin{gather}
    L = 0,37~\text{m}\\
    s_L=\sqrt{s_a^2+s_r^2}=\sqrt{(5\cdot 10^{-4}~\text{m})^2+(5\cdot 10^{-5}~\text{m}+\cdot 10^{-4}*L)^2} = 0,0006~\text{m}\\
    M_k = \frac{k}{gL} \Leftrightarrow k = M_k g L = 0,069762834~\text{Nm}\\
    s_k = \sqrt{(gLs_{M_k})^2+(M_kgs_L)^2} = 0,002833425~\text{Nm}\\[0,5cm]
    \Rightarrow\boxed{k = (0,070 \pm 0,003)~\text{Nm}}
\end{gather}
Hierbei ist anzumerken, dass $k$ nicht die Einheit einer Federkonstante hat sondern eines Drehmoments. Die Vermutung liegt mit einen Blick auf Kapitel \ref{sub:bewegungsgleichung} beschrieben Differentialgleichung, dass sich bei $k$ eigentlich um das Direktionsmoment handeln muss, wobei das Direktionsmoment der Federkonstante $k$ bei longitudinalen Auslenkungen entspricht.

Zur Verifikation des Ergebnisses haben wir mehrfache (10mal) die Schwingungsdauer des Pendels mit der befestigten Masse $M = 12,58$ g gemessen.
\begin{center}
    \begin{tabular}{ c | cccccccccc }
        {} & 1 & 2 & 3 &4 &5 &6 &7 &8 &9 &10\\
        \hline
        $T_10$/s&19,14&19,28&18,88&19,18&19,00&18,66&19,12&19,01&19,20&19,29\\
    \end{tabular}
    \captionof{table}{Messreihe Schwingungsdauer}
    \label{tab:schwingung}
\end{center}
Es wird nun der Mittelwert über eine Periode genommen, wobei der Ablesefehler der digitale Messuhr auch gleichzeitig als Restfehler abgeschätzt wird und der Fehler der einen Periode mit dem Fehlerfortpflanzungsgesetz berechnet wurde. Daraus folgt:
\begin{gather}
    T = \overline{T} = \frac{1}{10} \sum_{n = 1}^{10} \frac{T_{10,n}}{10} = 1,9076~\text{s}\\
    s_{T_{10}} = \sqrt{s_a^2+s_r^2}=\sqrt{2}s_a \Rightarrow s_T = \frac{s_T}{10\sqrt{10}} = \frac{s_a}{10\sqrt{5}} = 0,000447214\\[0,5cm]
    \Rightarrow\boxed{T = (1,9076 \pm 0,0004)~\text{s}}
\end{gather}
\newpage
Über Federkonstante (Direktionsmoment) $k$ lässt sich dann die Schwingungsdauer $T$ wie folgt berechnen \citep{Leifi}, wobei für den Fehler von $T$ der Fehler der Masse und Länge vernachlässigt wird:
\begin{gather}
    T = 2\pi\sqrt{\frac{ML^2}{k- MgL}} = 1,671383586~\text{s}\\
    s_T = \frac{\pi ML^2 s_k}{(k-MgL)^2 \sqrt{\frac{ML^2}{k- MgL}}} = 0,1030091566\\[0,5cm]
    \Rightarrow\boxed{T = (1,67 \pm 0,10)~\text{s}}
\end{gather}
Das Ergebnis zeigt, dass die bestimmte Federkonstante $k$ nahe an den tatsächlichen Wert der Blattfeder liegt. Die möglichen Abweichung könnten von denen in Kapitel \ref{sub:bewegungsgleichung} gemachten Näherungen oder dem vorangegangenen Fit verursacht werden. Auch könnte das Alter des Messaufbaus seinen Teil zu der Ungenauigkeit beigetragen haben. Dennoch ist das Ergebnis im Rahmen unserer Möglichkeiten akzeptabel.
\newpage
\begin{center}
    \includegraphics[scale=0.6]{Pendel/3.1/linearFit.png}
    \captionof{figure}{Linerar Fit der Messreihe}
    \label{image:linFit}
\end{center}