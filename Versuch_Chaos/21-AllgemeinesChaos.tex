% Manuel Lippert - Paul Schwanitz
% Physikalisches Praktikum

% Teilaufgabe 1

\section{Allgemeines zum Thema Chaos}
\label{sec:allgemeines}

\subsection{Dynamische Systeme}
\label{sub:dynamSys}
Ein \textit{\textbf{dynamisches System}} ist mathematisches Modell eines zeitabhängigen Prozesses, dessen Verlauf nur vom Anfangszustand abhängt.\footnote{\url{https://de.wikipedia.org/wiki/Dynamisches_System}}\\
Die Formulierung dieses Sachverhaltes in der Physik geschieht anhand von Differentialgleichungen mit dem Vektor $\vect{x}(t)=(x_1(t)$,...$x_n(t))\in\mathbb{R}^n$
\begin{gather}
    \dot{\vect{x}}(t)=\vect{F}(\vect{x}(t)),
    \label{eq:dynamDGL}
\end{gather}
dabei beschreibt $\vect{x}(t)$ den \textit{\textbf{Zustand}} des Systems zum Zeitpunkt $t\in\mathbb{R}$.\\
%Hierzu ist zu erwähnen, dass alle folgendenen Eigenschaften von (\ref{eq:dynamDGL}) für (\ref{eq:dynamDis}) gleichermaßen gilt und ab hier nur noch auf (\ref{eq:dynamDGL}) eingegangen.\\
Das dynamische System ist vollständig determiniert, wenn ein Zustand $\vect{x}(t)$ angegeben ist. Aus diesem Zustand lassen sich alle vorangegangen und folgenden Zustände des Systems bestimmen.
%Bei der Angabe einer speziellen Lösung für (\ref{eq:dynamDGL}) wird ein Anfangszustand $\vect{x_0}\in{U}$ zu einem Zeitpunkt $t$ eindeutig einem Zustand $\vect{x}(\vect{x_0},t)$ zugeordnet.
%bzw. mit einem Satz von zeitdiskreten Differentialgleichungen
%\begin{gather}
%    \vect{x}(t+1)=\vect{f}(\vect{x}(t)),
%    \label{eq:dynamDis}
%\end{gather}

\subsection{Begriffe des Phasenraums}
\label{sub:phasenraum}
\begin{itemize}
    \item[\textbf{1.}]\textbf{Phasenfluss}\\
    In der Mathematik wird ein dynamisches System durch den \textit{\textbf{Fluss}} bzw. \textit{\textbf{Phasenfluss}}  beschrieben. Unter dem \textit{Fluss} versteht man die Abbildung $\phi:\mathbb{R}^n\times\mathbb{R}\rightarrow\mathbb{R}^n$, welche die \textit{\textbf{Flussaxiome}} erfüllt:
    \begin{gather}
        \begin{aligned}
            (1)~&\phi(\vect{x},0)=\vect{x}\\
            (2)~&\phi(\phi(\vect{x},t),s)=\phi(\vect{x},t+s).
            \label{eq:flussaxiome}
        \end{aligned}
    \end{gather}
    Dieser ordnet \textbf{jedem} Anfangszustand einen neuen Zustand zur Zeit $t$ zu. Im Allgemeinen ist es nicht möglich den \textit{Fluss} $\phi$ explizit zu berechnen, dennoch sind gewisse Aussagen über das System möglich.
    \item[\textbf{2.}]\textbf{Trajektorie}\\
    Der \textit{Fluss} $\phi$ kann mit dem \textit{Zustand} $\vect{x}(t)$ in Verbindung gebracht werden mit der Beziehung: $\vect{x}(t)=\phi_{\vect{x}_0}(t)=\phi(\vect{x}_0,t)$ mit dem Anfangszustand $\vect{x}_0$, wobei nach  (\ref{eq:flussaxiome}) $\vect{x}(0)=\phi_{\vect{x}_0}(0)=\phi(\vect{x}_0,0)=\vect{x}_0$ gilt.\\
    Hierbei beschreibt $\phi_{\vect{x}_0}(t)$ oder $\vect{x}(t)$ die \textit{Lösungskurve}, welche auch \textit{Bahnkurve}, \textit{Orbit}, \textit{Phasenbahn} oder \textit{\textbf{Trajektorie}} genannt wird und eine spezielle Lösung von (\ref{eq:dynamDGL}) darstellt, welche wiederum die Bewegung des Punktes $\vect{x}$ unter Wirkung des Flusses $\phi$ beschreibt.
    \item[\textbf{3.}]\textbf{Phasenraum}\\
    Der \textit{\textbf{Phasenraum}} beschreibt eine Menge aller Zustände eines dynamischen Systems. Dieser wird von Zustand $\vect{x}(t)$ auf der \enquote{x-Achse} und dessen Ableitung $\dot{\vect{x}}(t)$ auf der \enquote{y-Achse} aufgespannt, was eine Parameterdarstellung einer Differentialgleichung über die Zeit $t$ darstellt.\\
    In diesem \textit{Phasenraum} lässt sich dann ein Vektor $(\vect{x},\dot{\vect{x}})$ definieren, welcher auf die \textit{\textbf{Phasenraumtajektorie}} zeigt. Die \textit{Phasenraumtrajektorie} stellt den Bewegungsvorgang des Systems dar.
    \item[\textbf{4.}]\textbf{Attraktor}\\
     
\end{itemize}
% TODO: #18 FzV 2.1.2 @ManeLippert

\subsection{Deterministisches Chaos}
\label{sub:determChaos}
% TODO: #17 FzV 2.1.3 @PaulSchwanitz

\subsection{Fouriertransformation und Leistungsspektrum}
\label{sub:fouriertrafo}
% TODO: #19 FzV 2.1.4 @PaulSchwanitz

\subsection{Darstellungsweisen eines chaotischen Attraktor}
\label{sub:darstellungAttraktor}
% TODO: #20 FzV 2.1.5 @ManeLippert