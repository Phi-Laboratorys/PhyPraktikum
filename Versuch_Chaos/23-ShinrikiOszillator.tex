% Manuel Lippert - Paul Schwanitz
% Physikalisches Praktikum

% Teilaufgabe 3
% TODO: #22 FzV 2.3 @PaulSchwanitz

\section{Der Shinriki-Oszillator}
\label{sec:shinrikiOszi}

\subsection{Differentialgleichung des Shinriki-Oszillator}
\label{sub:dgl}

Der Shinriki-Oszillator besteht aus einem negativen Impedanzkonverter (NIC) und einem LC-Parallelschwinkreis, die durch ein gegeneinader geschaltetes Zenerdiodenpaar und dem parallel geschalteten \(R_2\), gekoppelt sind. \\
Die Leitwertfunktion des Kopplungsglied ist \( f(V)\) und beschreibt den Strom, der über das Kopplungsglied fließt.
\(R_{NIC}\) ist der Widerstand des NIC innerhalb des relevanten Intervalls von -8,1 V bis 8,1 V \citep[]{Lueck}.\\
Damit und mit den Kirchhoffschen Regeln lassen sich nun die DGLs aufstellen:
\begin{align}
    C_1 \dot{V_1} &= V_1 (\frac{1}{R_{NIC}}-\frac{1}{R_1}) - f(V_1-V_2) \\
    C_2 \dot{V_2} &= f(V_1-V_2) - I_3 \\
    L \dot{I_3} &= -I_3R_3 + V_2
\end{align}

\subsection{NIC und Schwingung des Shinriki-Schaltkreis}
\label{sub:nic}

\subsection{Geräusche einer Bifurkation}
\label{sub:tonBifurkation}