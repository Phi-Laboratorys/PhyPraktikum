% 3. Protokoll

% Variables
\def\skalierung{0.65}

\chapter{Messprotokoll}
\label{chap:protokoll}
\textbf{Messgeräte und Proben:}
\begin{itemize}
    \item Konfokalmikroskop: Leica TCS SP5 mit HCX PLAPO lambda blue 63.0 x 1.4 OIL UV Objektiv (gesteuert durch Computer-Messprogramm)
    \item Argon-Ionen-Laser (Anregung bei Wellenlängen von \SI{458}{\nano\metre} und \SI{514}{\nano\metre})
    \item Detektor: Photomultiplier (PMT)
    \item Lebenszeitmessung: PicoHarp 300 TCSPC Module und Picosecond Event Timer
    \item Proben: CFP, YFP und CY markierte Proben mit Eigenschaften aus Kapitel \ref{sec:proben}
\end{itemize}

\section*{Bildaufnahme mit Konfokalmikroskop}
\label{sec:bildaufnahme}
\begin{itemize}
    \item \SI{400}{\hertz} ist die optimale Speed-Einstellung 
    \item Minimale Aufnahmedauer hängt von Speed ab
    \item Je länger die Aufnahmedauer, desto stärker wird die Zelle bestrahlt und somit gegebenenfalls zerstört oder gebleicht
    \item Zoom hängt mit eingestelltem Speed zusammen
\end{itemize}

\section*{Aufnahmen der Sensitized Emission}
\label{sec:aufnahme}
16 Aufnahmen von  $D_{CY}$, $A_{CY}$, $S_{CY}$ und Transmissionsbild mittels Messprogramm.\\

\textbf{Einstellungen:}
\begin{itemize}
    \item Donor-Anregung: \SI{458}{\nano\metre} mit 30\% Leistung
    \item FRET-Anregung: \SI{514}{\nano\metre} mit 4\% Leistung 
    \item Line-Average = 4
    \item Speed = \SI{400}{\hertz}
\end{itemize}
\newpage
\textbf{Modus:}
\begin{itemize}
    \item[(1)] $D_{CY}$ + $S_{CY}$:\\
    PMT1 (Donor):\\
    \SI{470}{\nano\metre} bis \SI{500}{\nano\metre}, Gain = \SI{1000}{\volt}, Offset = 0\%, Auswahl = None\\
    PMT3 (Akzeptor):\\
    \SI{520}{\nano\metre} bis \SI{550}{\nano\metre}, Gain = \SI{1000}{\volt}, Offset = 0\%, Auswahl = None
    \item[(2)] $A_{CY}$ + Transmissionsbild:\\
    PMT3 (Akzeptor):\\
     \SI{520}{\nano\metre} bis \SI{550}{\nano\metre}, Gain = \SI{1000}{\volt}, Offset = 0\%, Auswahl = None\\
    PMT Trans:\\
    Gain = \SI{400}{\volt}, Offset = 0\%, Auswahl = Scan-BF\\
\end{itemize} 
Reihenfolge Aufnahme Modus (1)$\rightarrow$(2)\\
Dateiordner: \textit{CY} (Beide Farbstoffe), \textit{CFP} (Cyan), \textit{YFP} (Yellowqq

\section*{Akzeptorbleichung (Bleaching)}
10 Aufnahmen der CFP/YFP-Zelle für $D_{CY}$, $S_{CY}$ mittels Messprogramm vor nach dem Bleaching.\\

\textbf{Einstellungen:}
\begin{itemize}
    \item Akzeptor-Anregung: \SI{514}{\nano\metre} mit 100\% Leistung
    \item Line-Average = 1
    \item Set background to zero = true
\end{itemize}
\textbf{Time Course:}
\begin{itemize}
    \item Pre-bleach: Frames = 10, t/frame[s] = 1.293, minimize = true
    \item Bleach: Frames = 12
    \item Post-bleach 1: Frames = 20, t/frame[s] = 1293, minimize = true
\end{itemize}
Dateinamen: \textit{CY-N-X.csv}; N = 1,...,10; X = D (Donor,PMT1), S (Sensitized, PMT3)\\
Dateiordner: \textit{Bleach-CY} (Alle Snapshots des Experiments) Ausgewählter Snapshot: \textit{FRAP\_005Snapshot1}\\

5 Aufnahmen von jeweils CFP- und YFP-Zelle zur überprüfung der Signifikanz der Messung, dabei bleiben die eingestellten Parameter gleich.\\
Dateinamen CFP: \textit{CFP-N-X.csv}; N = 1,...,5; X = D (Donor,PMT1), S (Sensitized,PMT3)\\
Dateinamen YFP: \textit{YFP-N.csv}; N = 1,...,3; (Akzeptor,PMT3)\\

\section*{Lebenszeit}
Messung der Lebenszeit an CFP/YFP, CFP und YFP Zellen mit einmal Exponentialfit und zweimal Exponentialfit und dem Gerätefehler IRF\\

\textbf{Einstellungen:}
\begin{itemize}
    \item Laser: \SI{470}{\nano\metre} mit \SI{40}{\mega\hertz}
    \item FILM: Zeitintervall = 2,01263 min, Repetitions = 93, max = 1000, Frames = 1  
\end{itemize}

\textbf{Modus}
\begin{itemize}
    \item PMT APD1:\\
    Gain = 10, Auswahl = Scan-BF
    \item PMT APD2:\\
    Gain = 10, Auswahl = Scan-BF
\end{itemize}

Dateiordner: \textit{TCPSP-data}\\
Dateinamen CY: \textit{CYN-chK.csv}; N = 1,...,6; K = 1, 2, 12 (Channals)\\
Dateinamen CFP,YFP: \textit{ZN-chK.csv}; Z = CFP, YFP; N = 1, 2, 3;  K = 1, 2, 12 (Channals)