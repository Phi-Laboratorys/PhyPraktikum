% Dokumentenklasse und Packages
% Die folgenden Zeilen sollen mφglichst nicht verδndert werden
\documentclass[paper=a4,bibliography=totoc,BCOR=10mm,twoside,numbers=noenddot,fontsize=11pt]{scrreprt}
\usepackage[ngerman]{babel}
\usepackage[T1]{fontenc}
\usepackage[latin1, utf8]{inputenc}
\usepackage{lmodern}
\usepackage{graphicx}
\usepackage{nicefrac}
\usepackage{fancyvrb}
\usepackage{amsmath,amssymb,amstext}
\usepackage{siunitx}
\usepackage{url}
\usepackage{natbib}
\usepackage{microtype}
\usepackage[format=plain]{caption}

% Packages von meinem Cover
%\usepackage{geometry} \geometry{top=30mm, bottom=20mm, inner=20mm, outer=20mm}
\usepackage{anyfontsize}
\usepackage{xcolor}
\usepackage{ifthen}
\usepackage[scaled]{uarial}
\usepackage[absolute,overlay]{textpos}
\usepackage{amsfonts}
\usepackage{xstring}
\usepackage{tikz}


% Abschnittseinrόckung und -abstand
% Die folgenden Zeilen sollen mφglichst nicht verδndert werden
\parindent 0.0cm
\parskip 0.8ex plus 0.5ex minus 0.5ex

% Anzahl und Grφίe von Gleitobjekten
% maximal 2 Objekte oben und unten
% erlaubt auch grφίere Bilder, welche die ganze Seite benφtigen
% Die folgenden Zeilen sollen möglichst nicht verändert werden
\setcounter{bottomnumber}{2}
\setcounter{topnumber}{2}
\renewcommand{\bottomfraction}{1.}
\renewcommand{\topfraction}{1.}
\renewcommand{\textfraction}{0.}

%\sc und \bc veraltet. Daher: (20.09.2018)
\DeclareOldFontCommand{\sc}{\normalfont\scshape}{\@nomath\sc}
\DeclareOldFontCommand{\bf}{\normalfont\scshape}{\textbf}

% Verschiedenes
\pagestyle{headings}          % Der Seitenstil sollte mφglichst nicht verδndert werden
\graphicspath{{./bilder/}}    % Der Pfad fόr die Abbildungen Abbildungen wird gesetzt
\VerbatimFootnotes            % \verb etc. auch in \footnotes mφglich
%

\begin{document}
    \nonfrenchspacing

    % 0. Kapitel Cover
    \documentclass[12pt, a4paper]{report}

%Package
\usepackage{geometry} \geometry{top=30mm, bottom=20mm, inner=20mm, outer=20mm}
\usepackage{graphicx}
\usepackage{anyfontsize}
\usepackage{xcolor}
\usepackage{ifthen}
\usepackage[utf8]{inputenc}
\usepackage[T1]{fontenc}
\usepackage{ngerman}
\usepackage[scaled]{uarial}
\usepackage[absolute,overlay]{textpos}
\usepackage{amsfonts}
\usepackage{xstring}

%Colors
\definecolor{Notablue}{HTML}{3498DB}		%Theoretische Physik
\definecolor{Notared}{HTML}{CF366C}			%Mathematik
\definecolor{Notagreen}{HTML}{19B092}		%Experimentalphysik
\definecolor{Notaorange}{HTML}{FA9D00}		%Chemie/Wahlfach nicht physikalisch
\definecolor{Notagrey}{HTML}{969696}		%Praktikum
\definecolor{Notalavendel}{HTML}{9DBBD8}	%Wahlfächer physikalisch
\definecolor{SPKred}{HTML}{E60005}

% Boolean
\newboolean{twoRows}

%Funktions
\makeatletter
   \def\vhrulefill#1{\leavevmode\leaders\hrule\@height#1\hfill \kern\z@}
\makeatother
\newcommand*\ruleline[1]{\par\noindent\raisebox{.8ex}{\makebox[\linewidth]{\vhrulefill{\linethickness}\hspace{1ex}\raisebox{-.8ex}{#1}\hspace{1ex}\vhrulefill{\linethickness}}}}

%Variables
\def\farbe{Notagreen}

\def\grafik{Grafik/CP1}
\def\scale{1.5}	
	
\def\schriftgrosse{70}
\def\linethickness{1,5pt}

\def\fach{EPB2}
\def\name{Manuel Lippert}
\def\uberschrift{Atome} %Absatz mit \\[0,5cm]; u = Übung, k = Klausur; s = Skript, e = Ergebnis
\def\bottom{SS2021}

\begin{document}
\begin{titlepage}
			
	\centering
	{\LARGE \sffamily {\textbf{\bottom}\par}}
	\vspace{2,5cm}
    {\fontsize{40}{0}\sffamily\ruleline{\textcolor{\farbe}{\textbf{\fach}}}\par}
    \vspace{6cm}
	{\Large\sffamily \ruleline{\name}\par}
	
	
	%Auswahl für Text
	\ifthenelse{\equal{\uberschrift}{s}} {\def\titel{Skript}}	
		{\ifthenelse{\equal{\uberschrift}{k}} {\def\titel{Klausur}}
			{\ifthenelse{\equal{\uberschrift}{u}} {\def\titel{Übung}}
				{\ifthenelse{\equal{\uberschrift}{e}} {\def\titel{Klausur \\[0,5cm] Ergebnis}}
					{\def\titel{\uberschrift}}}}}
	
	\IfSubStr {\titel} {\\[0,5cm]} {\setboolean{twoRows}{true}} {\setboolean{twoRows}{false}}				

	\ifthenelse{\boolean{twoRows}}
		{
			\begin{textblock*}{21cm}(0cm,8cm) % {block width} (coords), centered		
				{\fontsize{\schriftgrosse}{0}\sffamily\textcolor{\farbe}{\textbf{\titel}}\par}
			\end{textblock*}
		}
		{
			\begin{textblock*}{21cm}(0cm,9cm) % {block width} (coords), centered		
				{\fontsize{\schriftgrosse}{0}\sffamily\textcolor{\farbe}{\textbf{\titel}}\par}
			\end{textblock*} 
		}		
	
	\ifthenelse{\equal{\farbe}{Notared}}
		{\def\logo{justLogo/UniBTNotared}}
		
	\ifthenelse{\equal{\farbe}{Notagreen}}
		{\def\logo{justLogo/UniBTNotagreen}}
		
	\ifthenelse{\equal{\farbe}{Notablue}}
		{\def\logo{justLogo/UniBTNotablue}}
			
	\ifthenelse{\equal{\farbe}{Notaorange}}
		{\def\logo{justLogo/UniBTNotaorange}}
			
	\ifthenelse{\equal{\farbe}{Notagrey}}
		{\def\logo{justLogo/UniBTNotagrey}}
		
	\ifthenelse{\equal{\farbe}{Notalavendel}}
		{\def\logo{justLogo/UniBTNotalavendel}}	
		
	\ifthenelse{\equal{\farbe}{black}}
		{\def\logo{justLogo/UniBT}}	
	
	\ifthenelse{\equal{\farbe}{SPKred}}
		{\def\logo{justLogo/Sparkasse}}	

	
	%Logo
	\vfill
	\begin{figure}[h]
	\begin{center}
		
		\includegraphics[width=2cm]{\logo}
		
	\end{center}
	\end{figure}
	
\end{titlepage}
\end{document}


    %\thispagestyle{empty}
    %\cleardoublepage
    \tableofcontents
    %\cleardoublepage

    % 1. Kapitel Einleitung
    % Charlotte Geiger - Manuel Lippert - Leonard Schatt
% Physikalisches Praktikum

% 1. Kapitel Einleitung

\chapter{Einleitung}
\label{chap:einleitung}


Oft fragt man sich, woher wissen bestimmte Wissenschaftler Tatsachen, beipielsweise wie warm oder kalt es auf anderen 
Planeten ist und aus was ihre Atmosphäre ist. Dadurch können sie Aussagen über die Bewohnbarkeit und das vorkommen von Wasser auf dem Planeten machen. Dies verwundert so manchen 
vielleicht, da Wissenschaftler ja nicht auf jeden Planeten einen kleinen Roboter geschickt haben können - 
vorallem wenn sie außerhalb unserers Sonnensystems liegen.\\
Die einfache Antwort auf diese Frage ist Spektroskopie. Man muss nicht an einem Ort gewesen sein um Aussagen über ihn treffen zu können. Oft genügt die emittierte Strahlung,
um einen weitreichenden Einblich in die dortigen Gegebenheiten zu bekommen. Wenn man beispielsweise charakteristische Spektren aus der empfangenen Strahlung herausfiltern kann, 
ist es möglich detaillierte Aussagen über die Zusammensetzung der Materie zu machen.\\
In diesem Versuch werden wir einen etwas einfacheren Sachverhalt behandeln, um ein erstes Gefühl für die Spektroskopie zu bekommen. Wir werden das Spektrum einer Quecksilber-Dampflampe 
analysieren. Dabei wird uns wieder klar, wie vorsichtig man bei Phänomenen mit EM-Wellen sein muss, da sonst ungewollte Beugungseffekte auftreten, an die man nicht gedacht hat.
Dabei werden wir mit Hilfe eines Blazegitters das Spektrum unserer Lampe zerlegen.

    % 2.Kapitel Fragen zur Vorbereitung
    % 2.Kapitel Fragen zur Vorbereitung

\chapter{Fragen zur Vorbereitung}
\label{chap:fragen zur vorbereitung}

\section{Test ob der inport inherhalb klappt}

\end{document}