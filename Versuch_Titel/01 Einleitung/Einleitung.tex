% 1. Kapitel Einleitung

\chapter{Motivation und Ziel des Versuchs}
\label{chap:einleitung}
%
\section{\LaTeX --Distribution und Erstellen des Textes}


Dieses Dokument ist \textbf{keine} Musterauswertung und soll auch keine \LaTeX --Anleitung ersetzen. 
Es stellt lediglich eine Vorlage zum Erstellen einer Praktikumsauswertung dar und gibt eine Idee für 
eine Gliederung. Es soll Nicht--\LaTeX --Vertrauten den Einstieg erleichtern und \LaTeX --Experten von 
Formfragen befreien. Für \LaTeX --Feinheiten wird auf die entsprechende Literatur verwiesen.

Die meisten \LaTeX --Befehle werden hier nicht näher erklärt, deren Gebrauch sollte sich aber aus dem 
Quelltext ``Auswertung.tex'' und den weiteren, beschriebenen, Dateien
erschließen. Deshalb ist es als Einstieg sicher eine gute Idee, den Quelltext zusammen mit dem fertigen 
Ausdruck zu lesen, um ein Gefühl für die \LaTeX --Struktur zu bekommen. Es wird hier zwar auf grundlegende 
\LaTeX --Gestaltungsmöglichkeiten und einige Spezialitäten eingegangen, aber natürlich können nicht alle 
Eventualitäten abgedeckt werden. Gewünschte Ergänzungen oder Verbesserungsvorschläge können gerne an den 
Autor heran getragen werden.

Kapitel geht auf das grundsätzliche Arbeiten mit \LaTeX , den Aufbau und die
Gliederung des Textes sowie auf die Verwendung von Zitaten ein. Das Einfügen von Bildern und die 
Gestaltung von Tabellen wird in Kapitel behandelt. Kapitel
befasst sich mit Gleichungen und damit verwandten Problemen. Au\ss erdem werden 
verschiedene weitere Gestaltungsmöglichkeiten und einige wichtige Befehle diskutiert. Die Arbeit 
schließt mit einer Zusammenfassung und einem Fazit in Kapitel.
