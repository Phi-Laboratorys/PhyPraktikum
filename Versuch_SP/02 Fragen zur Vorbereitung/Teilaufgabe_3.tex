% Charlotte Geiger - Manuel Lippert - Leonard Schatt
% Physikalisches Praktikum

% Teilaufgabe X

\section{Teilaufgabe 3}

Wie bei jeder Gasentladungslampe beruht der leuchtproszess der Quecksilberdampflampe auf der Ionisation der Quecksilberatome. Den Quecksilber ist noch ein Edelgas beigemischt, welches
die Zündung der Lampe erleichtert. Das Leuten entsteht dabei nicht wie bei herkömmlichen Lampen durch einen glühenden Draht, sonder durch die Anregung der Quecksilberatome.
Dies geschieht durch "Stöße" der Atome mit Elektronen, welche durch das Gas geleitet werden. Die angeregten Atome emitierten bei ihrem zurückkehren in den Grundzustand elektromagnetische Wellen.
Diese sind bei Quecksilberlampen im UV-Bereich, welcher ungesund für Menschen ist.\\
Hier eine Liste der sieben hellsten Spektrallinien in Bereich 300 bis 900 nm.
\begin{itemize}
    \item 404,65 nm (violett)
    \item 407,78 nm (voilett)
    \item 435,83 nm (blau)
    \item 546,07 nm (grün)
    \item 576,95 nm (gelb-orange)
    \item 579,06 nm (gelb-orange)
    \item 614,95 nm (rot)
\end{itemize}
Außerdem exisiert noch eine Schwache Linie bei 491,60 nm.