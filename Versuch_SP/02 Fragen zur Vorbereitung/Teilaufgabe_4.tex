% Charlotte Geiger - Manuel Lippert - Leonard Schatt
% Physikalisches Praktikum

% Teilaufgabe X

\section{Teilaufgabe 4}

\begin{figure}[h]
    \begin{center}
        \includegraphics[width=10cm]{Bilder/SP1.PNG}
    \end{center}
    \caption{Versuchsaufbau}
    %\label{fig:meine-grafik}
   \end{figure}
Die Sammellinse und der Eintrittsspalt müssen nach der Abbildungsgleichung 
\begin{equation}
    \frac{1}{f}= \frac{1}{g}+\frac{1}{b}   
\end{equation}
so weit entfernt sein, dass der Brennpunkt im Spalt selbst liegt. In der Abbildungsgleichung bezeichnet $b$ den Abstand vom Bild zur Linse und $g$ den Abstand vom Gegenstand zur Linse. $f$ ist die Brennweite. Bei 
\begin{equation}
    \Rightarrow b = (\frac{1}{f}-\frac{1}{g})^{-1}
\end{equation}
ist dies der Fall. Die Sammellinse sollte also um $b$ vom Eintrittsspaltentfernt sein.\\
Der Hohlspiegel fungiert in diesem Aufbau wie eine Linse. Damit die Strahlen, welche über einen Spiegel umgeleitet werden, nach dem Spiegel parallel liegen, muss der Eintrittsspalt genau in der Brenneben liegen. Der Abstand zwischen Eintrittsspalt und Hohlspiegel muss also $f_{Spiegel}$ sein.\\
Beim zweiten Spiegel ist die Argumentation die Selbe, nur in die andere Richtung. Deswegen ist der Abstand zwischen Hohlspiegel und Austrittsspalt $f_{Spiegel}$.