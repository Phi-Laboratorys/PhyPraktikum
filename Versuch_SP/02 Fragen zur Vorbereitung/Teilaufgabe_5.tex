% Charlotte Geiger - Manuel Lippert - Leonard Schatt
% Physikalisches Praktikum

% Teilaufgabe X

\section{Teilaufgabe 5}

Das spektrale Auflösungsvermögen des Monokromators lasst sich in zweit Teile aufteilen. Der eine Teil $\Delta \lambda _{s}$, welcher von den Spalten herrührt und 
den Teil $\Delta \lambda_{G}$, welcher vom Gitter erzeugt wird.\\
$ \Delta \lambda_{S}$ weiderum lässt sich in die Teile $ \Delta \lambda_{S_{ein}}$ und $ \Delta \lambda_{S_aus}$ unterteilen, welche jeweils zu Eingangs- bzw. Ausgangsspalt gehören.
\begin{equation}
     \Delta \lambda_{S} = \sqrt{(\Delta \lambda_{S_{ein}})^{2}+(\Delta \lambda_{S_{aus}})^{2}} = \frac{b}{f} \sqrt{(s_{Ein})^{2}+(s_{Aus})^{2}} 
\end{equation}
Hier ist $b$ die Gitterkonstante, $f$ die Brennweite des Hohlspiegel und $s_{Ein/Aus}$ die Spaltbeite des Ein-/Ausgangsspaltes.\\
Dieser Linienbreite $\Delta\lambda$ ist in unserem Fall:
\begin{equation}
    \Delta \lambda_{s} =\frac{\frac{1}{1200\frac{1}{mm} } }{80mm}  \sqrt{2} * 20 \mu m = 294,6nm
\end{equation}
$\Delta \lambda _{G}$ ist in diesem Fall:
\begin{equation}
    \Delta \lambda _{G} = \frac{\lambda }{kN} = \frac{546nm}{1 \cdot 58mm \cdot1200\frac{1}{mm} } = 0.00784 nm
\end{equation}
Die beiden Werte liegen Größenordnungen außeinander. 
\begin{equation}
    \Rightarrow \Delta \lambda _{M}\simeq \Delta \lambda_{s} = 294,6nm
\end{equation}
