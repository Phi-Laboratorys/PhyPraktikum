% Charlotte Geiger - Manuel Lippert - Leonard Schatt
% Physikalisches Praktikum

% Teilauswertung X

\section{tests}
\subsection {Theoretischer Hintergrund}
In den Fragen zur Vorbereitung haben wir uns mit der Thematik der Faltung zweier Rechteckfunktionen, besch\"aftigt. Vor allem im Bezug zu unterschiedlichen Ein- und Austrittspaltbreiten haben wir die Ergebnisse interpretiert. So haben wir herausgefunden, dass es bei einer Faltung zu einer nah obenhin schmaler werdenden Trapezfunktion f\"uhrt, welche oben ein Plateau in der Mitte besitzt. Sind beide Rechteckfunktionen gleich gro\ss{}, so erh\"alt man eine Dreiecksfunktion mit optimaler Intensit\"at. Die Rechteckfunktionen entsprechen den (Anfangs- bzw. End-)Spaltbreiten. Wenn diese unterschiedlich sind, so ist das zu messende Signal kleiner und schw\"acher, als bei gleichen Spaltbreiten. Diesen Theoretischen Hintergrund kann man  durch unsere Messungen nur best\"atigen. 

\subsection {Interpretation} 
Wir betrachten im Folgenden diese Einstellungen bei unseren Messungen:\\
1. Eingangsspalt: 1,0mm \tab Ausgangsspalt: 0,5mm\\
2. Eingangsspalt: 0,5mm \tab Ausgangsspalt: 1,5mm\\
3. Eingangsspalt: 0,5mm \tab Ausgangsspalt: 1,0mm\\
4. Eingangsspalt: 0,5mm \tab Ausgangsspalt: 0,1mm\\
5. Eingangsspalt: 0,1mm \tab Ausgangsspalt: 0,5mm\\


Im Vergleich zu den vorherig gemessenen Messungen erkennt man, dass es ein deutlich ersichtliches Plateau gibt. Vor allem bei den Messungen 1. bis 4. ist dies sehr gut zu erkennen. Bei der letzten Messung ist dieses Plateau zwar auch zu erkennen, jedoch ist es nicht allzu stark ausgepr\"agt. Grund daf\"ur ist wahrscheinlich die sehr nah beieinander liegenden Spaltbreiten von 0,1mm und 0,5mm. 
Zus\"atzlich zu beachten ist die geringe Intensit\"at von ca. 1,2 Volt bei den Messungen 1. bis 3. und sogar nur 0,25 Volt bei Messung 4.  
Auch bei diesem Punkt bemerkt man, dass die Vierte Messung wieder aus dem Schema f\"allt, da die Intensit\"at hier bei knapp 2,2 Volt liegt. Verglichen mit den vorherigen Messungen bei Messaufgabe 4