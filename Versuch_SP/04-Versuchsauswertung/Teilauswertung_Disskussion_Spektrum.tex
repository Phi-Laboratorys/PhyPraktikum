\subsection{Literaturvergleich der gemessenen Werte} \label{Messlit}

Im folgenden werden die Literaturwerte \cite[S. 1474]{Lamport1995} mit den gemessenen Werten verglichen. Dabei wird der Peak im Diagramm abgelesen und dann der dazugehörenden 
exakt Messwert der x-Achse in den Daten nachgeschlagen.
1474 Seite
\begin{table}[h]
    \centering
    \begin{tabular}[h]{l|c|r}


        Literaturwerte in \r{A} & Messwerte in \r{A} & Differenz zwischen Messung und Literatur \\
       \hline
       %Nicht in der Literatur  & 3135 & sdfgsdf \\
       3650.15  & 3655 & 4.85\\
       4046.56  & 4045 & 1.56\\
       4358.34 & 4360 & 1.66 \\
       5460.75 & 5460 & 0.75 \\
       5769.6  & 5770 & 0.4 \\
       5790.7  & 5795 & 4.3 \\
       6234.4  & 6265 & 30.6 \\
       7346  & 7305 & 41.0 \\
       %Nicht in der Literatur  & 8100 & asdf \\
       %Nicht in der Literatur  & 8730 & asdf \\
\end{tabular}
    \caption{Vergleich der gemessenen Werte mit den Literaturwerten}
\end{table}