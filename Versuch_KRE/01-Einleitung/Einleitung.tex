% Charlotte Geiger - Manuel Lippert - Leonard Schatt
% Physikalisches Praktikum

% 1. Kapitel Einleitung


\chapter{Einleitung}

Ein Kreisel ist ein starrer Körper welcher um eine beliebige kräfte- und momentfreie Achse rotiert mit fester Lage im Raum. Dies wird mit einer Unterstützung der Achse in zwei Punkten durch Lager, welche i. Allg. durch Kräfte beansprucht werden, verwirklicht. \\
Kreisel werden häufig in der Technik und im Alltag bei rotierenden Bauteilen in Maschinen verwendet, wobei die Form der Kreisel dabei beliebig sein kann, es aber von Vorteil ist, wenn dieser die Form eines rotationssymmetrischen Körpers annimmt.
\newline\\
In diesem Versuch wird qualitativ das Verhalten eines luftgelagerten, symmetrischen Kreisel (ohne wesentliche Reibung) bei verschiedene Kreiselbewegungen sowie die Reaktion auf äußere Kräfte untersucht. Dabei werden besonders die Figurenachse, Drehimpulsachse und momentane Drehachse eine wichtige Rolle bei der Beobachtung spielen. Des Weiteren wird in diesem Versuch die Nutation und Präzession des Kreisels genauer betrachtet.