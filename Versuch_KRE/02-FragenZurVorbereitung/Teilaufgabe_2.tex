% Charlotte Geiger - Manuel Lippert - Leonard Schatt
% Physikalisches Praktikum

% Teilaufgabe 2

\section{Trägheitstensor $\underline{J}$}
\label{sec: Traegheitstensor}

Der Drehimpuls eines Körpers der Masse m ausgehen vom Ursprung des Koordinatensystems mit Ortsvektor $\mathbf{r}$ und Geschwindigkeitvektor
$\mathbf{v}$ ist angegeben durch die Gleichung
\begin{equation}
    \mathbf{L}=m\mathbf{r}\times\mathbf{v}
\end{equation}
Weiterhin gilt bei Rotation eines starren Körpers in einen beliebigen Koordinatensystems ein Zusammenhang zwischen Drehimpuls $\mathbf{L}$
und Winkelgeschwindigkeit $\mathbf{w}$ der Form
\begin{equation}
    \mathbf{L}=\underline{J}\vect{\omega}
\end{equation}
Betrachtet man nun einen starren Körper aus N Massepunkten der Masse $m_i$ aus der Sicht des Schwerpunkts S (Ursprung des Koordinatensystems) 
und interpretiert die Geschwindigkeit $\mathbf{v}$ als Kreuzprodukt des Ortsvektor $\mathbf{r}$ und der Winkelgeschwindigkeitsvektor $\vect{\omega}$ 
wird Gleichung (2.3) zu
\begin{gather}
    \vect{L}=\sum_{i=1}^{N} m_i\vect{r_i}\times(\vect{\omega}\times\vect{r_i})
\end{gather}
Auflösen der beiden Kreuzprodukte mit Hilfe des Graußmann-Identität und Einsetzen der Vektoren $\vect{\omega}=(\omega_x,\omega_y, \omega_z)$
und $\vect{r_i}=(x_i, y_i, z_i)$ 
\begin{gather}
    \begin{aligned}
    \vect{L}&=\sum_{i=1}^{N} m_i (\vect{r_i}^2\vect{\omega}-(\vect{r_i\omega})\vect{r_i})\\
            &=\sum_{i=1}^{N} m_i\left[(x_i^2+y_i^2+z_i^2)
              \left(\begin{array}{c} \omega_x \\ \omega_y \\ \omega_z \end{array}\right)
              - (x_i\omega_x+y_i\omega_y+z_i\omega_z)
              \left(\begin{array}{c} x_i \\ y_1 \\ z_i \end{array}\right)\right]\\
            &=\sum_{i=1}^{N} m_i
              \left(\begin{array}{ccc} y_i^2+z_i^2 & -x_iy_i & -x_iz_i \\ -y_ix_i & x_i^2+z_i^2 & -y_iz_i \\ -z_ix_i & -z_iy_i & x_1^2+y_i^2 \end{array}\right) 
              \left(\begin{array}{c} \omega_x \\ \omega_y \\ \omega_z \end{array}\right) \\
            &=\underline{J}\vect{\omega}
    \end{aligned} \\
    \Rightarrow \underline{J} = \sum_{i=1}^{N} m_i
                                \left(\begin{array}{ccc} y_i^2+z_i^2 & -x_iy_i & -x_iz_i \\ -y_ix_i & x_i^2+z_i^2 & -y_iz_i \\ -z_ix_i & -z_iy_i & x_1^2+y_i^2 \end{array}\right)
\end{gather}

Die Diagonalelemente von $J$ sind dabei die Trägheitsmomente des Körpers bei Drehungen um die Hauptachsen (siehe Kapitel \ref{sec: Traegheitsmoment}).