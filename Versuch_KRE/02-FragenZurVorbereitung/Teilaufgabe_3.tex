% Charlotte Geiger - Manuel Lippert - Leonard Schatt
% Physikalisches Praktikum

% Teilaufgabe 3

\section{Trägheitstensor $\underline{J}_{Rad}$ eines Rades}
\label{sec: Rad}

Ist ein Rad bei der Rotation um seine Symmetrieachse ausgewuchtet nimmt der Trägheitstensor $\underline{J}_{Rad}$ im Schwerpunkt des Rades die Form einer Diagonalmatrix an. Dabei sind die Diagonalelemente wie in Kapitel \ref{sec: Traegheitstensor} erwähnt die Trägheitsmomente des Rades bei Drehung um seine Hauptachsen bzw. Symmetrieachsen. \\
Ist das Rad bei der selben Bewegung nicht ausgewuchtet ist der Trägheitstensor $\underline{J}_{Rad}$ nicht immer diagonalisierbar heißt der Tensor hat nicht mehr die Form einer Diagonalmatrix. Diese Formänderung des Tensors wird durch zusätzliche Drehmomente, welche auf das Rad wirken, verursacht und beeinflussen das Rad bei seiner Bewegung um die Symmetrieachse. \\
Dies hat zur Folge, dass ein ausgewuchtetes Rad \dq rund\dq läuft und ein unausgewuchtetes Rad \dq eiert\dq{} (unregelmäßiges rotieren).