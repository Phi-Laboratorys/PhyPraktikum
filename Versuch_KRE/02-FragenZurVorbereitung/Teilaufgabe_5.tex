% Charlotte Geiger - Manuel Lippert - Leonard Schatt
% Physikalisches Praktikum

% Teilaufgabe 5

\section{Präzessionsfrequenz eines Kreisels}

Die Präzessionswinkelgeschwindigkeit $\omega_p$ lässt sich allgemein schreiben 
\begin{align}
    \omega_p = \frac{d\varphi}{dt}~.
\end{align}
Nun kann der infinitesmale Drehwinkel $d\varphi$ in Beziehung zu der Drehimpulsänderung $dL$ gesetzt werden
\begin{gather}
    d\varphi = \frac{dL}{L} = \frac{M}{L} dt~~\text{,wobei}~~dL = M dt~.\\
    \Rightarrow \frac{d\varphi}{dt} = \frac{M}{L} = \omega_p
\end{gather}
$M$ ist hierbei das äußere entsehende Drehmoment und $L$ der Drehimpuls, welcher durch die Rotation um die 3. Hauptachse entsteht. Daraus folgt
\begin{gather}
    M = \vect{r} \times \vect{F} = \abs*{\vect{r}}\abs*{\vect{g}}m\sin(\alpha) = mgl~~\text{,wobei}~~\sin(\alpha)=\frac{\abs*{\vect{l}}}{\abs*{\vect{r}}}\\
    L = J_3\omega_3\\
    \Rightarrow \omega_p = \frac{mgl}{J_3\omega_3}
\end{gather}
Hieraus folgt, dass $\omega_p$ unabhängig vom Neigungswinkel zwischen Figurenachse und der Horizontalen ist. Durch betrachten der Abbildung und (2.13) wird offensichtlich, dass die Richtung der Winkelgeschwindigkeit der Präzession $\omega_p$ nur von der Richtung des Drehimpuls abhängt, da das Drehmoment $M$ immer senkrecht auf die Vektoren $\vect{l}$,$\vect{r}$ und $\vect{g}$ steht. Daraus folgt für einen in positive Richtung zeigenden Drehimpuls auch eine positive Winkelgeschwindigkeit $\omega_p$ und umgekehrt, wodurch klar wird das die Richtung von $\omega_p$ gleich $\omega_3$ der 3. Hauptachse ist. \footnote{Demtröder, Experimentalphysik 1 Mechanik und Wärme, S.148}