% Charlotte Geiger - Manuel Lippert - Leonard Schatt
% Physikalisches Praktikum

% Teilaufgabe 5

\section{Präzessionsfrequenz eines Kreisels}
Die Präzessionswinkelgeschwindigkeit $w_p$ lässt sich allgemein schreiben 
\begin{align}
    w_p = \frac{d\varphi}{dt}
\end{align}
Nun kann der infinitesmale Drehwinkel $d\phi$ in Beziehung zu der Drehimpulsänderung $dL$ gesetzt werden
\begin{gather}
    d\varphi = \frac{dL}{L} = \frac{M}{L} dt \tab\text{wobei } dL = M dt \\
    \Rightarrow \frac{d\varphi}{dt} = \frac{M}{L} = w_p
\end{gather}
$M$ ist hierbei das äußere entsehende Drehmoment und $L$ der Drehimpuls, welcher durch die Rotation um die 3. Hauptachse entsteht. Daraus folgt
\begin{gather}
    M = \vect{r} \times \vect{F} = \abs*{\vect{r}}\abs*{\vect{g}}m\sin(\alpha) = mgl \tab\text{, wobei }\sin(\alpha)=\frac{\abs*{\vect{l}}}{\abs*{\vect{r}}}\\
    L = J_3w_3\\
    \Rightarrow w_p = \frac{mgl}{J_3w_3}
\end{gather}
Hieraus folgt, dass $w_p$ unabhängig vom Neigungswinkel zwischen Figurenachse und der Horizontalen ist.\\
%Grafik
Durch betrachten der Abbildung und (2.13) wird offensichtlich, dass die Richtung der Winkelgeschwindigkeit der Präzession $w_p$ nur von der Richtung des Drehimpuls abhängt, da das Drehmoment $M$ immer senkrecht auf die Vektoren $\vect{l}$,$\vect{r}$ und $\vect{g}$ steht. Daraus folgt für einen in positive Richtung zeigenden Drehimpuls auch eine positive Winkelgeschwindigkeit $w_p$ und umgekehrt, wodurch klar wird das die Richtung von $w_p$ gleich $w_3$ der 3. Hauptachse ist.