% Charlotte Geiger - Manuel Lippert - Leonard Schatt
% Physikalisches Praktikum

% Teilaufgabe 1

\documentclass[paper=a4,bibliography=totoc,BCOR=10mm,twoside,numbers=noenddot,fontsize=11pt]{scrreprt}
\usepackage[ngerman]{babel}
\usepackage[T1]{fontenc}
\usepackage[latin1]{inputenc}
\usepackage{lmodern}
\usepackage{graphicx}
\usepackage[utf8]{inputenc}
%\usepackage[dvipdfmx]{graphicx}
\usepackage{nicefrac}
\usepackage{fancyvrb}
\usepackage{amsmath,amssymb,amstext}
\usepackage{siunitx}
\usepackage{url}
\usepackage{natbib}
\usepackage{microtype}
\usepackage[format=plain]{caption}

\parindent 0.0cm
\parskip 0.8ex plus 0.5ex minus 0.5ex


\setcounter{bottomnumber}{2}
\setcounter{topnumber}{2}
\renewcommand{\bottomfraction}{1.}
\renewcommand{\topfraction}{1.}
\renewcommand{\textfraction}{0.}


\DeclareOldFontCommand{\sc}{\normalfont\scshape}{\@nomath\sc}
\DeclareOldFontCommand{\bf}{\normalfont\scshape}{\textbf}


\pagestyle{headings}          
\graphicspath{{./bilder/}}    
\VerbatimFootnotes            

\begin{document}
\nonfrenchspacing      


\section{Trägheitsmoment $I$ eines Körpers}

Das Trägheitsmoment $I$ eines Körpers wird im Kontinuum anschaulich durch die Gleichung
\begin{equation}
    I = \int_V \mathbf{r_\perp}^2 \rho(\mathbf{r}) dV
\end{equation}
dargestellt und gibt die Trägheit eines starren Körpers gegenüber
einer Winkelgeschwindigkeitsänderung bei einer Drehung um eine vorrausgesetzte Achse an. 
Dabei ist $\mathbf{r_\perp}$ der Ortsvektor, welcher senkrecht auf $\mathbf{\omega}$ steht und 
$\rho(\mathbf{r})$ die Dichte des Körpers in Abhängigkeit zum Ortsvektor $\mathbf{r}$, wobei die Dichte $\rho$ sich
bei homogenen Körper aus den Integral ziehen lässt, da diese in diesem Fall nicht mehr vom Ortsvektor $\mathbf{r}$ abhängt.\\
Für einen starren Körper aus $N$ Massepunkten der Masse $m_i$ hat (2.1) die Form
\begin{equation}
    I = \sum_{i=1}^{N} m_i r_{i,\perp}^2
\end{equation}
\end{document}
