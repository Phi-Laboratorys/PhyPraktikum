% Charlotte Geiger - Manuel Lippert - Leonard Schatt
% Physikalisches Praktikum

% Teilaufgabe 1
   
\section{Trägheitsmoment $I$ eines Körpers}
\label{sec: Traegheitsmoment}

Das Trägheitsmoment $I$ eines Körpers mit dem Volumen $V$ wird im Kontinuum durch die Gleichung
\begin{equation}
    I = \int_V \mathbf{r_\perp}^2 \rho(\mathbf{r}) dV
\end{equation}
dargestellt und gibt die Trägheit eines starren Körpers gegenüber einer Winkelgeschwindigkeitsänderung bei einer Drehung um eine vorrausgesetzte Achse an. Dabei ist $\mathbf{r_\perp}$ der Ortsvektor, welcher senkrecht auf $\mathbf{\omega}$ steht und $\rho(\mathbf{r})$ die Dichte des Körpers in Abhängigkeit zum Ortsvektor $\mathbf{r}$, wobei die Dichte $\rho$ sich bei homogenen Körper aus den Integral ziehen lässt, da diese in diesem Fall nicht mehr vom Ortsvektor $\mathbf{r}$ abhängt.\\
Für einen starren Körper aus $N$ Massepunkten der Masse $m_i$ hat (2.1) die Form
\begin{equation}
    I = \sum_{i=1}^{N} m_i r_{i,\perp}^2~.\text{\footnotemark}
\end{equation}
\footnotetext{\url{https://de.wikipedia.org/wiki/Tr%C3%A4gheitsmoment}}
