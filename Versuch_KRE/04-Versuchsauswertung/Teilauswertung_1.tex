% Charlotte Geiger - Manuel Lippert - Leonard Schatt
% Physikalisches Praktikum

% Teilauswertung 1

\section{Qualitative Beobachtung verschiedener Kreiselbewegungen}

\begin{itemize}
    \item Diskussion Beobachtung Stroboskop
    \item Bewegung Figurenachse im L-System und Drehimpulsachse bei Nutation
    \item Entspricht Bewegungsrichtung ihren Erwartungen
    \item Folge von $J_1 \approx J_3$
\end{itemize}

Im Nachhinein wurde uns bewusst, dass wir einige Fehler bei der in der quantitativen Beobactung gemacht haben. 
Dazu muss erstmal ein Fehler im Protokoll ausgebessert werden. Die Nutation ist nicht wie in Teil 1 fälschlicherweise 
behauptet, sondern gegen den Uhrzeigersinn, was später in der qualitativen Beobachtung auffiel. Nun zu den Beobachtungen.
\begin{itemize}
    \item Ohne Stroboskop:\\
    Nach dem Andrehen wurden auf den Kreisel mit den Fingern kräfte ausgeübt. Dabei war auffällig, dass je schneller der Kreisel sich dreht, desto stärker waren die Ausweichbewegungen.
    Vorallem bei langsamen Drehzahlen ist die Reaktion des Kreisels schwach. Dabei war in der qualitativen Messung noch nicht klar ersichtlich, ob das stärkere Ausweichen 
    von der höheren Drehfrequenz und dem damit verbundenen stärkeren "Abrollen" an dem Finger oder von der Präzession kommt. Die auftretende Ausweichgewegung war jedoch immer 
    senkrecht zu Rotationsachse und der Richtung in die die Kraft wirkt. \\
    Im Folgenden wurde das Gewicht an der Achse des Kreisel angebracht. Dabei konnte beobachtet werden, dass wenn der Kreisel geneigt war und das Gewicht an ihm zieht, er auch wieder 
    eine Ausweichbewegung startet. In diesem Fall kann man sehr schön die Richtung der einwirkenden Kraft mit den Richtungen des Drehimpulses und der resultierenden Kraft beoabchten.
    Die durch die Gravitation entstehende Kraft $F_{Grav}$ kann aufgespalten werden in einen Anteil $F_{\bot}$, der senkrecht zur Figurenachse ist, und einen der parallel Anteil $F_{\| }$. 
    Die Verhältnisser, in der sich die Gewichtskraft aufteilt, werden durch den Winkel der Drehachse und der Horizontalen. Wir haben zwei Ausrichtungen ausprobiert. Auffällig war. dass die 
    Kreisbewegung schneller zu sein schien, wenn die Drehachse parallel zu Horizont war als wenn der Kreisel in der $45\circ$-Position stand. Dabei gilt:
    \begin{equation*}
        F_{\bot} = F_{Grav} \cdot cos(\phi)
    \end{equation*}
    wobei $\phi$ der Winkel zwischen dem Lot und Drehachse.
    Daran ist klar zu erkennen, dass die Stärke der Präzision von der Größe der auf den Kreisel wirkenden Kraft abhängt.
\end{itemize}
