% Charlotte Geiger - Manuel Lippert - Leonard Schatt
% Physikalisches Praktikum

% Teilaufgabe 2

\section{Trägheitstensor $\mathbf{J}$}

Der Drehimpuls eines Körpers der Masse m ausgehen vom Ursprung des Koordinatensystems mit Ortsvektor $\mathbf{r}$ und Geschwindigkeitvektor
$\mathbf{v}$ ist angegeben durch die Gleichung
\begin{equation}
    \mathbf{L}=m\mathbf{r}\times\mathbf{v}
\end{equation}
Weiterhin gilt bei Rotation eines starren Körpers in einen beliebigen Koordinatensystems ein Zusammenhang zwischen Drehimpuls $\mathbf{L}$
und Winkelgeschwindigkeit $\mathbf{w}$ der Form
\begin{equation}
    \mathbf{L}=\underline{J}\mathbf{w}
\end{equation}