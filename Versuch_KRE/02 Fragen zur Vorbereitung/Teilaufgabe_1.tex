% Charlotte Geiger - Manuel Lippert - Leonard Schatt
% Physikalisches Praktikum

% Teilaufgabe 1

\section{Trägheitsmoment $I$ eines Körpers}

Das Trägheitsmoment $I$ eines Körpers wird im Kontinuum anschaulich durch die Gleichung
\begin{equation}
    I = \int_V \mathbf{r_\perp}^2 \rho(\mathbf{r}) dV
\end{equation}
dargestellt und gibt die Trägheit eines starren Körpers gegenüber
einer Winkelgeschwindigkeitsänderung bei einer Drehung um eine vorrausgesetzte Achse. 
Dabei ist $\mathbf{r_\perp}$ der Abstand, welcher senkrecht auf $\mathbf{\omega}$ steht und 
$\rho(\mathbf{r})$ die Dichte des Körpers in Abhängigkeit zum Abstand $\mathbf{r}$.\\
Für einen starren Körper aus $N$ Massepunkten hat die Gleichung (2.1) die Form
\begin{equation}
    I = \sum_{i=1}^{N} m_i r_{i,\perp}^2
\end{equation}
