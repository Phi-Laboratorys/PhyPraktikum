\documentclass[paper=a4,bibliography=totoc,BCOR=10mm,twoside,numbers=noenddot,fontsize=11pt]{article}
\usepackage[ngerman]{babel}
\usepackage[T1]{fontenc}
\usepackage[utf8]{inputenc}         
\usepackage{lmodern}
\usepackage{graphicx}
\usepackage{fancyvrb}
\usepackage{amsmath,amssymb,amstext}
\usepackage{url}
\usepackage{natbib}
\usepackage{textpos}
\usepackage{geometry} 
\usepackage{xcolor}


\begin{document}
\"a \\
Die einen dünsteten Hühnchen, die anderen überführten Betrüger durch Überprüfung der Bücher: Beim Ü-Gewinnspiel sprühten die Fünkchen! Es gibt Kochrezepte mit Würstchen und Brühe, Reiserouten von Günzburg nach Düren und unverblümte Schilderungen zügelloser Lüste.


\section{Einleitung}
\subparagraph{hallo}



Gleichungen:
\begin{equation}
5\geq3=\omega+3
\end{equation}
\begin{align}
27\\
29 \cdot x^{26}_3\\
47 \omega_3\\
\omega_n = \frac{J_3 - J_1}{J_1} \cdot \omega_3 = konstant\\
\Biggl( \omega_n = \frac{J_3 - J_1}{J_1} \cdot \omega_3 = konstant \Biggr)
\end{align}
hallo Name ist $134\alpha$
\end{document}
